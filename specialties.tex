
\usection{Cryptek Conclaves}

When taking a Cryptek Conclave, a \textbf{Discipline} must be taken from the list below, which grants a number of options, abilities, and restrictions to the unit.

\usubsection{Harbingers of Despair \hrulefill X pts}

Psychomancers must take an \quickref{Abyssal Staff} when selecting the Harbingers of Despair as their Discipline.

\usubsubsection{Abyssal Staff}
\label{Abyssal Staff}
\noindent
\begin{tabular}{||m{130pt} m{10pt} m{31pt} m{55pt} m{12pt} m{12pt} m{210pt}||}
	\hline
	Name & & Range & Type & S & AP & Abilities \\
	\hline
	\quickref{Abyssal Staff} (Shooting) & & Template & Assault 1 & 8 & 1 & Shroud of Despair \\
	\quickref{Abyssal Staff} (Melee) & & — & Melee & 8 & 1 & Shroud of Despair \\
	\hline
\end{tabular}

\textbf{Shroud of Despair:} To Wound rolls are made against the target's Leadership rather than Toughness. The attack has no effect against Vehicles. Successful wounds against Dreadnoughts and Automata must be re-rolled.

\usubsubsection{Atavindicator \hrulefill X pts}

The bearer can activate the Atavindicator at the end of their Movement, selecting an enemy unit that is not a Vehicle within 18". The targeted unit must roll a Leadership Check with a -4 penalty. Failure causes each model in the unit to automatically hit itself with a S +1 AP — melee attack.

\usubsubsection{Nightmare Shroud \hrulefill X pts} 

The bearer gains the Fear (1) rule. Additionally, the Shroud may be used during the Shooting Phase instead of firing a weapon. Choose an enemy unit within 18" of the bearer. That unit must immediately take a Morale Check.

\usubsubsection{Veil of Darkness \hrulefill X pts}

The bearer of the Veil of Darkness has the Deep Strike special rule. In addition, once per game, at the start of any friendly Movement phase, the bearer can use the Veil of Darkness to remove himself and his unit from the table, even if they are locked in combat. They then immediately arrive anywhere on the board using the rules for Deep Strike.



\usubsection{Harbingers of Destruction \hrulefill X pts}

Plasmancers must take an \quickref{Eldritch Lance} when selecting the Harbingers of Destruction as their Discipline.

\usubsubsection{Eldritch Lance}
\label{Eldritch Lance}
\noindent
\begin{tabular}{||m{130pt} m{10pt} m{31pt} m{55pt} m{12pt} m{12pt} m{210pt}||}
	\hline
	Name & & Range & Type & S & AP & Abilities \\
	\hline
	\quickref{Eldritch Lance} (Shooting) & & 36" & Assault 1 & 8 & 2 & Lance \\
	\quickref{Eldritch Lance} (Melee) & & — & Melee & User & 2 & Lance \\
	\hline
\end{tabular}

\usubsubsection{Gaze of Flame \hrulefill X pts}

The bearer, and its unit, are treated as being armed with Defensive grenades.

\usubsubsection{Plasmic Lance \hrulefill 0 pts}

Any Plasmancer may exchange their Eldritch Lance for a Plasmic Lance.
TODO: Probably too weak

\label{Plasmic Lance}
\noindent
\begin{tabular}{||m{130pt} m{10pt} m{31pt} m{55pt} m{12pt} m{12pt} m{210pt}||}
	\hline
	Name & & Range & Type & S & AP & Abilities \\
	\hline
	\quickref{Plasmic Lance} (Shooting) & & 18" & Assault 3 & 7 & 3 & — \\
	\quickref{Plasmic Lance} (Melee) & & — & Melee & User & 3 & — \\
	\hline
\end{tabular}

\usubsubsection{Solar Pulse \hrulefill X pts}

Once per game, at the start of any turn, the bearer can use this special rule. When he does, the Night Fighting rules are not in effect for the remainder of the turn (if they were in effect). In addition, when this special rule is used, enemy units targeting the bearer or his unit can only fire Snap Shots until the start of the bearer’s next turn.

\usubsubsection{Quantum Orb \hrulefill X pts}

Once per battle, at the start of your turn, the bearer can activate this item. If it does, select one point on the battlefield anywhere within 24" of the bearer and place a marker at that point. At the start of your next turn, resolve a S8 AP 3 Large Blast hit directly on that location.


\usubsection{Harbingers of Eternity \hrulefill X pts}

Chronomancers must take an \quickref{Aeonstave} when selecting the Harbingers of Eternity as their Discipline.

\usubsubsection{Aeonstave}
\label{Aeonstave}
\noindent
\begin{tabular}{||m{130pt} m{10pt} m{31pt} m{55pt} m{12pt} m{12pt} m{210pt}||}
	\hline
	Name & & Range & Type & S & AP & Abilities \\
	\hline
	\quickref{Aeonstave} & & — & Melee & User & — & \quickref{Entropic Strike} (6+), Chronal Charge \\
	\hline
\end{tabular}
\textbf{Chronal Charge:} Models which suffer an unsaved Wound or loses a Hull Point from this weapon loses the Fleet special rule and has its Weapon Skill, Ballistic Skill, Initiative and Attack values reduced to 1 for the remainder of the game.

\usubsubsection{Chronometron \hrulefill X pts}

A model with a Chronometron can re-roll one of its D6 rolls each phase. If the bearer is in a unit, this ability can be used to instead re-roll one of the units D6 rolls each phase.

\usubsubsection{Chronotendrils \hrulefill X pts}

The bearer's movement speed increases to 9" and they gain the Chronotendrils weapon. TODO: Consider special abilities?

\label{Chronotendrils}
\noindent
\begin{tabular}{||m{130pt} m{10pt} m{31pt} m{55pt} m{12pt} m{12pt} m{210pt}||}
	\hline
	Name & & Range & Type & S & AP & Abilities \\
	\hline
	\quickref{Chronotendrils} & & — & Melee & User & — & — \\
	\hline
\end{tabular}

\usubsubsection{Countertemporal Nanomines \hrulefill X pts}

Provide some sort of dangerous terrain / slowing / similar minefield effects. TODO: This

\usubsubsection{Entropic Lance \hrulefill X pts} \label{Entropic Lance}

Any Chronomancer may upgrade their Aeonstave to an Entropic Lance.

\noindent
\begin{tabular}{||m{130pt} m{10pt} m{31pt} m{55pt} m{12pt} m{12pt} m{210pt}||}
	\hline
	Name & & Range & Type & S & AP & Abilities \\
	\hline
	\quickref{Entropic Lance} (Shooting) & & Assault 1 & 18" & 7 & 3 & Brutal (2), \quickref{Entropic Strike} (2+) \\
	\quickref{Entropic Lance} (Melee) & & — & Melee & User & 3 & Brutal (2), \quickref{Entropic Strike} (2+) \\
	\hline
\end{tabular}

\usubsubsection{Timesplinter Cloak \hrulefill X pts}

A model with a Timesplinter Cloak has a 3+ Invulnerable save.



\usubsection{Harbingers of Storm \hrulefill X pts}

Ethermancers must take an \quickref{Voltaic Staff} when selecting the Harbingers of Storm as their Discipline.

\usubsubsection{Voltaic Staff}
\label{Voltaic Staff}
\noindent
\begin{tabular}{||m{130pt} m{10pt} m{31pt} m{55pt} m{12pt} m{12pt} m{210pt}||}
	\hline
	Name & & Range & Type & S & AP & Abilities \\
	\hline
	\quickref{Voltaic Staff} (Shooting) & & 12" & Assault 4 & 5 & — & Haywire \\
	\quickref{Voltaic Staff} (Melee) & & — & Melee & User & — & Haywire \\
	\hline
\end{tabular}

\usubsubsection{Ether Crystal \hrulefill X pts}

Any enemy unit arriving by Deep Strike within \quickref{Nodal Range} of the bearer suffers d6 S8 AP 5 hits. If they arrive within range of multiple Crystals, only increase the number of hits by 1 for each Crystal past the first.

\usubsubsection{Living Lightning \hrulefill X pts}

At the beginning of the Assault Phase, each enemy unit within \quickref{Nodal Range} of the bearer suffers 1 S8 AP 5 hit.

\usubsubsection{Metalodermal Tesla Weave \hrulefill X pts}

When an enemy unit successfully moves into assault with the Cryptek or his unit, the assaulting unit immediately suffers d6 S8 AP 5 hits.




\usubsection{Harbingers of Technomancy \hrulefill X pts}

Technomancers must take a \quickref{Staff of Light} when selecting the Haringers of Technomancy as their Discipline. Additionally, they must purchase the \quickref{Rites of Reanimation} ability.

\usubsubsection{Canoptek Cloak \hrulefill X pts}

Increase the bearer's move to 12" and it gains the Fleet (1) rule alongside the Antigrav and Light sub-type.

\usubsubsection{Canoptek Control Node \hrulefill X pts}

Increase your \quickref{Nodal Range} to 12" for the purposes of suppressing the \quickref{Soulless Hordes} trait for units with the Canoptek sub-type.

\usubsubsection{Fail-Safe Overcharger \hrulefill X pts}

Psychic power thing where you have a lot of option and roll with penalties for each you want, causing wounds on fail.

\begin{itemize}
	\item Test
\end{itemize}

\usubsubsection{Phylacterine Hive \hrulefill X pts}

Once per battle, when using your \quickref{Rites of Reanimation} ability, you may select a non-friendly unit with \quickref{Reanimation Protocols} (Such as Destroyer Cult or Flayer Virus units) to be affected.

\usubsubsection{Rites of Reanimation} \label{Rites of Reanimation}

After this model has moved, select a friendly unit with \quickref{Reanimation Protocols} within \quickref{Nodal Range} on the bearer. That unit immediately \textcolor{violet}{\hyperref[Reanimation Protocols]{reassembles}} a number of wounds equal to the number of wounds from all destroyed models (Do not include any lost wounds from non-destroyed models), but roll with a -1 modifier.



\usubsection{Harbingers of Transmogrification \hrulefill X pts}

Geomancers and Alchemists must take an \quickref{Tremorstave} when selecting the Harbingers of Transmogrification as their Discipline.

\usubsubsection{Tremorstave}
\label{Tremorstave}
\noindent
\begin{tabular}{||m{130pt} m{10pt} m{31pt} m{55pt} m{12pt} m{12pt} m{210pt}||}
	\hline
	Name & & Range & Type & S & AP & Abilities \\
	\hline
	\quickref{Tremorstave} (Shooting) & & 36" & Assault 1 & 4 & — & Blast, Pinning, Quake \\
	\quickref{Tremorstave} (Melee) & & — & Melee & User & — & Pinning \\
	\hline
\end{tabular}
\textbf{Quake:} After resolving all wounds, leave the Blast marker in place, or otherwise mark the area. This area now counts as Difficult Terrain until the start of the next turn of the player that made the attack.

\usubsubsection{Harp of Dissonance \hrulefill X pts}

\label{Harp of Dissonance}
\noindent
\begin{tabular}{||m{130pt} m{10pt} m{31pt} m{55pt} m{12pt} m{12pt} m{210pt}||}
	\hline
	Name & & Range & Type & S & AP & Abilities \\
	\hline
	\quickref{Harp of Dissonance} & & $\infty$ & Assault 1 & 6 & — & \quickref{Entropic Strike} (4+) \\
	\hline
\end{tabular}

\usubsubsection{Cryptogeometric Adjuster \hrulefill X pts}


\usection{Powers of the C'Tan} \label{Powers of the C'Tan}

\usubsubsection{General Powers}

\usubsubsection{Antimatter Meteor} \label{Antimatter Meteor}

\noindent
\begin{tabular}{||m{160pt} m{31pt} m{55pt} m{12pt} m{12pt} m{200pt}||}
	\hline
	Name & Range & Type & S & AP & Abilities \\
	\hline
	\quickref{Antimatter Meteor} (Shard) & 24" & Assault 1 & 8 & 3 & Large Blast \\
	\quickref{Antimatter Meteor} (Transcendent) & 48" & Assault 1 & 8 & 3 & Apocalyptic Blast \\
	\hline
\end{tabular}

\usubsubsection{Cosmic Fire} \label{Cosmic Fire}

\noindent
\begin{tabular}{||m{160pt} m{31pt} m{55pt} m{12pt} m{12pt} m{200pt}||}
	\hline
	Name & Range & Type & S & AP & Abilities \\
	\hline
	\quickref{Cosmic Fire} (Shard) & Template & Assault 1 & 6 & 4 & Torrent (24") \\
	\quickref{Cosmic Fire} (Transcendent) & Template & Assault 2 & 6 & 4 & Torrent (36") \\
	\hline
\end{tabular}

\usubsubsection{Entropic Touch} \label{Entropic Touch}

The C'Tan's weapons and powers gain the \quickref{Entropic Strike} trait at a level dependent on the C'Tan's level.

\textbf{Shard:} \quickref{Entropic Strike} (4+)

\textbf{Shard:} \quickref{Entropic Strike} (1+)

\usubsubsection{Moulder of Worlds} \label{Moulder of Worlds}

\noindent
\begin{tabular}{||m{160pt} m{31pt} m{55pt} m{12pt} m{12pt} m{200pt}||}
	\hline
	Name & Range & Type & S & AP & Abilities \\
	\hline
	\quickref{Moulder of Worlds} (Shard) & 24" & Assault 3 & 4 & 5 & Massive Blast, Pinning, Shell Shock (1) \\
	\quickref{Moulder of Worlds} (Transcendent) & 48" & Assault 6 & 4 & 5 & Apocalyptic Blast, Pinning, Shell Shock (1) \\
	\hline
\end{tabular}

\usubsubsection{Pyreshards} \label{Pyreshards}

\noindent
\begin{tabular}{||m{160pt} m{31pt} m{55pt} m{12pt} m{12pt} m{200pt}||}
	\hline
	Name & Range & Type & S & AP & Abilities \\
	\hline
	\quickref{Pyreshards} (Shard) & 18" & Assault 8 & 5 & — & Armourbane (Melta) \\
	\quickref{Pyreshards} (Transcendent) & 36" & Assault 16 & 5 & — & Armourbane (Melta) \\
	\hline
\end{tabular}

\usubsubsection{Sentient Singularity} \label{Sentient Singularity}

All terrain within the listed range of the C'Tan is treated as Difficult and Dangerous Terrain for enemy units. Additionally, Deep Striking enemy units arriving within range are automatically considered Disordered.

\textbf{Shard:} 6"

\textbf{Transcendent:} 12"

\usubsubsection{Seismic Assault} \label{Seismic Assault}

\noindent
\begin{tabular}{||m{160pt} m{31pt} m{55pt} m{12pt} m{12pt} m{200pt}||}
	\hline
	Name & Range & Type & S & AP & Abilities \\
	\hline
	\quickref{Seismic Assault} (Shard) & 24" & Assault 10 & 6 & 4 & Pinning \\
	\quickref{Seismic Assault} (Transcendent) & 48" & Assault 20 & 6 & 4 & Pinning \\
	\hline
\end{tabular}

\usubsubsection{Sky of Falling Stars} \label{Sky of Falling Stars}

\noindent
\begin{tabular}{||m{160pt} m{31pt} m{55pt} m{12pt} m{12pt} m{200pt}||}
	\hline
	Name & Range & Type & S & AP & Abilities \\
	\hline
	\quickref{Sky of Falling Stars} (Shard) & 24" & Assault 3 & 7 & 4 & Barrage, Large Blast \\
	\quickref{Sky of Falling Stars} (Transcendent) & 48" & Assault 6 & 7 & 4 & Apocalyptic Barrage \\
	\hline
\end{tabular}

\usubsubsection{Swarm of Spirit Dust} \label{Swarm of Spirit Dust}

The C'Tan gains Shrouded at a level dependent on its own level. Additionally, when targeted by a Shooting Attack, the range between an attacking unit and this model is considered to be a number of inches longer than actual, dependent on its level. In addition, when attacked by a weapon with the Barrage special rule a unit including one or more models with a distort field is always treated as though it was out of line of sight when scattering any attacks.

\textbf{Shard:} Shrouded (6+), +6"

\textbf{Shard:} Shrouded (5+), +9"

\usubsubsection{Time's Arrow} \label{Time's Arrow}

\noindent
\begin{tabular}{||m{160pt} m{31pt} m{55pt} m{12pt} m{12pt} m{200pt}||}
	\hline
	Name & Range & Type & S & AP & Abilities \\
	\hline
	\quickref{Time's Arrow} (Shard) & 24" & Destroyer 1 & 10 & 1 & Precision Shot (6+) \\
	\quickref{Time's Arrow} (Transcendent) & 48" & Destroyer 2 & 10 & 1 & Precision Shot (5+) \\
	\hline
\end{tabular}

\usubsubsection{Transdimensional Thunderbolt} \label{Transdimensional Thunderbolt}

\noindent
\begin{tabular}{||m{160pt} m{31pt} m{55pt} m{12pt} m{12pt} m{200pt}||}
	\hline
	Name & Range & Type & S & AP & Abilities \\
	\hline
	\quickref{Transdimensional Thunderbolt} (Shard) & 24" & Assault 1 & 9 & 1 & Tesla (6+) \\
	\quickref{Transdimensional Thunderbolt} (Transcendent) & 48" & Assault 2 & 9 & 1 & Tesla (5+) \\
	\hline
\end{tabular}

\usubsubsection{Withering Worldscape} \label{Withering Worldscape}

Whilst the C'tan Shard is on the battlefield, all Difficult terrain is Dangerous for the enemy. If the terrain is already Dangerous, the Dangerous Terrain test is failed on a 1 or a 2.

\usubsection{Specialist Powers}

\usubsubsection{Gaze of Death} \label{Gaze of Death}

In its Shooting phase, in addition to using Powers of the C’tan, this model can target one non-vehicle enemy unit within 12" to which it has line of sight. The unit suffers a number of Wounds, dependent on the level of the C'Tan. Shard level causes wounds equal to 3D6 minus the target's Leadership while Transcendent causes 4D6. These are resolved at AP2 and with the Ignores Cover special rule. If at least one unsaved Wound is inflicted, the C’tan Shard of the Nightbringer immediately regains one Wound lost earlier in the battle.

\usubsubsection{Lord of Fire (X)} \label{Lord of Fire}

All Flamer weapons (as well as Heat Rays, Burnas, Skorchas, Inferno Cannons and any other weapons which is described as using 'flame' or 'fire' as its effect or in its special rules) and weapons with the melta type fired within 12" of the C'tan Shard have a chance of exploding. Roll a D6 each time such a weapon is fired within range. On a roll equal or greater to this power's level, the weapon detonates. If carried by a non-vehicle model, the model is removed from play as a casualty with no saves. If mounted on a vehicle, it counts as Weapon Destroyed and causes the lose of a Hull Point. In either case, the shot(s) misfire and are lost.

\textbf{Shard:} Lord of Fire (6+)

\textbf{Transcendent:} Lord of Fire (5+)



\usubsubsection{Grand Illusion} \label{Grand Illusion}

After all forces have been deployed and all Scout moves have been made, roll a dice dependent on the C'Tan's level, being a D3 for Shard and D6 for Transcendent. You can immediately redeploy this many units, subject to the normal deployment rules for the mission. This power can be used to move units into or out of, reserve.


\usubsubsection{Voltaic Storm} \label{Voltaic Storm}

TODO: This

