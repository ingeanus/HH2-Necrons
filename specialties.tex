\section{Cryptek Conclave Disciplines} \label{Cryptek Conclave Discipline}

When taking a Cryptek or Cryptek Lord, a \textbf{Discipline} must be taken from the list below, which grants a number of options, abilities, and restrictions to the unit.

\subsection[Harbingers of Despair ]{Harbingers of Despair  \hrulefill X pts}

Psychomancers must take an \quickref{Abyssal Staff} when selecting the Harbingers of Despair as their Discipline.

\subsubsection{Abyssal Staff}
\label{Abyssal Staff}
\noindent
\begin{tabular}{||m{130pt} m{10pt} m{31pt} m{55pt} m{12pt} m{12pt} m{210pt}||}
	\hline
	Name & & Range & Type & S & AP & Abilities \\
	\hline
	\quickref{Abyssal Staff} (Shooting) & & Template & Assault 1 & 8 & 1 & Shroud of Despair \\
	\quickref{Abyssal Staff} (Melee) & & — & Melee & 8 & 1 & Shroud of Despair \\
	\hline
\end{tabular}

\textbf{Shroud of Despair:} To Wound rolls are made against the target's Leadership (modified by Fear) rather than Toughness. The attack has no effect against Vehicles.

\subsubsection[Atavindicator ]{Atavindicator  \hrulefill X pts}

The bearer can activate the Atavindicator at the end of their Movement: Target an enemy unit that does not have the Vehicle Unit Type within 18". The targeted unit must make a Leadership Check on 3D6. Failure causes each model in the unit to automatically hit itself with a S+1 AP — melee attack.

\subsubsection[Nightmare Shroud ]{Nightmare Shroud  \hrulefill X pts} 

The bearer gains the Fear (1) rule. Additionally, the Shroud may be used during the Shooting Phase instead of firing a weapon. Choose an enemy unit within 18" of the bearer. That unit must immediately take a Morale Check.

\subsubsection[Veil of Darkness ]{Veil of Darkness  \hrulefill X pts}

The bearer of the Veil of Darkness has the Deep Strike special rule. In addition, once per game, at the start of any friendly Movement phase, the bearer can use the Veil of Darkness to remove himself and his unit from the table, even if they are locked in combat. They then immediately arrive anywhere on the board using the rules for Deep Strike. The bearer also has the \quickref{Transpositional Defence} Advanced Reaction.



\subsection[Harbingers of Destruction ]{Harbingers of Destruction  \hrulefill X pts}

Plasmancers must take an \quickref{Eldritch Lance} when selecting the Harbingers of Destruction as their Discipline.

\subsubsection{Eldritch Lance}
\label{Eldritch Lance}
\noindent
\begin{tabular}{||m{130pt} m{10pt} m{31pt} m{55pt} m{12pt} m{12pt} m{210pt}||}
	\hline
	Name & & Range & Type & S & AP & Abilities \\
	\hline
	\quickref{Eldritch Lance} (Shooting) & & 36" & Assault 1 & 8 & 2 & Lance \\
	\quickref{Eldritch Lance} (Melee) & & — & Melee & User & 2 & Lance \\
	\hline
\end{tabular}

\subsubsection[Gaze of Flame ]{Gaze of Flame  \hrulefill X pts}

The bearer, and its unit, are treated as being armed with Defensive grenades.

\subsubsection[Plasmic Lance ]{Plasmic Lance  \hrulefill X pts}

Any Plasmancer may exchange their Eldritch Lance for a Plasmic Lance.
%TODO: Probably too weak

\label{Plasmic Lance}
\noindent
\begin{tabular}{||m{130pt} m{10pt} m{31pt} m{55pt} m{12pt} m{12pt} m{210pt}||}
	\hline
	Name & & Range & Type & S & AP & Abilities \\
	\hline
	\quickref{Plasmic Lance} (Shooting) & & 18" & Assault 3 & 7 & 3 & — \\
	\quickref{Plasmic Lance} (Melee) & & — & Melee & User & 3 & — \\
	\hline
\end{tabular}

\subsubsection[Solar Pulse ]{Solar Pulse  \hrulefill X pts}

Once per game, at the start of any turn, the bearer can use this special rule. When he does, the Night Fighting rules are not in effect for the remainder of the turn (if they were in effect). In addition, when this special rule is used, enemy units targeting the bearer or his unit can only fire Snap Shots until the start of the bearer’s next turn.

\subsubsection[Quantum Orb ]{Quantum Orb  \hrulefill X pts}

Once per battle, at the start of your turn, the bearer can activate this item. If it does, select one point on the battlefield anywhere within 24" of the bearer and place a marker at that point. At the start of your next turn, resolve a S8 AP 3 Large Blast hit directly on that location. %TODO: Reaction?


\subsection[Harbingers of Eternity ]{Harbingers of Eternity  \hrulefill X pts}

Chronomancers must take an \quickref{Aeonstave} when selecting the Harbingers of Eternity as their Discipline.

\subsubsection{Aeonstave}
\label{Aeonstave}
\noindent
\begin{tabular}{||m{130pt} m{10pt} m{31pt} m{55pt} m{12pt} m{12pt} m{210pt}||}
	\hline
	Name & & Range & Type & S & AP & Abilities \\
	\hline
	\quickref{Aeonstave} & & — & Melee & User & — & \quickref{Entropic Strike} (4+), Chronal Charge \\
	\hline
\end{tabular}
\textbf{Chronal Charge:} Models which suffer an unsaved Wound or loses a Hull Point from this weapon loses the Fleet special rule and has its Weapon Skill, Ballistic Skill, Initiative and Attack values reduced to 1 for the remainder of the game.

\subsubsection[Chronometron ]{Chronometron  \hrulefill X pts}

A model with a Chronometron can re-roll one of its D6 rolls each phase alongside granting a 6+ Invulnerable Save. If the bearer is in a unit, this ability can be used to instead re-roll one of the units D6 rolls each phase and provides the 6+ Invulnerable Save to the attached unit as well. In addition, the bearer may make use of the \quickref{Strategical Timeweaver} Advanced Reaction.

\subsubsection[Chronotendrils ]{Chronotendrils  \hrulefill X pts}

The bearer's movement speed increases to 9" and they gain the Hammer of Wrath (3) ability.

\subsubsection[Countertemporal Nanomines ]{Countertemporal Nanomines  \hrulefill X pts}

Charges made against the bearer or their attached unit are always considered Disordered Charges. In addition, when measuring range between the target unit and the charging unit, consider the range as 3" longer than the actual distance during the Charge Sub-Phase.

\subsubsection[Entropic Lance ]{Entropic Lance  \hrulefill X pts} \label{Entropic Lance}

Any Chronomancer may upgrade their Aeonstave to an Entropic Lance.

\noindent
\begin{tabular}{||m{130pt} m{10pt} m{31pt} m{55pt} m{12pt} m{12pt} m{210pt}||}
	\hline
	Name & & Range & Type & S & AP & Abilities \\
	\hline
	\quickref{Entropic Lance} (Shooting) & & Assault 1 & 18" & 7 & 3 & Brutal (2), \quickref{Entropic Strike} (2+) \\
	\quickref{Entropic Lance} (Melee) & & — & Melee & User & 3 & Brutal (2), \quickref{Entropic Strike} (2+) \\
	\hline
\end{tabular}

\subsubsection[Timesplinter Cloak ]{Timesplinter Cloak  \hrulefill X pts}

A model with a Timesplinter Cloak has a 3+ Invulnerable save.

\subsection[Harbingers of Storm ]{Harbingers of Storm  \hrulefill X pts}

Ethermancers must take an \quickref{Voltaic Staff} when selecting the Harbingers of Storm as their Discipline.

\subsubsection{Voltaic Staff}
\label{Voltaic Staff}
\noindent
\begin{tabular}{||m{130pt} m{10pt} m{31pt} m{55pt} m{12pt} m{12pt} m{210pt}||}
	\hline
	Name & & Range & Type & S & AP & Abilities \\
	\hline
	\quickref{Voltaic Staff} (Shooting) & & 12" & Assault 4 & 5 & — & Haywire \\
	\quickref{Voltaic Staff} (Melee) & & — & Melee & User & — & Haywire \\
	\hline
\end{tabular}

\subsubsection[Ether Crystal ]{Ether Crystal  \hrulefill X pts}

Any enemy unit arriving by Deep Strike within \quickref{Nodal Range} of the bearer suffers d6 S8 AP 5 hits. If they arrive within range of multiple Crystals, only increase the number of hits by 1 for each Crystal past the first.

\subsubsection[Living Lightning ]{Living Lightning  \hrulefill X pts}

At the beginning of the Assault Phase, each enemy unit within \quickref{Nodal Range} of the bearer suffers 1 S8 AP 5 hit. TODO: Reaction?

\subsubsection[Metalodermal Tesla Weave ]{Metalodermal Tesla Weave  \hrulefill X pts}

When an enemy unit successfully moves into assault with the Cryptek or his unit, the assaulting unit immediately suffers d6 S8 AP 5 hits.




\subsection[Harbingers of Technomancy ]{Harbingers of Technomancy  \hrulefill +25 pts}

Technomancers must take a \quickref{Staff of Light} when selecting the Harbingers of Technomancy as their Discipline. Additionally, they must purchase the \quickref{Rites of Reanimation} ability.

\subsubsection[Canoptek Cloak ]{Canoptek Cloak  \hrulefill +5 pts}

Increase the bearer's move to 12" and it gains the Fleet (1) rule alongside the Floating and Light sub-type.

\subsubsection[Canoptek Control Node ]{Canoptek Control Node  \hrulefill +12 pts}

Double your \quickref{Nodal Range} when determining whether a friendly Necron unit with the Canoptek Sub-Type is within it. TODO: Reaction to shoot back better?

\subsubsection[Fail-Safe Overcharger ]{Fail-Safe Overcharger  \hrulefill +10 pts}

At the start of the Movement Phase, this model may give up its Shooting attacks for this turn to use this power. If you do so, make a Leadership Check. Failure causes an immediate wound to the selected unit that only Invulnerability and Damage Mitigation rolls can prevent. Success allows you to select a single friendly Necron unit with the Canoptek Sub-Type with \quickref{Nodal Range} and apply one of the following effects to that unit.

\begin{itemize}
	\item The chosen unit gains +1 BS until the end of the next turn.
	\item The chosen unit gains +1 WS until the end of the next turn.
	\item The chosen unit gains +1 A until the end of the next turn.
	\item The chosen unit gains a 6+ Invulnerability Save until the end of the next turn.
	\item The chose unit gains +3 M until the end of the next turn.
\end{itemize}

You may attempt to apply multiple options at once to the same unit or to multiple units within \quickref{Nodal Range} by taking a cumulative -1 penalty to your Leadership Check for each option after the first. The same option may be taken multiple times if selecting different units and still counts as taking multiple options. If applying options to multiple units, a failed check causes an immediate wound to each unit as described above.

\subsubsection[Phylacterine Hive ]{Phylacterine Hive  \hrulefill +5 pts}

Once per battle, when using your \quickref{Rites of Reanimation} ability, you may select a non-friendly unit with \quickref{Reanimation Protocols} (Such as Destroyer Cult or Flayer units) to be affected.

\subsubsection{Rites of Reanimation} \label{Rites of Reanimation}

After this model has moved, select a friendly unit with \quickref{Reanimation Protocols} within \quickref{Nodal Range} on the bearer. That unit immediately \textcolor{violet}{\hyperref[Reanimation Protocols]{reassembles}} a number of wounds equal to the number of lost wounds plus the number of wounds from all destroyed models, but roll with a -1 modifier if the unit is not a Dynastic Warriors Phalanx.



\subsection[Harbingers of Transmogrification ]{Harbingers of Transmogrification  \hrulefill X pts}

Geomancers and Alchemists must take an \quickref{Tremorstave} when selecting the Harbingers of Transmogrification as their Discipline.

\subsubsection[Tremorstave]{Tremorstave \hrulefill X pts}
\label{Tremorstave}
\noindent
\begin{tabular}{||m{130pt} m{10pt} m{31pt} m{55pt} m{12pt} m{12pt} m{210pt}||}
	\hline
	Name & & Range & Type & S & AP & Abilities \\
	\hline
	\quickref{Tremorstave} (Shooting) & & 36" & Assault 1 & 4 & — & Blast, Pinning, Quake \\
	\quickref{Tremorstave} (Melee) & & — & Melee & User & — & Pinning \\
	\hline
\end{tabular}
\textbf{Quake:} After resolving all wounds, leave the Blast marker in place, or otherwise mark the area. This area now counts as Difficult Terrain until the start of the next turn of the player that made the attack.

\subsubsection[Harp of Dissonance ]{Harp of Dissonance  \hrulefill X pts}

\label{Harp of Dissonance}
\noindent
\begin{tabular}{||m{130pt} m{10pt} m{31pt} m{55pt} m{12pt} m{12pt} m{210pt}||}
	\hline
	Name & & Range & Type & S & AP & Abilities \\
	\hline
	\quickref{Harp of Dissonance} & & $\infty$ & Assault 1 & 6 & — & \quickref{Entropic Strike} (4+) \\
	\hline
\end{tabular}

\subsubsection[Cryptogeometric Adjuster ]{Cryptogeometric Adjuster  \hrulefill X pts}

At the start of your Shooting Phase, select an enemy unit within 18". Until the end of your next turn, whenever that unit measures distance for Shooting attacks or Charges, treat the distance between the attacking unit and the target as 6" longer than the actual distance for Shooting Attacks and 3" longer than the actual distance for Charges.


\subsubsection[Seismic Crucible]{Seismic Crucible \hrulefill X pts}

At the start of your Shooting Phase, select an enemy unit within 18". Until the start of your next turn, if that unit has the Vehicle Unit-Type it treats all terrain as Difficult terrain, otherwise it treats all terrain as Difficult and Dangerous terrain.


\section{Powers of the C'Tan} \label{Powers of the C'Tan}

\subsection{General Powers}

\subsubsection{Antimatter Meteor} \label{Antimatter Meteor}

\noindent
\begin{tabular}{||m{160pt} m{31pt} m{55pt} m{12pt} m{12pt} m{200pt}||}
	\hline
	Name & Range & Type & S & AP & Abilities \\
	\hline
	\quickref{Antimatter Meteor} (Shard) & 24" & Assault 1 & 8 & 3 & Large Blast \\
	\quickref{Antimatter Meteor} (Transcendent) & 48" & Assault 1 & 8 & 3 & Apocalyptic Blast \\
	\hline
\end{tabular}

\subsubsection{Cosmic Fire} \label{Cosmic Fire}

\noindent
\begin{tabular}{||m{160pt} m{31pt} m{55pt} m{12pt} m{12pt} m{200pt}||}
	\hline
	Name & Range & Type & S & AP & Abilities \\
	\hline
	\quickref{Cosmic Fire} (Shard) & Template & Assault 1 & 6 & 4 & Torrent (24") \\
	\quickref{Cosmic Fire} (Transcendent) & Template & Assault 2 & 6 & 4 & Torrent (36") \\
	\hline
\end{tabular}

\subsubsection{Entropic Touch} \label{Entropic Touch}

The C'Tan's weapons and powers gain the \quickref{Entropic Strike} trait at a level dependent on the C'Tan's level.

\textbf{Shard:} \quickref{Entropic Strike} (4+)

\textbf{Transcendent:} \quickref{Entropic Strike} (1+)

\subsubsection{Moulder of Worlds} \label{Moulder of Worlds}

\noindent
\begin{tabular}{||m{160pt} m{31pt} m{55pt} m{12pt} m{12pt} m{200pt}||}
	\hline
	Name & Range & Type & S & AP & Abilities \\
	\hline
	\quickref{Moulder of Worlds} (Shard) & 24" & Assault 3 & 4 & 5 & Massive Blast, Pinning, Shell Shock (1) \\
	\quickref{Moulder of Worlds} (Transcendent) & 48" & Assault 6 & 4 & 5 & Apocalyptic Blast, Pinning, Shell Shock (1) \\
	\hline
\end{tabular}

\subsubsection{Pyreshards} \label{Pyreshards}

\noindent
\begin{tabular}{||m{160pt} m{31pt} m{55pt} m{12pt} m{12pt} m{200pt}||}
	\hline
	Name & Range & Type & S & AP & Abilities \\
	\hline
	\quickref{Pyreshards} (Shard) & 18" & Assault 8 & 5 & — & Armourbane (Melta) \\
	\quickref{Pyreshards} (Transcendent) & 36" & Assault 16 & 5 & — & Armourbane (Melta) \\
	\hline
\end{tabular}

\subsubsection{Sentient Singularity} \label{Sentient Singularity}

All terrain within the listed range of the C'Tan is treated as Difficult and Dangerous Terrain for enemy units. Additionally, Deep Striking enemy units arriving within range are automatically considered Disordered.

\textbf{Shard:} 6"

\textbf{Transcendent:} 12"

\subsubsection{Seismic Assault} \label{Seismic Assault}

\noindent
\begin{tabular}{||m{160pt} m{31pt} m{55pt} m{12pt} m{12pt} m{200pt}||}
	\hline
	Name & Range & Type & S & AP & Abilities \\
	\hline
	\quickref{Seismic Assault} (Shard) & 24" & Assault 10 & 6 & 4 & Pinning \\
	\quickref{Seismic Assault} (Transcendent) & 48" & Assault 20 & 6 & 4 & Pinning \\
	\hline
\end{tabular}

\subsubsection{Sky of Falling Stars} \label{Sky of Falling Stars}

\noindent
\begin{tabular}{||m{160pt} m{31pt} m{55pt} m{12pt} m{12pt} m{200pt}||}
	\hline
	Name & Range & Type & S & AP & Abilities \\
	\hline
	\quickref{Sky of Falling Stars} (Shard) & 24" & Assault 3 & 7 & 4 & Barrage, Large Blast \\
	\quickref{Sky of Falling Stars} (Transcendent) & 48" & Assault 6 & 7 & 4 & Apocalyptic Barrage \\
	\hline
\end{tabular}

\subsubsection{Swarm of Spirit Dust} \label{Swarm of Spirit Dust}

The C'Tan gains Shrouded at a level dependent on its own level. Additionally, when targeted by a Shooting Attack, the range between an attacking unit and this model is considered to be a number of inches longer than actual, dependent on its level. In addition, when attacked by a weapon with the Barrage special rule a unit including one or more models with a distort field is always treated as though it was out of line of sight when scattering any attacks.

\textbf{Shard:} Shrouded (6+), +6"

\textbf{Transcendent:} Shrouded (5+), +9"

\subsubsection{Time's Arrow} \label{Time's Arrow}

\noindent
\begin{tabular}{||m{160pt} m{31pt} m{55pt} m{12pt} m{12pt} m{200pt}||}
	\hline
	Name & Range & Type & S & AP & Abilities \\
	\hline
	\quickref{Time's Arrow} (Shard) & 24" & Destroyer 1 & 10 & 1 & Precision Shot (6+) \\
	\quickref{Time's Arrow} (Transcendent) & 48" & Destroyer 2 & 10 & 1 & Precision Shot (5+) \\
	\hline
\end{tabular}

\subsubsection{Transdimensional Thunderbolt} \label{Transdimensional Thunderbolt}

\noindent
\begin{tabular}{||m{160pt} m{31pt} m{55pt} m{12pt} m{12pt} m{200pt}||}
	\hline
	Name & Range & Type & S & AP & Abilities \\
	\hline
	\quickref{Transdimensional Thunderbolt} (Shard) & 24" & Assault 1 & 9 & 1 & Tesla (6+) \\
	\quickref{Transdimensional Thunderbolt} (Transcendent) & 48" & Assault 2 & 9 & 1 & Tesla (5+) \\
	\hline
\end{tabular}


\subsubsection{Withering Worldscape} \label{Withering Worldscape}

Whilst the C'tan Shard is on the battlefield, all Difficult terrain is Dangerous for the enemy. If the terrain is already Dangerous, the Dangerous Terrain test is failed on a 1 or a 2.


\subsection{Transcendent Powers}

\subsubsection{Seismic Shockwave} \label{Seismic Shockwave}

The C'Tan's weapons gain the Reaping Blows (4) special rule.

\subsubsection{Storm of Heavenly Fire} \label{Storm of Heavenly Fire}

At the end of the C'tan's Movement phase, place a Large Blast marker centered over the model. All models under the marker (friend and foe, other than the C'tan) immediately suffer a single Strength 6 AP 3 hit with the Ignores Cover special rule.

\subsubsection{Transdimensional Maelstrom} \label{Transdimensional Maelstrom}

\noindent
\begin{tabular}{||m{160pt} m{31pt} m{55pt} m{12pt} m{12pt} m{200pt}||}
	\hline
	Name & Range & Type & S & AP & Abilities \\
	\hline
	\quickref{Transdimensional Maelstrom} (Transcendent) & 36" & Heavy 1 & — & — & Large Blast, Vortex \\
	\hline
\end{tabular}

\subsubsection{Transliminal Slide} \label{Transliminal Slide}

Instead of moving normally, the C'Tan can choose to move 18" in a straight line, ignoring intervening models and terrain. Any models passed over (fried or foe) suffer a Strength 6 AP — Destroyer hit. The C'tan cannot charge in the same turn it uses this ability.

\subsubsection{Wave of Withering} \label{Wave of Withering}

\noindent
\begin{tabular}{||m{160pt} m{31pt} m{55pt} m{12pt} m{12pt} m{200pt}||}
	\hline
	Name & Range & Type & S & AP & Abilities \\
	\hline
	\quickref{Wave of Withering} (Transcendent) & Hellstorm & Destroyer 1 & 9 & 1 & — \\
	\hline
\end{tabular}

\subsection{Specialist Powers}

\subsubsection{Gaze of Death} \label{Gaze of Death}

In its Shooting phase this model can target one non-vehicle enemy unit within 12" to which it has line of sight. Roll a number of dice determined by the C'Tan's level: a Shard rolls 3D6 and a Transcendent rolls 4D6. The enemy unit suffers a number of wounds equal to the rolled total minus their Leadership. These are resolved at AP2 and with the Ignores Cover special rule. If at least one unsaved Wound is inflicted, the C’tan immediately regains one Wound lost earlier in the battle.

\subsubsection{Lord of Fire (X)} \label{Lord of Fire}

All Flamer weapons (as well as Heat Rays, Burnas, Skorchas, Inferno Cannons and any other weapons which is described as using 'flame' or 'fire' as its effect or in its special rules) and weapons with the melta type fired within 12" of the C'Tan have a chance of exploding. 

Roll a D6 each time such a weapon is fired within range. On a roll equal or greater to this power's level, the weapon detonates. If carried by a non-vehicle model, the model immediately suffers a single Wound with an AP value equal to that of the weapon that was used to attack (Armour Saves, Invulnerable Saves and Feel No Pain rolls can be taken, but not Cover Saves or Shrouded rolls) – this Wound cannot be allocated to any other model in the unit. A Vehicle instead rolls an additional D6. If this roll results in a 1 or 2, the Vehicle suffers a Glancing Hit.

\textbf{Shard:} Lord of Fire (6+)

\textbf{Transcendent:} Lord of Fire (5+)


\subsubsection{Gaze of the Abyss} \label{Gaze of the Abyss}

If this model would be targeted by a Shooting Attack or Charge, the attacking unit must immediately make a Morale Test. Additionally, the C'Tan has the Fear ability at a level dependent on its level.

\textbf{Shard:} Fear (2)

\textbf{Transcendent:} Fear (3)


\subsubsection{Grand Illusion} \label{Grand Illusion}

After all forces have been deployed and all Scout moves have been made, roll a dice dependent on the C'Tan's level: a D3 for Shard and D6 for Transcendent. You can immediately redeploy this many units, subject to the normal deployment rules for the mission. This power can be used to move units into or out of, reserve.


\subsubsection{Voltaic Storm} \label{Voltaic Storm}

After this model has completed its movement, target an enemy Vehicle, Dreadnought, or Automata unit within 18" with this power. Roll a D6 to determine the effect, dependent on the C'Tan's level.

\begin{tabular}{|Z{15pt} Z{450pt}|}
	\hline
	D6 & Result \\
	\hline
	1 & No Effect \\
	2-3 & A Vehicle Unit Type that is part of the target unit suffers 1 Glancing Hit if Shard Level, or 2 if Transcendent level. Any other model suffers 1 or 2 Wounds respectively. Only Invulnerable Saves or Damage Mitigation rolls may be taken against these. \\
	4-5 &  A Vehicle Unit Type that is part of the target unit suffers 1 Penetrating Hit if Shard Level, or 2 if Transcendent level. Any other model suffers 1 or 2 Wounds respectively. Only Invulnerable Saves or Damage Mitigation rolls may be taken against these. Re-roll results of Crew Stunned and Explodes. \\
	6 &  A Vehicle Unit Type that is part of the target unit suffers 1 Penetrating Hi, or 2 if Transcendent level. Any other model suffers 1 or 2 Wounds respectively. Only Invulnerable Saves or Damage Mitigation rolls may be taken against these. \\
	\hline
\end{tabular}

\subsection{C'Tan Special Rules}

\subsubsection{Drain Life} \label{Drain Life}

Damage Mitigation rolls cannot be taken for wounds caused by this model.

\subsubsection{Flaming Vessel} \label{Flaming Vessel}

At the start of the Fight sub-phase, center a 5" Large Blast Template on this model. Each unit, except for the C'Tan, suffers a S6 AP 5 Armourbane (Melta) hit for each model underneath the template.
 
\subsubsection{Matter Absorption} \label{Matter Absorption}

At the end of the turn, if an enemy Vehicle or Dreadnought was destroyed as a result of an attack made by this model, the Void Dragon immediately make a check against this model's It Will Not Die ability for each model destroyed. If successful, it regains a lost Wound and remove any wrecks for the respective vehicle from play.

\subsubsection{Misdirection} \label{Misdirection}

Attacks made against this model suffer a -1 penalty to BS and WS. When targeted by a Shooting Attack, the range between an attacking unit and this unit is considered to be 6" further than the actual range between the two units In addition, when attacked by a weapon with the Barrage special rule, this model is always treated as thought it was out of light of sight when scattering any attacks.

\subsubsection{Unfathomable Horror} \label{Unfathomable Horror}

When an enemy unit is called to take a Morale test caused by this model, enemy models with the Fearless special rule are treated as instead having the Stubborn special rule, and enemy models with the Stubborn special rule are treated as not having that special rule.

\section{Reactions}

\subsection{Ethereal Interception} \label{Ethereal Interception}

If this unit is in Deep-Strike reserves, this reaction may be made after an enemy units arrives from Reserves and has finished all of their movement. The reacting unit immediately arrives from Reserves, following the rules for Deep Strike Assaults. Then, if it within Line of Sight of the triggering unit, it may make a Shooting attack against the triggering unit. This reaction may only be used once per turn for each unit with it. \\

\subsection{Strategical Timeweaver} \label{Strategical Timeweaver}

When an opponent declares a Shooting Attack or Fights this unit or a unit it is attached to, this reaction may be declared. The triggering unit must re-roll all successful To Hit and To Wound rolls against the reacting unit until the end of the attack, keeping the second result. This re-roll occurs even if the triggering unit's rules (e.g. Shred, Hatred, etc.) have already re-rolled these dice. This reaction may only be used once per game for each unit that is able to make use of this reaction.

