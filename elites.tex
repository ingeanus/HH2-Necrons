\usubsection{Elites}

\usubsubsection{Pariah Lychguard}

\noindent
\begin{tabular}{||m{10pt} m{95pt} m{30pt} m{11pt} m{11pt} m{11pt} m{11pt} m{11pt} m{11pt} m{11pt} m{11pt} m{11pt} m{11pt} m{125pt}||}
	\hline
	No & Name & & M & WS & BS & S & T & W & I & A & LD & Sv & Type \\
	\hline
	5 & Pariah Lychguard & X pts & 7" & 4 & 4 & 5 & 5 & 1 & 2 & 1 & 10 & 3+ & Infantry (Anathema)\\
	\hline
	\hline
	\multicolumn{14}{||Z{532 pt}||}{May include up to 5 additional Pariah Lychguard for X pts/model.}\\	
	\multicolumn{14}{||Z{532 pt}||}{Dedicated Transport: May use a Night Scythe with Teleportation Reserves, as a dedicated transport.}\\	
	\hline
	\hline
	\multicolumn{14}{||Z{532 pt}||}{Wargear: \quickref{Warscythe}}\\
	\multicolumn{14}{||Z{532 pt}||}{Wargear Options:} \\
	\multicolumn{14}{||Z{532 pt}||}{\begin{itemize}
			\item The entire unit may upgrade their \quickref{Warscythe} to include a built-in \quickref{Gauss Blaster} \hrulefill 5 pts
			\item The entire unit may replace their \quickref{Warscythe} with a \quickref{Hyperphase Sword} and \quickref{Dispersion Shield} \hrulefill 10 pts
		\end{itemize}} \\
	\hline
\end{tabular}

\noindent
\begin{tabular}{||m{110pt} m{30pt} m{31pt} m{55pt} m{12pt} m{12pt} m{210pt}||}
	\hline
	Name & & Range & Type & S & AP & Abilities \\
	\hline
	\quickref{Hyperphase Sword} & X pt & — & Melee & User & 3 & Rending (5+) \\
	\quickref{Warscythe} & x pts& — & Melee & +2 & 2 & Armourbane (Melee), Two-Handed \\
	\quickref{Gauss Blaster} & x pts& 24" & Rapid Fire 1 & 5 & 4 & \quickref{Gauss} (6+) \\
	\hline
\end{tabular}

\noindent
\begin{tabular}{||m{532pt}||}
	\hline
	Abilities \\
	\hline
	\quickref{Awakening Protocols} (Silver), Fearless, \quickref{Living Metal}, \quickref{Reanimation Protocols} \\
	\hline
\end{tabular}


\newpage
\usubsubsection{Royal Lychguard}

\noindent
\begin{tabular}{||m{10pt} m{95pt} m{30pt} m{11pt} m{11pt} m{11pt} m{11pt} m{11pt} m{11pt} m{11pt} m{11pt} m{11pt} m{11pt} m{125pt}||}
	\hline
	No & Name & & M & WS & BS & S & T & W & I & A & LD & Sv & Type \\
	\hline
	5 & Royal Lychguard & X pts & 7" & 4 & 4 & 5 & 5 & 2 & 2 & 1 & 10 & 3+ & Infantry\\
	\hline
	\hline
	\multicolumn{14}{||Z{532 pt}||}{May include up to 5 additional Pariah Lychguard for X pts/model.}\\	
	\multicolumn{14}{||Z{532 pt}||}{Dedicated Transport: May use a Night Scythe with Teleportation Reserves, as a dedicated transport.}\\	
	\hline
	\hline
	\multicolumn{14}{||Z{532 pt}||}{Wargear: \quickref{Warscythe}}\\
	\multicolumn{14}{||Z{532 pt}||}{Wargear Options:} \\
	\multicolumn{14}{||Z{532 pt}||}{\begin{itemize}
			\item The entire unit may upgrade their \quickref{Warscythe} to include a built-in \quickref{Gauss Blaster} \hrulefill 5 pts
			\item The entire unit may replace their \quickref{Warscythe} with a \quickref{Hyperphase Sword} and \quickref{Dispersion Shield} \hrulefill 10 pts
	\end{itemize}} \\
	\hline
\end{tabular}

\noindent
\begin{tabular}{||m{110pt} m{30pt} m{31pt} m{55pt} m{12pt} m{12pt} m{210pt}||}
	\hline
	Name & & Range & Type & S & AP & Abilities \\
	\hline
	\quickref{Hyperphase Sword} & X pt & — & Melee & User & 3 & Rending (5+) \\
	\quickref{Warscythe} & x pts& — & Melee & +2 & 2 & Armourbane (Melee), Two-Handed \\
	\quickref{Gauss Blaster} & x pts& 24" & Rapid Fire 1 & 5 & 4 & \quickref{Gauss} (6+) \\
	\hline
\end{tabular}

\noindent
\begin{tabular}{||m{532pt}||}
	\hline
	Abilities \\
	\hline
	\quickref{Awakening Protocols} (Bronze), \quickref{Living Metal}, \quickref{Reanimation Protocols} \\
	Royal Guard: Only a single unit of Lychguard may be purchased for each Lord, Nemesor Lord, Nemesor Overlord, and/or Phaeron and are treated as their personal retinue. They take up a single Force Organisation chart choice with that Character, but do not have to be deployed with them and are treated as a separate unit during the game. In addition they count as within Nodal Command Range of their respective HQ while they are both on the table. \\
	\hline
\end{tabular}




\newpage
\usubsubsection{C'Tan Shard of Aza'gorod, the Nightbringer}

\noindent
\begin{tabular}{||m{10pt} m{95pt} m{30pt} m{11pt} m{11pt} m{11pt} m{11pt} m{11pt} m{11pt} m{11pt} m{11pt} m{11pt} m{11pt} m{125pt}||}
	\hline
	No & Name & & M & WS & BS & S & T & W & I & A & LD & Sv & Type \\
	\hline
	1 & Nightbringer & X pts & 9" & 6 & 4 & 7 & 7 & 4 & 4 & 4 & 10 & 4+ & Infantry (Monstrous)\\
	\hline
	\hline
	\multicolumn{14}{||Z{532 pt}||}{Wargear: Scythe of the Nightbringer}\\
	\hline
\end{tabular}

\noindent
\begin{tabular}{||m{140pt} m{0pt} m{31pt} m{55pt} m{12pt} m{12pt} m{210pt}||}
	\hline
	Name & & Range & Type & S & AP & Abilities \\
	\hline
	Scythe of the Nightbringer (Reaping Sweep) &  & — & Melee & User & 3 & Murderous Strike (5+) \\
	Scythe of the Nightbringer (Entropic Blow) &  & — & Melee & x2 & 2 & Murderous Strike (3+), Two-Handed \\
	\hline
\end{tabular}

\noindent
\begin{tabular}{||m{532pt}||}
	\hline
	Abilities \\
	\hline
	\quickref{Awakening Protocols} (Silver), Eternal Warrior, Fearless, \quickref{Living Metal}, \quickref{Reanimation Protocols} \\
	Necrodermis Vessel: The Nightbringer has a 4+ invulnerable save. \\
	Enslaved Star God: If this model would be removed (After \quickref{Reanimation Protocols} have been failed), roll a D6. On a 1, the shackles for the C'Tan Shard have been broken and it is now rampaging. The enemy player may return the model to a point within 3" of where it died with 1 Wound remaining. It is now treated as an enemy unit to all players, taking its turns at the beginning of its owner's turns. It will attempt to attack the closest nearby unit, preferring its owner's units on a tie. If it would be removed with rampaging, this ability does not trigger again. \\
	Drain Life: Each time this model allocates a wound to an enemy model, Damage Mitigation rolls cannot be taken for those wounds. \\
	Immune to Natural Laws: When moving, this model can move over all other models and terrain 	freely, and automatically passes Dangerous Terrain tests. However, it cannot end its move on top of other models and can only end its move on top of impassable terrain if it is possible to actually place the model on top of it. \\
	Powers of the C'Tan: The Nightbringer gains the \quickref{Gaze of Death} power alongside an additional power of choice from the list below. It uses powers at a Shard level. \\
	\begin{itemize}
		\item \quickref{Antimatter Meteor} \hrulefill X pt
		\item \quickref{Cosmic Fire} \hrulefill X pt
		\item \quickref{Entropic Touch} \hrulefill X pt
		\item \quickref{Moulder of Worlds} \hrulefill X pt
		\item \quickref{Pyreshards} \hrulefill X pt
		\item \quickref{Sentient Singularity} \hrulefill X pt
		\item \quickref{Seismic Assault} \hrulefill X pt
		\item \quickref{Sky of Falling Stars} \hrulefill X pt
		\item \quickref{Swarm of Spirit Dust} \hrulefill X pt
		\item \quickref{Time's Arrow} \hrulefill X pt
		\item \quickref{Transdimensional Thunderbolt} \hrulefill X pt
		\item \quickref{Withering Worldscape} \hrulefill X pt
	\end{itemize} \\
	\hline
\end{tabular}




\newpage
\usubsubsection{C'Tan Shard of Mephet'ran, the Deceiver}

\noindent
\begin{tabular}{||m{10pt} m{95pt} m{30pt} m{11pt} m{11pt} m{11pt} m{11pt} m{11pt} m{11pt} m{11pt} m{11pt} m{11pt} m{11pt} m{125pt}||}
\hline
No & Name & & M & WS & BS & S & T & W & I & A & LD & Sv & Type \\
\hline
1 & Deceiver & X pts & 9" & 5 & 5 & 7 & 7 & 4 & 4 & 4 & 10 & 4+ & Infantry (Monstrous)\\
\hline
\hline
\multicolumn{14}{||Z{532 pt}||}{Wargear: Golden Fists}\\
\hline
\end{tabular}

\noindent
\begin{tabular}{||m{140pt} m{0pt} m{31pt} m{55pt} m{12pt} m{12pt} m{210pt}||}
\hline
Name & & Range & Type & S & AP & Abilities \\
\hline
Golden Fists &  & — & Melee & User & 3 & Brutal (2) \\
\hline
\end{tabular}

\noindent
\begin{tabular}{||m{532pt}||}
\hline
Abilities \\
\hline
\quickref{Awakening Protocols} (Silver), Eternal Warrior, Fearless, \quickref{Living Metal}, \quickref{Reanimation Protocols} \\
Necrodermis Vessel: The Nightbringer has a 4+ invulnerable save. \\
Enslaved Star God: If this model would be removed (After \quickref{Reanimation Protocols} have been failed), roll a D6. On a 1, the shackles for the C'Tan Shard have been broken and it is now rampaging. The enemy player may return the model to a point within 3" of where it died with 1 Wound remaining. It is now treated as an enemy unit to all players, taking its turns at the beginning of its owner's turns. It will attempt to attack the closest nearby unit, preferring its owner's units on a tie. If it would be removed with rampaging, this ability does not trigger again. \\
Misdirection: Attacks made against this model suffer a -1 penalty to BS and WS. \\
Immune to Natural Laws: When moving, this model can move over all other models and terrain 	freely, and automatically passes Dangerous Terrain tests. However, it 	cannot end its move on top of other models and can only end its move on top of impassable terrain if it is possible to actually place the model on top of it. \\
Powers of the C'Tan: The Deceiver gains the \quickref{Grand Illusion} power alongside an additional power of choice from the list below. It uses powers at a Shard level. \\
\begin{itemize}
	\item \quickref{Antimatter Meteor} \hrulefill X pt
	\item \quickref{Cosmic Fire} \hrulefill X pt
	\item \quickref{Entropic Touch} \hrulefill X pt
	\item \quickref{Moulder of Worlds} \hrulefill X pt
	\item \quickref{Pyreshards} \hrulefill X pt
	\item \quickref{Sentient Singularity} \hrulefill X pt
	\item \quickref{Seismic Assault} \hrulefill X pt
	\item \quickref{Sky of Falling Stars} \hrulefill X pt
	\item \quickref{Swarm of Spirit Dust} \hrulefill X pt
	\item \quickref{Time's Arrow} \hrulefill X pt
	\item \quickref{Transdimensional Thunderbolt} \hrulefill X pt
	\item \quickref{Withering Worldscape} \hrulefill X pt
\end{itemize} \\
\hline
\end{tabular}




\newpage
\usubsubsection{C'Tan Shard of Mag'ladroth, the Void Dragon}
TODO: Maybe 5 attacks

\noindent
\begin{tabular}{||m{10pt} m{95pt} m{30pt} m{11pt} m{11pt} m{11pt} m{11pt} m{11pt} m{11pt} m{11pt} m{11pt} m{11pt} m{11pt} m{125pt}||}
\hline
No & Name & & M & WS & BS & S & T & W & I & A & LD & Sv & Type \\
\hline
1 & Deceiver & X pts & 9" & 5 & 5 & 7 & 7 & 4 & 4 & 4 & 10 & 4+ & Infantry (Monstrous)\\
\hline
\hline
\multicolumn{14}{||Z{532 pt}||}{Wargear: Spear of the Void Dragon}\\
\hline
\end{tabular}

\noindent
\begin{tabular}{||m{140pt} m{0pt} m{31pt} m{55pt} m{12pt} m{12pt} m{210pt}||}
\hline
Name & & Range & Type & S & AP & Abilities \\
\hline
Canoptek tail blades & & — & Melee & User & 4 & \\
Spear of the Void Dragon (Shooting) &  & 12" & Heavy 1 & 9 & 1 & Exoshock (5+), Lance, Line \\
Spear of the Void Dragon (Melee) &  & — & Melee & +3 & 1 & Exoshock (4+), Lance, Two-Handed \\
\hline
\end{tabular}

\noindent
\begin{tabular}{||m{532pt}||}
\hline
Abilities \\
\hline
\quickref{Awakening Protocols} (Silver), Eternal Warrior, Fearless, \quickref{Living Metal}, Preferred Enemy (Vehicles and Dreadnoughts), \quickref{Reanimation Protocols} \\
Necrodermis Vessel: The Nightbringer has a 4+ invulnerable save. \\
Enslaved Star God: If this model would be removed (After \quickref{Reanimation Protocols} have been failed), roll a D6. On a 1, the shackles for the C'Tan Shard have been broken and it is now rampaging. The enemy player may return the model to a point within 3" of where it died with 1 Wound remaining. It is now treated as an enemy unit to all players, taking its turns at the beginning of its owner's turns. It will attempt to attack the closest nearby unit, preferring its owner's units on a tie. If it would be removed with rampaging, this ability does not trigger again. \\
Matter Absorption: At the end of the turn, if an enemy Vehicle or Dreadnought was destroyed as a result of an attack made by this model, immediately make a test against this model's It Will Not Die ability for each such model. If successful, remove the destroyed enemy model from play. \\
Immune to Natural Laws: When moving, this model can move over all other models and terrain 	freely, and automatically passes Dangerous Terrain tests. However, it 	cannot end its move on top of other models and can only end its move on top of impassable terrain if it is possible to actually place the model on top of it. \\
Powers of the C'Tan: The Void Dragon gains the \quickref{Voltaic Storm} power alongside an additional power of choice from the list below. It uses powers at a Shard level. \\
\begin{itemize}
	\item \quickref{Antimatter Meteor} \hrulefill X pt
	\item \quickref{Cosmic Fire} \hrulefill X pt
	\item \quickref{Entropic Touch} \hrulefill X pt
	\item \quickref{Moulder of Worlds} \hrulefill X pt
	\item \quickref{Pyreshards} \hrulefill X pt
	\item \quickref{Sentient Singularity} \hrulefill X pt
	\item \quickref{Seismic Assault} \hrulefill X pt
	\item \quickref{Sky of Falling Stars} \hrulefill X pt
	\item \quickref{Swarm of Spirit Dust} \hrulefill X pt
	\item \quickref{Time's Arrow} \hrulefill X pt
	\item \quickref{Transdimensional Thunderbolt} \hrulefill X pt
	\item \quickref{Withering Worldscape} \hrulefill X pt
\end{itemize} \\
\hline
\end{tabular}




\newpage
\usubsubsection{C'Tan Shard of Nyadra'zatha, the Burning One}

\noindent
\begin{tabular}{||m{10pt} m{95pt} m{30pt} m{11pt} m{11pt} m{11pt} m{11pt} m{11pt} m{11pt} m{11pt} m{11pt} m{11pt} m{11pt} m{125pt}||}
	\hline
	No & Name & & M & WS & BS & S & T & W & I & A & LD & Sv & Type \\
	\hline
	1 & Burning One & X pts & 9" & 4 & 6 & 7 & 7 & 4 & 4 & 4 & 10 & 4+ & Infantry (Monstrous)\\
	\hline
	\hline
	\multicolumn{14}{||Z{532 pt}||}{Wargear: Scythe of the Nightbringer}\\
	\hline
\end{tabular}

\noindent
\begin{tabular}{||m{140pt} m{0pt} m{31pt} m{55pt} m{12pt} m{12pt} m{210pt}||}
	\hline
	Name & & Range & Type & S & AP & Abilities \\
	Voidflame Fists & & — & Melee & User & 3 & Armourbane (Melee) \\
	\hline	
	\hline
\end{tabular}

\noindent
\begin{tabular}{||m{532pt}||}
	\hline
	Abilities \\
	\hline
	\quickref{Awakening Protocols} (Silver), Eternal Warrior, Fearless, \quickref{Living Metal}, Preferred Enemy (Vehicles and Dreadnoughts), \quickref{Reanimation Protocols} \\
	Necrodermis Vessel: The Nightbringer has a 4+ invulnerable save. \\
	Enslaved Star God: If this model would be removed (After \quickref{Reanimation Protocols} have been failed), roll a D6. On a 1, the shackles for the C'Tan Shard have been broken and it is now rampaging. The enemy player may return the model to a point within 3" of where it died with 1 Wound remaining. It is now treated as an enemy unit to all players, taking its turns at the beginning of its owner's turns. It will attempt to attack the closest nearby unit, preferring its owner's units on a tie. If it would be removed with rampaging, this ability does not trigger again. \\
	Flaming Vessel: At the start of the Fight sub-phase, enemy units within 4" suffer D3 S6 AP 5 Armourbane (Melta) hits.\\
	Immune to Natural Laws: When moving, this model can move over all other models and terrain 	freely, and automatically passes Dangerous Terrain tests. However, it 	cannot end its move on top of other models and can only end its move on top of impassable terrain if it is possible to actually place the model on top of it. \\
	Powers of the C'Tan: The Burning One gains the \quickref{Lord of Fire} power alongside an additional power of choice from the list below. It uses powers at a Shard level. \\
	\begin{itemize}
		\item \quickref{Antimatter Meteor} \hrulefill X pt
		\item \quickref{Cosmic Fire} \hrulefill X pt
		\item \quickref{Entropic Touch} \hrulefill X pt
		\item \quickref{Moulder of Worlds} \hrulefill X pt
		\item \quickref{Pyreshards} \hrulefill X pt
		\item \quickref{Sentient Singularity} \hrulefill X pt
		\item \quickref{Seismic Assault} \hrulefill X pt
		\item \quickref{Sky of Falling Stars} \hrulefill X pt
		\item \quickref{Swarm of Spirit Dust} \hrulefill X pt
		\item \quickref{Time's Arrow} \hrulefill X pt
		\item \quickref{Transdimensional Thunderbolt} \hrulefill X pt
		\item \quickref{Withering Worldscape} \hrulefill X pt
	\end{itemize} \\
	\hline
\end{tabular}


\newpage
\usubsubsection{C'Tan Shard of Tsara'noga, the Outsider}

\noindent
\begin{tabular}{||m{10pt} m{95pt} m{30pt} m{11pt} m{11pt} m{11pt} m{11pt} m{11pt} m{11pt} m{11pt} m{11pt} m{11pt} m{11pt} m{125pt}||}
	\hline
	No & Name & & M & WS & BS & S & T & W & I & A & LD & Sv & Type \\
	\hline
	1 & Outsider & X pts & 9" & 5 & 5 & 7 & 7 & 4 & 4 & 4 & 10 & 4+ & Infantry (Monstrous)\\
	\hline
	\hline
	\multicolumn{14}{||Z{532 pt}||}{Wargear: Scythe of the Nightbringer}\\
	\hline
\end{tabular}

\noindent
\begin{tabular}{||m{140pt} m{0pt} m{31pt} m{55pt} m{12pt} m{12pt} m{210pt}||}
	\hline
	Name & & Range & Type & S & AP & Abilities \\
	Touch of Eternity & & — & Melee & +3 & 1 & Shroud of Despair \\
	\hline	
	\hline
\end{tabular}

\noindent
\begin{tabular}{||m{532pt}||}
	\hline
	Abilities \\
	\hline
	\quickref{Awakening Protocols} (Silver), Eternal Warrior, Fearless, \quickref{Living Metal}, Preferred Enemy (Vehicles and Dreadnoughts), \quickref{Reanimation Protocols} \\
	Shroud of Despair: To Wound rolls are made against the target's Leadership rather than Toughness. The attack has no effect against Vehicles. Successful wounds against Dreadnoughts and Automata must be re-rolled.
	Necrodermis Vessel: The Nightbringer has a 4+ invulnerable save. \\
	Enslaved Star God: If this model would be removed (After \quickref{Reanimation Protocols} have been failed), roll a D6. On a 1, the shackles for the C'Tan Shard have been broken and it is now rampaging. The enemy player may return the model to a point within 3" of where it died with 1 Wound remaining. It is now treated as an enemy unit to all players, taking its turns at the beginning of its owner's turns. It will attempt to attack the closest nearby unit, preferring its owner's units on a tie. If it would be removed with rampaging, this ability does not trigger again. \\
	Special power \\
	Immune to Natural Laws: When moving, this model can move over all other models and terrain 	freely, and automatically passes Dangerous Terrain tests. However, it 	cannot end its move on top of other models and can only end its move on top of impassable terrain if it is possible to actually place the model on top of it. \\
	Powers of the C'Tan: The Burning One gains the \quickref{Call of the Abyss} power alongside an additional power of choice from the list below. It uses powers at a Shard level. \\
	\begin{itemize}
		\item \quickref{Antimatter Meteor} \hrulefill X pt
		\item \quickref{Cosmic Fire} \hrulefill X pt
		\item \quickref{Entropic Touch} \hrulefill X pt
		\item \quickref{Moulder of Worlds} \hrulefill X pt
		\item \quickref{Pyreshards} \hrulefill X pt
		\item \quickref{Sentient Singularity} \hrulefill X pt
		\item \quickref{Seismic Assault} \hrulefill X pt
		\item \quickref{Sky of Falling Stars} \hrulefill X pt
		\item \quickref{Swarm of Spirit Dust} \hrulefill X pt
		\item \quickref{Time's Arrow} \hrulefill X pt
		\item \quickref{Transdimensional Thunderbolt} \hrulefill X pt
		\item \quickref{Withering Worldscape} \hrulefill X pt
	\end{itemize} \\
	\hline
\end{tabular}



