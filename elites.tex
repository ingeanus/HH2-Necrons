\subsection{Elites}

\subsubsection{Pariah Lychguard}

\noindent
\begin{tabular}{||m{10pt} m{95pt} m{30pt} m{11pt} m{11pt} m{11pt} m{11pt} m{11pt} m{11pt} m{11pt} m{11pt} m{11pt} m{11pt} m{125pt}||}
	\hline
	No & Name & & M & WS & BS & S & T & W & I & A & LD & Sv & Type \\
	\hline
	5 & Pariah Lychguard & X pts & 7" & 4 & 4 & 5 & 5 & 1 & 2 & 1 & 10 & 3+ & Infantry (Anathema)\\
	\hline
	\hline
	\multicolumn{14}{||Z{532 pt}||}{May include up to 5 additional Pariah Lychguard for X pts/model.}\\	
	\multicolumn{14}{||Z{532 pt}||}{Dedicated Transport: May use a Night Scythe as a dedicated transport.}\\
	\hline
	\hline
	\multicolumn{14}{||Z{532 pt}||}{Wargear: \quickref{Warscythe}}\\
	\multicolumn{14}{||Z{532 pt}||}{Wargear Options:} \\
	\multicolumn{14}{||Z{532 pt}||}{\begin{itemize}
			\item The entire unit may upgrade their \quickref{Warscythe} to include a built-in \quickref{Gauss Blaster} \hrulefill 5 pts
			\item The entire unit may replace their \quickref{Warscythe} with a \quickref{Hyperphase Sword} and \quickref{Dispersion Shield} \hrulefill 10 pts
		\end{itemize}} \\
	\hline
\end{tabular}

\noindent
\begin{tabular}{||m{110pt} m{30pt} m{31pt} m{55pt} m{12pt} m{12pt} m{210pt}||}
	\hline
	Name & & Range & Type & S & AP & Abilities \\
	\hline
	\quickref{Hyperphase Sword} & X pt & — & Melee & User & 3 & Rending (5+) \\
	\quickref{Warscythe} & x pts& — & Melee & +2 & 2 & Armourbane (Melee), Two-Handed \\
	\quickref{Gauss Blaster} & x pts& 24" & Rapid Fire 1 & 5 & 4 & \quickref{Gauss} (6+) \\
	\hline
\end{tabular}

\noindent
\begin{tabular}{||m{532pt}||}
	\hline
	Abilities \\
	\hline
	\quickref{Awakening Protocols} (Silver), Fearless, \quickref{Living Metal}, \quickref{Reanimation Protocols} \\
	\hline
\end{tabular}


\newpage
\subsubsection{Royal Lychguard}

\noindent
\begin{tabular}{||m{10pt} m{95pt} m{30pt} m{11pt} m{11pt} m{11pt} m{11pt} m{11pt} m{11pt} m{11pt} m{11pt} m{11pt} m{11pt} m{125pt}||}
	\hline
	No & Name & & M & WS & BS & S & T & W & I & A & LD & Sv & Type \\
	\hline
	5 & Royal Lychguard & X pts & 7" & 4 & 4 & 5 & 5 & 2 & 2 & 2 & 10 & 3+ & Infantry (Line)\\
	\hline
	\hline
	\multicolumn{14}{||Z{532 pt}||}{May include up to 5 additional Royal Lychguard for X pts/model.}\\	
	\multicolumn{14}{||Z{532 pt}||}{Dedicated Transport: May use a Night Scythe as a dedicated transport.}\\	
	\hline
	\hline
	\multicolumn{14}{||Z{532 pt}||}{Wargear: \quickref{Warscythe}}\\
	\multicolumn{14}{||Z{532 pt}||}{Wargear Options:} \\
	\multicolumn{14}{||Z{532 pt}||}{\begin{itemize}
			\item The entire unit may upgrade their \quickref{Warscythe} to include a built-in \quickref{Gauss Blaster} \hrulefill 5 pts
			\item The entire unit may replace their \quickref{Warscythe} with a \quickref{Hyperphase Sword} and \quickref{Dispersion Shield} \hrulefill 10 pts
	\end{itemize}} \\
	\hline
\end{tabular}

\noindent
\begin{tabular}{||m{110pt} m{30pt} m{31pt} m{55pt} m{12pt} m{12pt} m{210pt}||}
	\hline
	Name & & Range & Type & S & AP & Abilities \\
	\hline
	\quickref{Hyperphase Sword} & X pt & — & Melee & User & 3 & Rending (5+) \\
	\quickref{Warscythe} & x pts& — & Melee & +2 & 2 & Armourbane (Melee), Two-Handed \\
	\quickref{Gauss Blaster} & x pts& 24" & Rapid Fire 1 & 5 & 4 & \quickref{Gauss} (6+) \\
	\hline
\end{tabular}

\noindent
\begin{tabular}{||m{532pt}||}
	\hline
	Abilities \\
	\hline
	\quickref{Awakening Protocols} (Bronze), Chosen Warriors, \quickref{Living Metal}, \quickref{Reanimation Protocols} \\
	\textbf{Royal Guard:} Only a single unit of Lychguard may be purchased for each Lord, Nemesor Lord, Nemesor Overlord, and/or Phaeron and are treated as their personal retinue. They take up a single Force Organisation chart choice with that Character, but do not have to be deployed with them and are treated as a separate unit during the game. In addition they count as within Nodal Command Range of their respective HQ while they are both on the table. \\
	\hline
\end{tabular}




\newpage
\subsubsection{Apprentek}

\noindent
\begin{tabular}{||m{10pt} m{95pt} m{30pt} m{11pt} m{11pt} m{11pt} m{11pt} m{11pt} m{11pt} m{11pt} m{11pt} m{11pt} m{11pt} m{125pt}||}
	\hline
	No & Name & & M & WS & BS & S & T & W & I & A & LD & Sv & Type \\
	\hline
	5 & Royal Lychguard & X pts & 7" & 4 & 4 & 5 & 5 & 2 & 2 & 2 & 10 & 3+ & Infantry (Line)\\
	\hline
	\hline
	\multicolumn{14}{||Z{532 pt}||}{May include up to 5 additional Royal Lychguard for X pts/model.}\\	
	\multicolumn{14}{||Z{532 pt}||}{Dedicated Transport: May use a Night Scythe as a dedicated transport.}\\	
	\hline
	\hline
	\multicolumn{14}{||Z{532 pt}||}{Wargear: \quickref{Warscythe}}\\
	\multicolumn{14}{||Z{532 pt}||}{Wargear Options:} \\
	\multicolumn{14}{||Z{532 pt}||}{\begin{itemize}
			\item The entire unit may upgrade their \quickref{Warscythe} to include a built-in \quickref{Gauss Blaster} \hrulefill 5 pts
			\item The entire unit may replace their \quickref{Warscythe} with a \quickref{Hyperphase Sword} and \quickref{Dispersion Shield} \hrulefill 10 pts
	\end{itemize}} \\
	\hline
\end{tabular}

\noindent
\begin{tabular}{||m{110pt} m{30pt} m{31pt} m{55pt} m{12pt} m{12pt} m{210pt}||}
	\hline
	Name & & Range & Type & S & AP & Abilities \\
	\hline
	\quickref{Hyperphase Sword} & X pt & — & Melee & User & 3 & Rending (5+) \\
	\quickref{Warscythe} & x pts& — & Melee & +2 & 2 & Armourbane (Melee), Two-Handed \\
	\quickref{Gauss Blaster} & x pts& 24" & Rapid Fire 1 & 5 & 4 & \quickref{Gauss} (6+) \\
	\hline
\end{tabular}

\noindent
\begin{tabular}{||m{532pt}||}
	\hline
	Abilities \\
	\hline
	\quickref{Awakening Protocols} (Bronze), Chosen Warriors, \quickref{Living Metal}, \quickref{Reanimation Protocols} \\
	\textbf{Royal Guard:} Only a single unit of Lychguard may be purchased for each Lord, Nemesor Lord, Nemesor Overlord, and/or Phaeron and are treated as their personal retinue. They take up a single Force Organisation chart choice with that Character, but do not have to be deployed with them and are treated as a separate unit during the game. In addition they count as within Nodal Command Range of their respective HQ while they are both on the table. \\
	\hline
\end{tabular}


\newpage
\subsubsection{Canoptek Plasmacyte}

\noindent
\begin{tabular}{||m{10pt} m{95pt} m{30pt} m{11pt} m{11pt} m{11pt} m{11pt} m{11pt} m{11pt} m{11pt} m{11pt} m{11pt} m{11pt} m{125pt}||}
	\hline
	No & Name & & M & WS & BS & S & T & W & I & A & LD & Sv & Type \\
	\hline
	5 & Canoptek Plasmacyte & X pts & 9" & 3 & 3 & 4 & 5 & 1 & 2 & 1 & 10 & 4+ & Infantry (Anti-Grav, Monstrous) \\
	\hline
	\hline
	\multicolumn{14}{||Z{532 pt}||}{When taking this model, determine if it is a Destructor, Accelerator, or Reanimator.}\\	
	\multicolumn{14}{||Z{532 pt}||}{Dedicated Transport: May use a Night Scythe as a dedicated transport.}\\
	\hline
	\hline
	\multicolumn{14}{||Z{532 pt}||}{Wargear: Each model is armed with a Close Combat Weapon.} \\
	\hline
\end{tabular}

\noindent
\begin{tabular}{||m{532pt}||}
	\hline
	Abilities \\
	\hline
	\quickref{Living Metal}, \quickref{Reanimation Protocols} \\
	\textbf{Viral Construct:} For each unit with the Destroyer sub-type in your army, a Canoptek Plasmacyte Destructor may be taken without taking up a Force Org slot. For each Cryptek or Cryptek Lord in your army, a Canoptek Plasmacyte Accelerator or Canoptek Plasmacyte Reanimator can be taken without taking up a Force Org slot. \\
	\textbf{Evasion Protocols:} This unit is able to join other units as if it had the Independent Character special rule. \\
	\textbf{Infused Madness (Destructor Only):} Once per turn at the start of the \\
	\textbf{Acceleration Logis (Accelerator Only):} \\
	\textbf{Reanimation Beam (Reanimator Only):} \\
	\hline
\end{tabular}



\newpage
\subsubsection{Deathmarks}

\noindent
\begin{tabular}{||m{10pt} m{95pt} m{30pt} m{11pt} m{11pt} m{11pt} m{11pt} m{11pt} m{11pt} m{11pt} m{11pt} m{11pt} m{11pt} m{125pt}||}
	\hline
	No & Name & & M & WS & BS & S & T & W & I & A & LD & Sv & Type \\
	\hline
	5 & Deathmarks & X pts & 6" & 4 & 6 & 4 & 5 & 1 & 2 & 1 & 10 & 3+ & Infantry \\
	\hline
	\hline
	\multicolumn{14}{||Z{532 pt}||}{May include up to 5 additional Deathmarks for X pts/model.}\\	
	\multicolumn{14}{||Z{532 pt}||}{Dedicated Transport: May use a Night Scythe as a dedicated transport.}\\
	\hline
	\hline
	\multicolumn{14}{||Z{532 pt}||}{Wargear: Each model is armed with a \quickref{Synaptic Disnitegrator}.} \\
	\hline
\end{tabular}

\noindent
\begin{tabular}{||m{140pt} m{00pt} m{31pt} m{55pt} m{12pt} m{12pt} m{210pt}||}
	\hline
	Name & & Range & Type & S & AP & Abilities \\
	\hline
	\quickref{Synaptic Disintegrator} &  & 36" & Rapid Fire & 5 & 5 & Rending (5+), Pinning, Sniper \\
	\hline
\end{tabular}

\noindent
\begin{tabular}{||m{532pt}||}
	\hline
	Abilities \\
	\hline
	\quickref{Awakening Protocols} (Bronze), Deep-Strike, \quickref{Hyperspace Hunters}, \quickref{Living Metal}, \quickref{Reanimation Protocols} \\
	\textbf{Ethereal Interception:} This unit is may perform as many additional Deep Strike Assaults as desired and does not have to take part in the initial assault. If this unit is in Deep Strike Reserve, immediately after an enemy unit arrives from Deep Strike Reserve this unit may choose to immediately arrive using the rules for Deep Strike (if this unit does not enter play in this manner, make Reserve Rolls for it as normal in subsequent turns). At the end of that enemy Movement phase, any friendly Deathmarks unit that arrived on the board in this manner during that turn may fire its weapons at any enemy unit that arrived from Reserves that phase; any Deathmarks unit that does so cannot fire its weapons in its following turn. \\
	\textbf{Hyperspace Ambush:} During the player turn in which this unit arrives from Deep Strike Reserve, all shooting attacks made by the Deathmarks in this unit will wound on To Wound rolls of 2+, regardless of the victim’s Toughness. \\
	\hline
\end{tabular}


\newpage
\subsubsection{C'Tan Shard of Aza'gorod, the Nightbringer}

\noindent
\begin{tabular}{||m{10pt} m{95pt} m{30pt} m{11pt} m{11pt} m{11pt} m{11pt} m{11pt} m{11pt} m{11pt} m{11pt} m{11pt} m{11pt} m{125pt}||}
	\hline
	No & Name & & M & WS & BS & S & T & W & I & A & LD & Sv & Type \\
	\hline
	1 & Nightbringer & X pts & 9" & 6 & 4 & 7 & 7 & 5 & 4 & 4 & 10 & 4+ & Infantry (Monstrous)\\
	\hline
	\hline
	\multicolumn{14}{||Z{532 pt}||}{Wargear: Scythe of the Nightbringer}\\
	\hline
\end{tabular}

\noindent
\begin{tabular}{||m{140pt} m{0pt} m{31pt} m{55pt} m{12pt} m{12pt} m{210pt}||}
	\hline
	Name & & Range & Type & S & AP & Abilities \\
	\hline
	Scythe of the Nightbringer & & & & & & \\
	— Reaping Sweep &  & — & Melee & User & 3 & Murderous Strike (6+), Reaping Blow (4) \\
	— Entropic Blow &  & — & Melee & x2 & 2 & Brutal (3), Murderous Strike (6+),Two-Handed \\
	\hline
\end{tabular}

\noindent
\begin{tabular}{||m{532pt}||}
	\hline
	Abilities \\
	\hline
	\quickref{Awakening Protocols} (Silver), Eternal Warrior, Fearless, \quickref{Living Metal}, \quickref{Reanimation Protocols} \\
	\textbf{Necrodermis Vessel:} The C'Tan has a 4+ invulnerable save. \\
	\textbf{Enslaved Star God:} If this model would be removed (After \quickref{Reanimation Protocols} have been failed), roll a D6. On a 1, the shackles for the C'Tan Shard have been broken and it is now rampaging. The enemy player may return the model to a point within 3" of where it died with 1 Wound remaining. It is now treated as an enemy unit to all players, taking its turns at the beginning of its owner's turns. It will attempt to attack the closest nearby unit, preferring its owner's units on a tie. If it would be removed while rampaging, this ability does not trigger again. \\
	\textbf{Drain Life:} Each time this model allocates a wound to an enemy model, Damage Mitigation rolls cannot be taken for those wounds. \\
	\textbf{Immune to Natural Laws:} When moving, this model can move over all other models and terrain 	freely, and automatically passes Dangerous Terrain tests. However, it cannot end its move on top of other models and can only end its move on top of impassable terrain if it is possible to actually place the model on top of it. \\
	Powers of the C'Tan: The Nightbringer gains the \quickref{Gaze of Death} power alongside an additional power of choice from the list below. It uses powers at a Shard level. \\
	\begin{itemize}
		\item \quickref{Antimatter Meteor} \hrulefill X pt
		\item \quickref{Cosmic Fire} \hrulefill X pt
		\item \quickref{Entropic Touch} \hrulefill X pt
		\item \quickref{Moulder of Worlds} \hrulefill X pt
		\item \quickref{Pyreshards} \hrulefill X pt
		\item \quickref{Sentient Singularity} \hrulefill X pt
		\item \quickref{Seismic Assault} \hrulefill X pt
		\item \quickref{Sky of Falling Stars} \hrulefill X pt
		\item \quickref{Swarm of Spirit Dust} \hrulefill X pt
		\item \quickref{Time's Arrow} \hrulefill X pt
		\item \quickref{Transdimensional Thunderbolt} \hrulefill X pt
		\item \quickref{Withering Worldscape} \hrulefill X pt
	\end{itemize} \\
	\hline
\end{tabular}




\newpage
\subsubsection{C'Tan Shard of Mephet'ran, the Deceiver}

\noindent
\begin{tabular}{||m{10pt} m{95pt} m{30pt} m{11pt} m{11pt} m{11pt} m{11pt} m{11pt} m{11pt} m{11pt} m{11pt} m{11pt} m{11pt} m{125pt}||}
\hline
No & Name & & M & WS & BS & S & T & W & I & A & LD & Sv & Type \\
\hline
1 & Deceiver & X pts & 9" & 5 & 5 & 7 & 7 & 5 & 4 & 4 & 10 & 4+ & Infantry (Monstrous)\\
\hline
\hline
\multicolumn{14}{||Z{532 pt}||}{Wargear: Golden Fists}\\
\hline
\end{tabular}

\noindent
\begin{tabular}{||m{140pt} m{0pt} m{31pt} m{55pt} m{12pt} m{12pt} m{210pt}||}
\hline
Name & & Range & Type & S & AP & Abilities \\
\hline
Golden Fists &  & — & Melee & User & 3 & Brutal (2) \\
\hline
\end{tabular}

\noindent
\begin{tabular}{||m{532pt}||}
	\hline
	Abilities \\
	\hline
	\quickref{Awakening Protocols} (Silver), Eternal Warrior, Fearless, \quickref{Living Metal}, \quickref{Reanimation Protocols} \\
	\textbf{Necrodermis Vessel:} The C'Tan has a 4+ invulnerable save. \\
	\textbf{Enslaved Star God:} If this model would be removed (After \quickref{Reanimation Protocols} have been failed), roll a D6. On a 1, the shackles for the C'Tan Shard have been broken and it is now rampaging. The enemy player may return the model to a point within 3" of where it died with 1 Wound remaining. It is now treated as an enemy unit to all players, taking its turns at the beginning of its owner's turns. It will attempt to attack the closest nearby unit, preferring its owner's units on a tie. If it would be removed while rampaging, this ability does not trigger again. \\
	\textbf{Misdirection:} Attacks made against this model suffer a -1 penalty to BS and WS. \\
	\textbf{Immune to Natural Laws:} When moving, this model can move over all other models and terrain 	freely, and automatically passes Dangerous Terrain tests. However, it 	cannot end its move on top of other models and can only end its move on top of impassable terrain if it is possible to actually place the model on top of it. \\
	Powers of the C'Tan: The Deceiver gains the \quickref{Grand Illusion} power alongside an additional power of choice from the list below. It uses powers at a Shard level. \\
	\begin{itemize}
		\item \quickref{Antimatter Meteor} \hrulefill X pt
		\item \quickref{Cosmic Fire} \hrulefill X pt
		\item \quickref{Entropic Touch} \hrulefill X pt
		\item \quickref{Moulder of Worlds} \hrulefill X pt
		\item \quickref{Pyreshards} \hrulefill X pt
		\item \quickref{Sentient Singularity} \hrulefill X pt
		\item \quickref{Seismic Assault} \hrulefill X pt
		\item \quickref{Sky of Falling Stars} \hrulefill X pt
		\item \quickref{Swarm of Spirit Dust} \hrulefill X pt
		\item \quickref{Time's Arrow} \hrulefill X pt
		\item \quickref{Transdimensional Thunderbolt} \hrulefill X pt
		\item \quickref{Withering Worldscape} \hrulefill X pt
	\end{itemize} \\
	\hline
\end{tabular}




\newpage
\subsubsection{C'Tan Shard of Mag'ladroth, the Void Dragon}
TODO: Maybe 5 attacks

\noindent
\begin{tabular}{||m{10pt} m{95pt} m{30pt} m{11pt} m{11pt} m{11pt} m{11pt} m{11pt} m{11pt} m{11pt} m{11pt} m{11pt} m{11pt} m{125pt}||}
\hline
No & Name & & M & WS & BS & S & T & W & I & A & LD & Sv & Type \\
\hline
1 & Deceiver & X pts & 9" & 5 & 5 & 7 & 7 & 5 & 4 & 4 & 10 & 4+ & Infantry (Monstrous)\\
\hline
\hline
\multicolumn{14}{||Z{532 pt}||}{Wargear: Spear of the Void Dragon}\\
\hline
\end{tabular}

\noindent
\begin{tabular}{||m{140pt} m{0pt} m{31pt} m{55pt} m{12pt} m{12pt} m{210pt}||}
\hline
Name & & Range & Type & S & AP & Abilities \\
\hline
Canoptek tail blades & & — & Melee & User & 4 & Reaping Blow (3) \\
Spear of the Void Dragon (Shooting) &  & 12" & Heavy 1 & 9 & 1 & Exoshock (5+), Lance, Line, Torsion Crusher \\
Spear of the Void Dragon (Melee) &  & — & Melee & +3 & 1 & Exoshock (4+), Lance, Torsion Crusher, Two-Handed \\
\hline
\end{tabular}

\noindent
\begin{tabular}{||m{532pt}||}
	\hline
	Abilities \\
	\hline
	\quickref{Awakening Protocols} (Silver), Eternal Warrior, Fearless, \quickref{Living Metal}, Preferred Enemy (Vehicles and Dreadnoughts), \quickref{Reanimation Protocols} \\
	\textbf{Necrodermis Vessel:} The C'Tan has a 4+ invulnerable save. \\
	\textbf{Enslaved Star God:} If this model would be removed (After \quickref{Reanimation Protocols} have been failed), roll a D6. On a 1, the shackles for the C'Tan Shard have been broken and it is now rampaging. The enemy player may return the model to a point within 3" of where it died with 1 Wound remaining. It is now treated as an enemy unit to all players, taking its turns at the beginning of its owner's turns. It will attempt to attack the closest nearby unit, preferring its owner's units on a tie. If it would be removed while rampaging, this ability does not trigger again. \\
	\textbf{Matter Absorption:} At the end of the turn, if an enemy Vehicle or Dreadnought was destroyed as a result of an attack made by this model, immediately make a test against this model's It Will Not Die ability for each such model. If successful, remove the destroyed enemy model from play. \\
	\textbf{Immune to Natural Laws:} When moving, this model can move over all other models and terrain 	freely, and automatically passes Dangerous Terrain tests. However, it 	cannot end its move on top of other models and can only end its move on top of impassable terrain if it is possible to actually place the model on top of it. \\
	Powers of the C'Tan: The Void Dragon gains the \quickref{Voltaic Storm} power alongside an additional power of choice from the list below. It uses powers at a Shard level. \\
	\begin{itemize}
		\item \quickref{Antimatter Meteor} \hrulefill X pt
		\item \quickref{Cosmic Fire} \hrulefill X pt
		\item \quickref{Entropic Touch} \hrulefill X pt
		\item \quickref{Moulder of Worlds} \hrulefill X pt
		\item \quickref{Pyreshards} \hrulefill X pt
		\item \quickref{Sentient Singularity} \hrulefill X pt
		\item \quickref{Seismic Assault} \hrulefill X pt
		\item \quickref{Sky of Falling Stars} \hrulefill X pt
		\item \quickref{Swarm of Spirit Dust} \hrulefill X pt
		\item \quickref{Time's Arrow} \hrulefill X pt
		\item \quickref{Transdimensional Thunderbolt} \hrulefill X pt
		\item \quickref{Withering Worldscape} \hrulefill X pt
	\end{itemize} \\
	\hline
\end{tabular}




\newpage
\subsubsection{C'Tan Shard of Nyadra'zatha, the Burning One}

\noindent
\begin{tabular}{||m{10pt} m{95pt} m{30pt} m{11pt} m{11pt} m{11pt} m{11pt} m{11pt} m{11pt} m{11pt} m{11pt} m{11pt} m{11pt} m{125pt}||}
	\hline
	No & Name & & M & WS & BS & S & T & W & I & A & LD & Sv & Type \\
	\hline
	1 & Burning One & X pts & 9" & 4 & 6 & 7 & 7 & 5 & 4 & 4 & 10 & 4+ & Infantry (Monstrous)\\
	\hline
	\hline
	\multicolumn{14}{||Z{532 pt}||}{Wargear: Scythe of the Nightbringer}\\
	\hline
\end{tabular}

\noindent
\begin{tabular}{||m{140pt} m{0pt} m{31pt} m{55pt} m{12pt} m{12pt} m{210pt}||}
	\hline
	Name & & Range & Type & S & AP & Abilities \\
	Voidflame Fists & & — & Melee & User & 3 & Armourbane (Melee) \\
	\hline	
	\hline
\end{tabular}

\noindent
\begin{tabular}{||m{532pt}||}
	\hline
	Abilities \\
	\hline
	\quickref{Awakening Protocols} (Silver), Eternal Warrior, Fearless, \quickref{Living Metal}, Preferred Enemy (Vehicles and Dreadnoughts), \quickref{Reanimation Protocols} \\
	\textbf{Necrodermis Vessel:} The C'Tan has a 4+ invulnerable save. \\
	\textbf{Enslaved Star God:} If this model would be removed (After \quickref{Reanimation Protocols} have been failed), roll a D6. On a 1, the shackles for the C'Tan Shard have been broken and it is now rampaging. The enemy player may return the model to a point within 3" of where it died with 1 Wound remaining. It is now treated as an enemy unit to all players, taking its turns at the beginning of its owner's turns. It will attempt to attack the closest nearby unit, preferring its owner's units on a tie. If it would be removed while rampaging, this ability does not trigger again. \\
	\textbf{Flaming Vessel:} At the start of the Fight sub-phase, enemy units within 4" suffer D3 S6 AP 5 Armourbane (Melta) hits.\\
	\textbf{Immune to Natural Laws:} When moving, this model can move over all other models and terrain 	freely, and automatically passes Dangerous Terrain tests. However, it 	cannot end its move on top of other models and can only end its move on top of impassable terrain if it is possible to actually place the model on top of it. \\
	Powers of the C'Tan: The Burning One gains the \quickref{Lord of Fire} power alongside an additional power of choice from the list below. It uses powers at a Shard level. \\
	\begin{itemize}
		\item \quickref{Antimatter Meteor} \hrulefill X pt
		\item \quickref{Cosmic Fire} \hrulefill X pt
		\item \quickref{Entropic Touch} \hrulefill X pt
		\item \quickref{Moulder of Worlds} \hrulefill X pt
		\item \quickref{Pyreshards} \hrulefill X pt
		\item \quickref{Sentient Singularity} \hrulefill X pt
		\item \quickref{Seismic Assault} \hrulefill X pt
		\item \quickref{Sky of Falling Stars} \hrulefill X pt
		\item \quickref{Swarm of Spirit Dust} \hrulefill X pt
		\item \quickref{Time's Arrow} \hrulefill X pt
		\item \quickref{Transdimensional Thunderbolt} \hrulefill X pt
		\item \quickref{Withering Worldscape} \hrulefill X pt
	\end{itemize} \\
	\hline
\end{tabular}


\newpage
\subsubsection{C'Tan Shard of Tsara'noga, the Outsider}

\noindent
\begin{tabular}{||m{10pt} m{95pt} m{30pt} m{11pt} m{11pt} m{11pt} m{11pt} m{11pt} m{11pt} m{11pt} m{11pt} m{11pt} m{11pt} m{125pt}||}
	\hline
	No & Name & & M & WS & BS & S & T & W & I & A & LD & Sv & Type \\
	\hline
	1 & Outsider & X pts & 9" & 5 & 5 & 7 & 7 & 5 & 4 & 4 & 10 & 4+ & Infantry (Monstrous)\\
	\hline
	\hline
	\multicolumn{14}{||Z{532 pt}||}{Wargear: Scythe of the Nightbringer}\\
	\hline
\end{tabular}

\noindent
\begin{tabular}{||m{140pt} m{0pt} m{31pt} m{55pt} m{12pt} m{12pt} m{210pt}||}
	\hline
	Name & & Range & Type & S & AP & Abilities \\
	Touch of Eternity & & — & Melee & 10 & 1 & Shroud of Despair \\
	\hline	
	\hline
\end{tabular}

\noindent
\begin{tabular}{||m{532pt}||}
	\hline
	Abilities \\
	\hline
	\quickref{Awakening Protocols} (Silver), Eternal Warrior, Fearless, \quickref{Living Metal}, Preferred Enemy (Vehicles and Dreadnoughts), \quickref{Reanimation Protocols} \\
	\textbf{Shroud of Despair:} To Wound rolls are made against the target's Leadership (modified by Fear) rather than Toughness. The attack has no effect against Vehicles. Successful wounds against Dreadnoughts and Automata must be re-rolled. \\
	\textbf{Necrodermis Vessel:} The C'Tan has a 4+ invulnerable save. \\
	\textbf{Enslaved Star God:} If this model would be removed (After \quickref{Reanimation Protocols} have been failed), roll a D6. On a 1, the shackles for the C'Tan Shard have been broken and it is now rampaging. The enemy player may return the model to a point within 3" of where it died with 1 Wound remaining. It is now treated as an enemy unit to all players, taking its turns at the beginning of its owner's turns. It will attempt to attack the closest nearby unit, preferring its owner's units on a tie. If it would be removed while rampaging, this ability does not trigger again. \\
	\textbf{Unfathomable Horror:} Enemy models with Fearless are treated as only having Stubborn and those with Stubborn are treated as having no ability, for the purposes of determining Morale checks caused by this model. \\
	\textbf{Immune to Natural Laws:} When moving, this model can move over all other models and terrain freely, and automatically passes Dangerous Terrain tests. However, it cannot end its move on top of other models and can only end its move on top of impassable terrain if it is possible to actually place the model on top of it. \\
	Powers of the C'Tan: The Outsider gains the \quickref{Gaze of the Abyss} power alongside an additional power of choice from the list below. It uses powers at a Shard level. \\
	\begin{itemize}
		\item \quickref{Antimatter Meteor} \hrulefill X pt
		\item \quickref{Cosmic Fire} \hrulefill X pt
		\item \quickref{Entropic Touch} \hrulefill X pt
		\item \quickref{Moulder of Worlds} \hrulefill X pt
		\item \quickref{Pyreshards} \hrulefill X pt
		\item \quickref{Sentient Singularity} \hrulefill X pt
		\item \quickref{Seismic Assault} \hrulefill X pt
		\item \quickref{Sky of Falling Stars} \hrulefill X pt
		\item \quickref{Swarm of Spirit Dust} \hrulefill X pt
		\item \quickref{Time's Arrow} \hrulefill X pt
		\item \quickref{Transdimensional Thunderbolt} \hrulefill X pt
		\item \quickref{Withering Worldscape} \hrulefill X pt
	\end{itemize} \\
	\hline
\end{tabular}



