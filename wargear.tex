\section{Wargear}

\subsection{Melee Weapons} \label{Melee Weapons}

\subsubsection{Hyperphase Weapons} 

\label{Hyperphase Sword} \label{Hyperphase Thresher} \label{Hyperphase Reap-Blade} \label{Hyperphase Harvester}
\noindent
\begin{tabular}{||m{110pt} m{30pt} m{31pt} m{55pt} m{12pt} m{12pt} m{210pt}||}
	\hline
	Name & & Range & Type & S & AP & Abilities \\
	\hline
	\quickref{Hyperphase Harvester} &  & — & Melee & +2 & 2 & Murderous Strike (4+), Two-Handed, Unwieldy \\
	\quickref{Hyperphase Sword} &  & — & Melee & User & 3 & Rending (5+) \\
	\quickref{Hyperphase Reap-Blade} &  & — & Melee & +2 & 2 & Murderous Strike (5+), Two-Handed \\
	\quickref{Hyperphase Thresher} &  & — & Melee & User & 3 & Reaping Blow (1), Specialist Weapon \\
	\hline
\end{tabular}

\subsubsection{Rod of Covenant} \label{Rod of Covenant}

\noindent
\begin{tabular}{||m{110pt} m{31pt} m{55pt} m{12pt} m{12pt} m{210pt}||}
	\quickref{Rod of Covenant} &  &  &  &  & \\
	— Shooting & 12" & Assault 1 & 5 & 2 & — \\
	— Melee & — & Melee & User & 2 & Breaching(6+), Two-Handed \\
\end{tabular}

\subsubsection{Rod of Night} \label{Rod of Night}
\noindent
\begin{tabular}{||m{110pt} m{30pt} m{31pt} m{55pt} m{12pt} m{12pt} m{210pt}||}
	\hline
	Name & & Range & Type & S & AP & Abilities \\
	\hline
	\quickref{Rod of Night} &  &  &  &  & \\
	— Shooting & & 24" & Assault 2 & 5 & — & Haywire, \quickref{Tesla} (6+) \\
	— Melee & & — & Melee & User & — & Energy Siphon, Haywire \\
	\hline
\end{tabular}
\label{Energy Siphon}
\textbf{Energy Siphon:} At the end of the Fight sub-phase, if the bearer successfully hit with one or more attacks using the Rod of Night, they may roll a D6. On a roll of 4+, they may do one of the following to a friendly model within 3":
\begin{itemize}
	\item Restore a lost Wound
	\item Restore a lost Hull Point
\end{itemize}
	

\subsubsection{Staff of Light} \label{Staff of Light}
\noindent
\begin{tabular}{||m{110pt} m{30pt} m{31pt} m{55pt} m{12pt} m{12pt} m{210pt}||}
	\hline
	Name & & Range & Type & S & AP & Abilities \\
	\hline
	\quickref{Staff of Light} (Shooting) & & 18" & Assault 3 & 5 & 3 & — \\
	\quickref{Staff of Light} (Melee) & & — & Melee & User & 3 & Rending (6+) \\
	\hline
\end{tabular}

\subsubsection{Voidblade} \label{Voidblade}
\noindent
\begin{tabular}{||m{110pt} m{30pt} m{31pt} m{55pt} m{12pt} m{12pt} m{210pt}||}
	\hline
	Name & & Range & Type & S & AP & Abilities \\
	\hline
	\quickref{Voidblade} &  & — & Melee & User & 4 & \quickref{Entropic Strike} (4+), Rending (6+) \\
	\hline
\end{tabular}

\subsubsection{Voidscythe} \label{Voidscythe}
\noindent
\begin{tabular}{||m{110pt} m{30pt} m{31pt} m{55pt} m{12pt} m{12pt} m{210pt}||}
	\hline
	Name & & Range & Type & S & AP & Abilities \\
	\hline
	\quickref{Voidscythe} &  & — & Melee & x2 & 1 & \quickref{Entropic Strike} (2+), Brutal (2), Unwieldy, Two-Handed \\
	\hline
\end{tabular}

\subsubsection{Warscythe}
\label{Warscythe}
\noindent
\begin{tabular}{||m{110pt} m{30pt} m{31pt} m{55pt} m{12pt} m{12pt} m{210pt}||}
	\hline
	Name & & Range & Type & S & AP & Abilities \\
	\hline
	\quickref{Warscythe} & x pts& — & Melee & +2 & 2 & Armourbane (Melee), Two-Handed \\
	\hline
\end{tabular}

\subsubsection{Whip Coils}
\label{Whip Coils}
\noindent
\begin{tabular}{||m{110pt} m{30pt} m{31pt} m{55pt} m{12pt} m{12pt} m{210pt}||}
	\hline
	Name & & Range & Type & S & AP & Abilities \\
	\hline
	\quickref{Whip Coils} & & — & Melee & User & — & Reach (3) \\
	\hline
\end{tabular}



\subsection{Ranged Weapons} \label{Ranged Weapons}


\subsubsection{Atomiser Weapons}

\label{Atomiser Beam Lance}
\noindent
\begin{tabular}{||m{110pt} m{30pt} m{31pt} m{55pt} m{12pt} m{12pt} m{210pt}||}
	\hline
	Name & & Range & Type & S & AP & Abilities \\
	\hline
	\quickref{Atomiser Beam Lance} & & 12" & Heavy 3 & 6 & 4 & Murderous Strike (6+) \\
	\hline
\end{tabular}

\subsubsection{Doomsday Weapons} 
\label{Doomsday Cannon} \label{Doomsday Blaster}
\noindent
\begin{tabular}{||m{110pt} m{30pt} m{31pt} m{55pt} m{12pt} m{12pt} m{210pt}||}
\hline
Name & & Range & Type & S & AP & Abilities \\
\hline
\quickref{Doomsday Blaster} & & & & & & \\
— Low Power &  & 24" & Ordnance 1 & 8 & 3 & Blast \\
— High Power & & 48" & Ordnance 1 & 10 & 1 & Large Blast, Divert Power \\
\quickref{Doomsday Cannon} & & & & & & \\
— Low Power &  & 36" & Heavy 1 & 8 & 3 & Blast \\
— High Power & & 72" & Heavy 1 & 10 & 1 & Large Blast, Divert Power \\
\hline
\end{tabular}
\textbf{Divert Power:} A vehicle can only fire a weapon with this rule if it remained stationary in its preceding Movement phase. 

\subsubsection{Enmitic Weapons}

\label{Enmitic Exterminator} \label{Enmitic Annihilator} \label{Enmitic Disintegrator Pistol}
\noindent
\begin{tabular}{||m{140pt} m{0pt} m{31pt} m{55pt} m{12pt} m{12pt} m{210pt}||}
	\hline
	Name & & Range & Type & S & AP & Abilities \\
	\hline
	\quickref{Enmitic Annihilator} &  & 18" & Assault 1 & 6 & 4 & Blast, Molecular Dissonance \\
	\quickref{Enmitic Disintegrator Pistol} &  & 18" & Pistol 1 & 6 & 4 & Molecular Dissonance \\
	\quickref{Enmitic Exterminator} &  & 36" & Heavy 1 & 7 & 4 & Large Blast, Molecular Dissonance \\
	\hline
\end{tabular}

\subsubsection{Gauntlet Weapons}

\label{Gauntlet of Fire} \label{Tachyon Arrow}
\noindent
\begin{tabular}{||m{110pt} m{30pt} m{31pt} m{55pt} m{12pt} m{12pt} m{210pt}||}
	\hline
	Name & & Range & Type & S & AP & Abilities \\
	\hline
	\quickref{Gauntlet of Fire} & & Template & Assault 1 & 4 & 5 & — \\
	\quickref{Tachyon Arrow} & & $\infty$ & Assault 1 & 10 & 1 & Destructor, One use, \quickref{Path of Annihilation} (1) \\
	\hline
\end{tabular}


\subsubsection{Gauss Weapons}

\label{Gauss Cannon} \label{Gauss Blaster} \label{Gauss Flayer} \label{Gauss Flux Arcs} \label{Gauss Reaper} \label{Heavy Gauss Cannon} \label{Relic Gauss Blaster}
\noindent
\begin{tabular}{||m{110pt} m{30pt} m{31pt} m{55pt} m{12pt} m{12pt} m{210pt}||}
	\hline
	Name & & Range & Type & S & AP & Abilities \\
	\hline
	\quickref{Gauss Cannon} & & 24" & Heavy 3 & 6 & 3 & \quickref{Gauss} (6+) \\
	\quickref{Gauss Flayer} & & 24" & Rapid Fire & 4 & 5 & \quickref{Gauss} (6+) \\
	\quickref{Gauss Flux Arcs} & & 24" & Heavy 3 & 4 & 5 & \quickref{Gauss} (6+) \\
	\quickref{Gauss Reaper} & & 12" & Assault 2 & 5 & 4 & \quickref{Gauss} (6+) \\
	\quickref{Gauss Blaster} & & 24" & Rapid Fire & 5 & 4 & \quickref{Gauss} (6+) \\
	\quickref{Heavy Gauss Cannon} & & 36" & Heavy 1 & 9 & 2 & \quickref{Gauss} (6+) \\
	\quickref{Relic Gauss Blaster} & & 30" & Rapid Fire 2 & 5 & 4 & \quickref{Gauss} (6+), Master-Crafted \\	
	\hline
\end{tabular}

\subsubsection{Particle Weapons}

\label{Particle Caster} \label{Particle Beamer} \label{Particle Shredder} \label{Particle Whip}
\noindent
\begin{tabular}{||m{110pt} m{30pt} m{31pt} m{55pt} m{12pt} m{12pt} m{210pt}||}
	\hline
	Name & & Range & Type & S & AP & Abilities \\
	\hline
	\quickref{Particle Caster} & & 12" & Pistol 1 & 6 & 5 & \\
	\quickref{Particle Beamer} & & 24" & Heavy 1 & 6 & 5 & Blast \\
	\quickref{Particle Shredder} && 24" & Heavy 1 & 7 & 4 & Large Blast \\
	\quickref{Particle Whip} & & 24" & Ordnance 1 & 8 & 3 & Discriminatory, Large Blast \\
	\hline
\end{tabular}
\textbf{Discriminatory:} Friendly Necron units that have models under this weapon's Blast template are never hit.

\subsubsection{Synaptic Weapons}
\label{Synaptic Disintegrator}
\noindent
\begin{tabular}{||m{110pt} m{30pt} m{31pt} m{55pt} m{12pt} m{12pt} m{210pt}||}
\hline
Name & & Range & Type & S & AP & Abilities \\
\hline
\quickref{Synaptic Disintegrator} & & 36" & Rapid Fire & 5 & 5 & Rending (5+), Pinning, Sniper \\	
\hline
\end{tabular}

\subsubsection{Tesla Weapons}
\label{Tesla Cannon} \label{Tesla Carbine}
\noindent
\begin{tabular}{||m{110pt} m{30pt} m{31pt} m{55pt} m{12pt} m{12pt} m{210pt}||}
	\hline
	Name & & Range & Type & S & AP & Abilities \\
	\hline
	\quickref{Tesla Cannon} & & 30" & Heavy 3 & 6 & — & \quickref{Tesla} (6+) \\
	\quickref{Tesla Carbine} & & 24" & Assault 2 & 5 & — & \quickref{Tesla} (6+) \\	
	\hline
\end{tabular}


\subsubsection{Transdimensional Weapons}
\label{Transdimensional Beamer}
\noindent
\begin{tabular}{||m{110pt} m{30pt} m{31pt} m{55pt} m{12pt} m{12pt} m{210pt}||}
	\hline
	Name & & Range & Type & S & AP & Abilities \\
	\hline
	\quickref{Transdimensional Beamer} & & 12" & Heavy 1 & 4 & 5 & \quickref{Exile Ray} (6+) \\	
	\hline
\end{tabular}


\subsection{Technoarkana} \label{Technoarcana}


\subsubsection{Bloodswarm Scarabs} \label{Bloodswarm Scarabs}

Friendly units with the Flayer sub-type can re-roll the scatter dice when arriving from Deep Strike Reserve. 

\subsubsection{Dispersion Shield} \label{Dispersion Shield}

Grants a 3+ Invulnerable Save, but the wielder never receives +1 Attack for fighting with two Melee weapons. In addition, if this save is made against a shooting attack, the bearer may take a -1 penalty to the Invulnerable Save to attempt to reflect the weapon into a nearby enemy. If the save is successful, choose an unengaged enemy unit within 6": that unit then suffers a single hit equivalent to the attack that was saved. This option may not be used on Blast or Template weapons. For the purpose of cover and other effects, treat the reflected shots as having come from the bearer's unit.

\subsubsection{Dynastic Ankh} \label{Dynastic Ankh}

%TODO: This
Some sort of vexilla style benefit.

\subsubsection{Eternity Gate} \label{Eternity Gate}

For each Eternity gate, at the start of each friendly turn you may choose one mode of operation:

\begin{itemize}
	\item Choose one friendly unengaged Necron unit without the Vehicle Unit-Type that is on the battlefield or in \quickref{Teleportation Reserves}. If the unit is in \quickref{Teleportation Reserves} it immediately arrives this turn (no dice roll is required) and is placed as if it were disembarking from the Monolith. If the chosen unit is currently on the battlefield it is first removed from the table and place into \quickref{Teleportation Reserves}, after which you may also place the unit as if it were disembarking from the Monolith.
	\item All enemy models without the Vehicle Unit-Type within D6" of the Monolith's portal and in line of sight to it must make a Strength Check. Failure causes the model to suffer an immediate Instant Death wound with no saves or Damage Mitigation rolls allowed.
\end{itemize}

Models with an Eternity Gate can be boarded following the normal rules for Transport, however place the unit into \quickref{Teleportation Reserves} instead.


\subsubsection{Flensing Scarabs} \label{Flensing Scarabs}

During the first round of each close combat, this unit's weapons count as having the Shred special rule. Units with the Necron Dynasty (Maynarkh) special rule may also take this wargear.

\subsubsection{Gloom Prism} \label{Gloom Prism}

This model gains the Anathema sub-type. Additionally, any psychic power targeting a unit within 3" of this model is nullified on a 4+. \\

\subsubsection{Gravity Displacement Pack} \label{Gravity Displacement Pack}

At the start of the controlling player’s Movement phase, a model with a Gravity Displacement Pack may set its Move Characteristic to a value of 12 for the duration of the controlling player’s turn (sometimes referred to as ‘activating’ the jump pack). This allows a model with a Gravity Displacement Pack to move up to 12", regardless of the Movement Characteristic shown on its profile and gain any other benefits of a Movement Characteristic of 12 (including the bonus to Charge distance). In addition, all models with a Gravity Displacement Pack that have been activated ignore terrain while Moving and Charging, but must take Dangerous Terrain tests as normal when beginning or ending their Movement in Dangerous Terrain. A model with an activated Gravity Displacement Pack pack may move over both friendly and enemy models or units without penalty – but must end its Movement at least 1" away from any model from another unit.

A model with a Gravity Displacement Pack may still Run if it would normally be able to Run (this does not allow units that include any models with the Heavy Sub-type to Run). When making a Run move for a model with an activated Gravity Displacement Pack, add the Initiative Characteristic of that model to 12 to determine how far it may move – the model ignores terrain and models from other units while making a Run move with a Gravity Displacement Pack as previously noted, but may not make Shooting Attacks or declare a Charge in the same turn in which it has Run as per the normal rules for Running.

During a Reaction made in any Phase, a player may not choose to activate a model’s Gravity Displacement Pack to gain any bonus to its Movement Characteristic.

\subsubsection{Hyper-Oubliette Navigator} \label{Hyper-Oubliette Navigator}

The \quickref{Ethereal Interception} Advanced Reaction can be performed by this unit for 0 Reaction Allotment. Additionally, the may opt to re-roll Deep-Strike scatter rolls.

\subsubsection{Mindshackle Scarabs} \label{Mindshackle Scarabs}

At the start of the Fight sub-phase, select a model in base contact with the bearer. The target must take a Leadership Check on 3D6. If the Check is failed, the victim strikes at his allies instead of attacking normally. During the Fight sub-phase, the target makes his attacks against his own unit (automatically hitting if they are the only model in the unit), resolved as normal with any abilities and penalties from his weapons (the controller of the Mindshackle Scarbs chooses which, if there is a choice). If the target is still alive, the victim returns to normal once all blows in that round of combat have been struck. Dreadnoughts and Automata may re-roll failed Leadership tests for this effect.

\subsubsection{Phylactery} \label{Phylactery}

Increase the models It Will Not Die level to 3+.

\subsubsection{Phase Shifter} \label{Phase Shifter}

Grants a 4+ Invulnerable Save.

\subsubsection{Stellar Energetic Reactors} \label{Stellar Energetic Reactors}

This wargear may only be taken by units with the Necron Dynasty (Mephrit) special rule. When making Shooting attacks, this unit's weapons count as having the Shred special rule.

\subsubsection{Shadow Ankh} \label{Shadow Ankh}

The bearer gains the Anathema sub-type.

\subsubsection{Radioactive Energetics} \label{Radioactive Energetics}

Any melee Hits allocated to models locked in combat with one or more units that include a model with Radioactive Energetics require one lower result To Wound than they would normally, to a minimum of 2+. This effect is not cumulative with itself if more than one model in a combat has Radioactive Energetics. Models with Radioactive Energetics are immune to the effects of rad grenades, the Rad-phage special rule and the rad furnaces or Radioactive Energetics of models they are locked in combat with. In addition, Hits from weapons with the Rad-phage special rule that are allocated to a model with Radioactive Energetics only successfully wound on a To Wound roll of a 6+. 

\subsubsection{Rad-Receptors} \label{Rad Receptors}

This wargear may only be taken by units with the Necron Dynasty (Thokt) special rule. When making Shooting attacks, whis unit's weapons count as having the Rad-Phage special rule.s

\subsubsection{Resurrection Orb} \label{Resurrection Orb}

Once per battle, on your turn, the bearer can activate their Resurrection Orb. If it does, select one friendly unit with \quickref{Reanimation Protocols} within \quickref{Nodal Range}. The bearer of the Orb and the selected unit immediately \textcolor{violet}{\hyperref[Reanimation Protocols]{reassembles}} a number of wounds equal to the total number of missing wounds and number of wounds from all destroyed models.

\subsubsection{Semipternal Weave} \label{Sempiternal Weave}

Increase the model's save to 2+.

\subsubsection{Sepulchral Scarabs} \label{Sepulchral Scarabs}

Increase this model's It Will Not Die level to 3+.

\subsubsection{Tesseract Labyrinth} \label{Tesseract Labyrinth}

One use only.

The bearer can use the Tesseract Labyrinth in lieu of making a Shooting or Close Combat attack this round. Target a unit within 6": the target unit must immediately roll a Wounds Check for each model based on their current remaining wounds or be trapped within the Labyrinth.

This can also be used to carry a unit alongside the bearer. If you do so, select a unit to start the game inside the Tesseract Labyrinth. They can be disembarked as if they were in a Transport with no Special Abilities, following the relevant rules. If the bearer dies, the embarked unit is lost and considered destroyed for the purposes of objectives.

If paired with the \quickref{Mindshackle Scarabs} wargear, an embarked unit can also be chosen from an enemy faction. The unit is treated as a Distrusted Ally and must still take up a relevant Force Org Slot. In narrative games, units that are captured at the end of the game can be included as well. Leave it your opponents as to whether it includes a Force Org Slot or points.

\subsubsection{Translocation Shroud} \label{Translocation Shroud}

The bearer and its attached unit gains the Fleet (2) special rule. When moving, the bearer and its attached unit can move over all other models and terrain as if they were open ground. However, they cannot end their move on top of other models and can only end their move on top of impassable terrain if it is possible to actually place the models on top of it.

\subsubsection{Quantum Shielding} \label{Quantum Shielding}

A vehicle equipped with active quantum shielding reduces the strength of attacks to its Front and Side Armour by -2. A vehicle’s quantum shielding is active until it suffers a penetrating hit, at which point it immediately
deactivates. For the remainder of the battle after a vehicle’s quantum shielding deactivates, all subsequent hits against that vehicle (including hits made from subsequent shooting attacks in the same phase – either
from a different weapon or a different unit – or hits made at a lower Initiative step in close combat) are treated as though the vehicle was not equipped with quantum shielding.

\subsubsection{Quantum Shielding} \label{Quantum Shielding Matrix}

A vehicle equipped with a Quantum Shielding Matrix and \quickref{Quantum Shielding} whose \quickref{Quantum Shielding} has been deactivated in a previous round may attempt to reactivate it. At the start of your turn, roll a D6: on a 5+ the vehicle \quickref{Quantum Shielding} reactivates and functions as normal.

\subsection{Artefacts of the Aeons} \label{Artefacts of the Aeons}

%TODO: This
