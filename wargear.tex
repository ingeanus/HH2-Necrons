\usection{Wargear}

\usubsection{Melee Weapons} \label{Melee Weapons}

\usubsubsection{Staff of Light} \label{Staff of Light}
	\noindent
	\begin{tabular}{||m{110pt} m{30pt} m{31pt} m{55pt} m{12pt} m{12pt} m{210pt}||}
		\hline
		Name & & Range & Type & S & AP & Abilities \\
		\hline
		\quickref{Staff of Light} (Shooting) & & 18" & Assault 3 & 5 & 3 & — \\
		\quickref{Staff of Light} (Melee) & & — & Melee & User & 3 & Rending (6+) \\
		\hline
\end{tabular}

\usubsubsection{Hyperphase Sword} \label{Hyperphase Sword}
\noindent
\begin{tabular}{||m{110pt} m{30pt} m{31pt} m{55pt} m{12pt} m{12pt} m{210pt}||}
	\hline
	Name & & Range & Type & S & AP & Abilities \\
	\hline
	\quickref{Hyperphase Sword} &  & — & Melee & User & 3 & Rending (5+) \\
	\hline
\end{tabular}

\usubsubsection{Voidblade} \label{Voidblade}
\noindent
\begin{tabular}{||m{110pt} m{30pt} m{31pt} m{55pt} m{12pt} m{12pt} m{210pt}||}
	\hline
	Name & & Range & Type & S & AP & Abilities \\
	\hline
	\quickref{Voidblade} &  & — & Melee & User & 4 & \quickref{Entropic Strike} (5+), Rending (6+) \\
	\hline
\end{tabular}

\usubsubsection{Voidscythe} \label{Voidscythe}
\noindent
\begin{tabular}{||m{110pt} m{30pt} m{31pt} m{55pt} m{12pt} m{12pt} m{210pt}||}
	\hline
	Name & & Range & Type & S & AP & Abilities \\
	\hline
	\quickref{Voidscythe} &  & — & Melee & x2 & 1 & \quickref{Entropic Strike} (2+), Brutal (2), Unwieldy, Two-Handed \\
	\hline
\end{tabular}

\usubsubsection{Warscythe}
\label{Warscythe}
\noindent
\begin{tabular}{||m{110pt} m{30pt} m{31pt} m{55pt} m{12pt} m{12pt} m{210pt}||}
	\hline
	Name & & Range & Type & S & AP & Abilities \\
	\hline
	\quickref{Warscythe} & x pts& — & Melee & +2 & 2 & Armourbane (Melee), Two-Handed \\
	\hline
\end{tabular}



\usubsection{Ranged Weapons} \label{Ranged Weapons}

\usubsubsection{Gauntlet Weapons}

\label{Gauntlet of Fire} \label{Tachyon Arrow}
\noindent
\begin{tabular}{||m{110pt} m{30pt} m{31pt} m{55pt} m{12pt} m{12pt} m{210pt}||}
	\hline
	Name & & Range & Type & S & AP & Abilities \\
	\hline
	\quickref{Gauntlet of Fire} & x pts& Template & Assault 1 & 4 & 5 & — \\
	\quickref{Tachyon Arrow} & x pts& 120" & Destroyer 1 & 10 & 1 & One use \\
	\hline
\end{tabular}



\usubsubsection{Gauss Weapons}

\label{Gauss Blaster} \label{Gauss Flayer} \label{Gauss Reaper} \label{Relic Gauss Blaster}
\noindent
\begin{tabular}{||m{110pt} m{30pt} m{31pt} m{55pt} m{12pt} m{12pt} m{210pt}||}
	\hline
	Name & & Range & Type & S & AP & Abilities \\
	\hline
	\quickref{Gauss Flayer} & x pts& 24" & Rapid Fire 1 & 4 & 5 & \quickref{Gauss} (6+) \\
	\quickref{Gauss Reaper} & x pts& 12" & Assault 2 & 5 & 4 & \quickref{Gauss} (6+) \\
	\quickref{Gauss Blaster} & x pts& 24" & Rapid Fire 1 & 5 & 4 & \quickref{Gauss} (6+) \\
	\quickref{Relic Gauss Blaster} & & 30" & Rapid Fire 2 & 5 & 4 & \quickref{Gauss} (6+), Master-Crafted \\	
	\hline
\end{tabular}


\usubsubsection{Tesla Weapons}
\label{Tesla Carbine}
\noindent
\begin{tabular}{||m{110pt} m{30pt} m{31pt} m{55pt} m{12pt} m{12pt} m{210pt}||}
	\hline
	Name & & Range & Type & S & AP & Abilities \\
	\hline
	\quickref{Tesla Carbine} & x pts& 24" & Assault 1 & 5 & — & \quickref{Tesla} (6+) \\	
	\hline
\end{tabular}


\usubsection{Technoarkana} \label{Technoarcana}

\paragraph*{Dispersion Shield} \label{Dispersion Shield}

Grants a 3+ Invulnerable Save, but the wielder never receives +1 Attack for fighting with two Melee weapons.

\paragraph*{Mindshackle Scarabs} \label{Mindshackle Scarabs}

When fighting in a challenge, a model with mindshackle scarabs has the Fear (1) special rule. Fear tests taken as a result of Mindshackle Scarabs must be taken on 3D6. 

\paragraph*{Phylactery} \label{Phylactery}

Increase the models It Will Not Die level to 3+.

\paragraph*{Phase Shifter} \label{Phase Shifter}

Grants a 4+ Invulnerable Save.

\paragraph*{Resurrection Orb} \label{Resurrection Orb}

Once per battle, on your turn, the bearer can activate their Resurrection Orb. If it does, select one friendly unit with \quickref{Reanimation Protocols} within \quickref{Nodal Range}. The bearer of the Orb and the selected unit immediately \textcolor{violet}{\hyperref[Reanimation Protocols]{reassembles}} a number of wounds equal to the total number of missing wounds and number of wounds from all destroyed models.

\paragraph*{Semipternal Weave} \label{Sempiternal Weave}

Increase the model's save to 2+.

\paragraph*{Tesseract Labyrinth} \label{Tesseract Labyrinth}

One use only.

The bearer can use the Tesseract Labyrinth in lieu of making close combat attacks that round. Choose a Character or Monstrous Creature in base contact with the bearer. The victim must immediately roll equal to or under their current remaining Wounds on a D6 or be trapped within the Labyrinth while the Necron Character remains alive. Should the bearer be killed, the trapped model is immediately released from the Labyrinth and placed within 3" of where the bearer was.


\usubsection{Artefacts of the Aeons} \label{Artefacts of the Aeons}

TODO: This

\usubsection{Cryptek Conclaves}

When taking a Cryptek Conclave, a \textbf{Discipline} must be taken from the list below, which grants a number of options, abilities, and restrictions to the unit.

\usubsection{Harbingers of Despair \hrulefill X pts}

Psychomancers must take an \quickref{Abyssal Staff} when selecting the Harbingers of Despair as their Discipline.

\usubsubsection{Abyssal Staff}
\label{Abyssal Staff}
\noindent
\begin{tabular}{||m{130pt} m{10pt} m{31pt} m{55pt} m{12pt} m{12pt} m{210pt}||}
	\hline
	Name & & Range & Type & S & AP & Abilities \\
	\hline
	\quickref{Abyssal Staff} (Shooting) & & Template & Assault 1 & 8 & 1 & Shroud of Despair \\
	\quickref{Abyssal Staff} (Melee) & & — & Melee & 8 & 1 & Shroud of Despair \\
	\hline
\end{tabular}

\textbf{Shroud of Despair:} To Wound rolls are made against the target's Leadership rather than Toughness. The attack has no effect against Vehicles.

\usubsubsection{Atavindicator \hrulefill X pts}

\usubsubsection{Nightmare Shroud \hrulefill X pts} 

The bearer gains the Fear (1) rule. Additionally, the Shroud may be used during the Shooting Phase instead of firing a weapon. Choose an enemy unit within 18" of the bearer. That unit must immediately take a Morale Check.



\usubsection{Harbingers of Destruction \hrulefill X pts}

Plasmancers must take an \quickref{Eldritch Lance} when selecting the Harbingers of Destruction as their Discipline.

\usubsubsection{Eldritch Lance}
\label{Eldritch Lance}
\noindent
\begin{tabular}{||m{130pt} m{10pt} m{31pt} m{55pt} m{12pt} m{12pt} m{210pt}||}
	\hline
	Name & & Range & Type & S & AP & Abilities \\
	\hline
	\quickref{Eldritch Lance} (Shooting) & & 36" & Assault 1 & 8 & 2 & Lance \\
	\quickref{Eldritch Lance} (Melee) & & — & Melee & User & 2 & Lance \\
	\hline
\end{tabular}

\usubsubsection{Quantum Orb \hrulefill X pts}

Once per battle, at the start of your turn, the bearer can activate this item. If it does, select one point on the battlefield anywhere within 24" of the bearer and place a marker at that point. At the start of your next turn, resolve a S8 AP 3 Large Blast hit directly on that location.

\usubsubsection{Plasmic Lance \hrulefill 0 pts}

Any Plasmancer may exchange their Eldritch Lance for a Plasmic Lance.
TODO: Probably too weak

\label{Plasmic Lance}
\noindent
\begin{tabular}{||m{130pt} m{10pt} m{31pt} m{55pt} m{12pt} m{12pt} m{210pt}||}
	\hline
	Name & & Range & Type & S & AP & Abilities \\
	\hline
	\quickref{Plasmic Lance} (Shooting) & & 18" & Assault 3 & 7 & 3 & — \\
	\quickref{Plasmic Lance} (Melee) & & — & Melee & User & 3 & — \\
	\hline
\end{tabular}


\usubsection{Harbingers of Eternity \hrulefill X pts}

Chronomancers must take an \quickref{Aeonstave} when selecting the Harbingers of Eternity as their Discipline.

\usubsubsection{Aeonstave}
\label{Aeonstave}
\noindent
\begin{tabular}{||m{130pt} m{10pt} m{31pt} m{55pt} m{12pt} m{12pt} m{210pt}||}
	\hline
	Name & & Range & Type & S & AP & Abilities \\
	\hline
	\quickref{Aeonstave} & & — & Melee & User & — & \quickref{Entropic Strike} (6+), Chronal Charge \\
	\hline
\end{tabular}
\textbf{Chronal Charge:} Models which suffer an unsaved Wound or loses a Hull Point from this weapon loses the Fleet special rule and has its Weapon Skill, Ballistic Skill, Initiative and Attack values reduced to 1 for the remainder of the game.

\usubsubsection{Chronometron \hrulefill X pts}

A model with a Chronometron can re-roll one of its D6 rolls each phase. If the bearer is in a unit, this ability can be used to instead re-roll one of the units D6 rolls each phase.

\usubsubsection{Countertemporal Nanomines \hrulefill X pts}

Provide some sort of dangerous terrain / slowing / similar minefield effects

\usubsubsection{Entropic Lance \hrulefill X pts} \label{Entropic Lance}

Any Chronomancer may upgrade their Aeonstave to an Entropic Lance.

\noindent
\begin{tabular}{||m{130pt} m{10pt} m{31pt} m{55pt} m{12pt} m{12pt} m{210pt}||}
	\hline
	Name & & Range & Type & S & AP & Abilities \\
	\hline
	\quickref{Entropic Lance} (Shooting) & & Assault 1 & 18" & 7 & 3 & Brutal (2), \quickref{Entropic Strike} (2+) \\
	\quickref{Entropic Lance} (Melee) & & — & Melee & User & 3 & Brutal (2), \quickref{Entropic Strike} (2+) \\
	\hline
\end{tabular}

\usubsubsection{Timesplinter Cloak \hrulefill X pts}

A model with a Timesplinter Cloak has a 3+ Invulnerable save.



\usubsection{Harbingers of Storm \hrulefill X pts}

Ethermancers must take an \quickref{Voltaic Staff} when selecting the Harbingers of Storm as their Discipline.

\usubsubsection{Voltaic Staff}
\label{Voltaic Staff}
\noindent
\begin{tabular}{||m{130pt} m{10pt} m{31pt} m{55pt} m{12pt} m{12pt} m{210pt}||}
	\hline
	Name & & Range & Type & S & AP & Abilities \\
	\hline
	\quickref{Voltaic Staff} (Shooting) & & 12" & Assault 4 & 5 & — & Haywire \\
	\quickref{Voltaic Staff} (Melee) & & — & Melee & User & — & Haywire \\
	\hline
\end{tabular}

\usubsubsection{Ether Crystal \hrulefill X pts}

Any enemy unit arriving by Deep Strike within \quickref{Nodal Range} of the bearer suffers d6 S8 AP 5 hits. If they arrive within range of multiple C5rystals, only increase the number of hits by 1 for each Crystal past the first.

\usubsubsection{Metalodermal Tesla Weave \hrulefill X pts}

When an enemy unit successfully moves into assault with the Cryptek or his unit, the assaulting unit immediately suffers d6 S8 AP 5 hits.




\usubsection{Harbingers of Technomancy \hrulefill X pts}

Technomancers must take a \quickref{Staff of Light} when selecting the Haringers of Technomancy as their Discipline. Additionally, they must purchase the \quickref{Rites of Reanimation} ability.

\usubsubsection{Canoptek Cloak \hrulefill X pts}

Increase the bearer's move to 12" and it gains the Fleet (1) rule alongside the Antigrav and Light sub-type.

\usubsubsection{Canoptek Control Node \hrulefill X pts}

Increase your \quickref{Nodal Range} to 12" for the purposes of suppressing the \quickref{Soulless Hordes} trait for units with the Canoptek sub-type.

\usubsubsection{Fail-Safe Overcharger \hrulefill X pts}

Psychic power thing where you have a lot of option and roll with penalties for each you want, causing wounds on fail.

\begin{itemize}
	\item Test
\end{itemize}

\usubsubsection{Phylacterine Hive \hrulefill X pts}

Once per battle, when using your \quickref{Rites of Reanimation} ability, you may select a non-friendly unit with \quickref{Reanimation Protocols} (Such as Destroyer Cult or Flayer Virus units) to be affected.

\usubsubsection{Rites of Reanimation} \label{Rites of Reanimation}

After this model has moved, select a friendly unit with \quickref{Reanimation Protocols} within \quickref{Nodal Range} on the bearer. That unit immediately \textcolor{violet}{\hyperref[Reanimation Protocols]{reassembles}} a number of wounds equal to the number of wounds from all destroyed models (Do not include any lost wounds from non-destroyed models), but roll with a -1 modifier.



\usubsection{Harbingers of Transmogrification \hrulefill X pts}

Geomancers and Alchemists must take an \quickref{Tremorstave} when selecting the Harbingers of Transmogrification as their Discipline.

\usubsubsection{Tremorstave}
\label{Tremorstave}
\noindent
\begin{tabular}{||m{130pt} m{10pt} m{31pt} m{55pt} m{12pt} m{12pt} m{210pt}||}
	\hline
	Name & & Range & Type & S & AP & Abilities \\
	\hline
	\quickref{Tremorstave} (Shooting) & & 36" & Assault 1 & 4 & — & Blast, Pinning, Quake \\
	\quickref{Tremorstave} (Melee) & & — & Melee & User & — & Pinning \\
	\hline
\end{tabular}
\textbf{Quake:} After resolving all wounds, leave the Blast marker in place, or otherwise mark the area. This area now counts as Difficult Terrain until the start of the next turn of the player that made the attack.

\usubsubsection{Cryptogeometric Adjuster \hrulefill X pts}