\section{Wargear}

\subsection{Melee Weapons} \label{Melee Weapons}

\subsubsection{Hyperphase Weapons} 

\label{Hyperphase Sword} \label{Hyperphase Thresher} \label{Hyperphase Reap-Blade} \label{Hyperphase Harvester}
\noindent
\begin{tabular}{||m{110pt} m{30pt} m{31pt} m{55pt} m{12pt} m{12pt} m{210pt}||}
	\hline
	Name & & Range & Type & S & AP & Abilities \\
	\hline
	\quickref{Hyperphase Harvester} &  & — & Melee & +2 & 2 & Murderous Strike (4+), Two-Handed, Unwieldy \\
	\quickref{Hyperphase Sword} &  & — & Melee & User & 3 & Rending (5+) \\
	\quickref{Hyperphase Reap-Blade} &  & — & Melee & +2 & 2 & Murderous Strike (5+), Two-Handed \\
	\quickref{Hyperphase Thresher} &  & — & Melee & User & 3 & Reaping Blow (1), Specialist Weapon \\
	\hline
\end{tabular}

\subsubsection{Rod of Night} \label{Rod of Night}
\noindent
\begin{tabular}{||m{110pt} m{30pt} m{31pt} m{55pt} m{12pt} m{12pt} m{210pt}||}
	\hline
	Name & & Range & Type & S & AP & Abilities \\
	\hline
	\quickref{Rod of Night} (Melee) & & — & Melee & User & — & Energy Siphon, Haywire \\
	\quickref{Rod of Night} (Shooting) & & 24" & Assault 2 & 5 & — & Haywire, \quickref{Tesla} (6+) \\
	\hline
\end{tabular}
\label{Energy Siphon}
\textbf{Energy Siphon:} At the end of the Fight sub-phase, if the bearer successfully hit with one or more attacks using the Rod of Night, they may roll a D6. On a roll of 4+, they may do one of the following to a friendly model within 3":
\begin{itemize}
	\item Restore a lost Wound
	\item Restore a lost Hull Point
\end{itemize}
	

\subsubsection{Staff of Light} \label{Staff of Light}
\noindent
\begin{tabular}{||m{110pt} m{30pt} m{31pt} m{55pt} m{12pt} m{12pt} m{210pt}||}
	\hline
	Name & & Range & Type & S & AP & Abilities \\
	\hline
	\quickref{Staff of Light} (Shooting) & & 18" & Assault 3 & 5 & 3 & — \\
	\quickref{Staff of Light} (Melee) & & — & Melee & User & 3 & Rending (6+) \\
	\hline
\end{tabular}

\subsubsection{Voidblade} \label{Voidblade}
\noindent
\begin{tabular}{||m{110pt} m{30pt} m{31pt} m{55pt} m{12pt} m{12pt} m{210pt}||}
	\hline
	Name & & Range & Type & S & AP & Abilities \\
	\hline
	\quickref{Voidblade} &  & — & Melee & User & 4 & \quickref{Entropic Strike} (4+), Rending (6+) \\
	\hline
\end{tabular}

\subsubsection{Voidscythe} \label{Voidscythe}
\noindent
\begin{tabular}{||m{110pt} m{30pt} m{31pt} m{55pt} m{12pt} m{12pt} m{210pt}||}
	\hline
	Name & & Range & Type & S & AP & Abilities \\
	\hline
	\quickref{Voidscythe} &  & — & Melee & x2 & 1 & \quickref{Entropic Strike} (2+), Brutal (2), Unwieldy, Two-Handed \\
	\hline
\end{tabular}

\subsubsection{Warscythe}
\label{Warscythe}
\noindent
\begin{tabular}{||m{110pt} m{30pt} m{31pt} m{55pt} m{12pt} m{12pt} m{210pt}||}
	\hline
	Name & & Range & Type & S & AP & Abilities \\
	\hline
	\quickref{Warscythe} & x pts& — & Melee & +2 & 2 & Armourbane (Melee), Two-Handed \\
	\hline
\end{tabular}

\subsubsection{Whip Coils}
\label{Whip Coils}
\noindent
\begin{tabular}{||m{110pt} m{30pt} m{31pt} m{55pt} m{12pt} m{12pt} m{210pt}||}
	\hline
	Name & & Range & Type & S & AP & Abilities \\
	\hline
	\quickref{Whip Coils} & & — & Melee & User & — & Reach (3) \\
	\hline
\end{tabular}



\subsection{Ranged Weapons} \label{Ranged Weapons}


\subsubsection{Atomiser Weapons}

\label{Atomiser Beam Lance}
\noindent
\begin{tabular}{||m{110pt} m{30pt} m{31pt} m{55pt} m{12pt} m{12pt} m{210pt}||}
	\hline
	Name & & Range & Type & S & AP & Abilities \\
	\hline
	\quickref{Atomiser Beam Lance} & & 12" & Heavy 3 & 6 & 4 & Murderous Strike (6+) \\
	\hline
\end{tabular}

\subsubsection{Enmitic Weapons}

\label{Enmitic Exterminator} \label{Enmitic Annihilator} \label{Enmitic Disintegrator Pistol}
\noindent
\begin{tabular}{||m{110pt} m{30pt} m{31pt} m{55pt} m{12pt} m{12pt} m{210pt}||}
	\hline
	Name & & Range & Type & S & AP & Abilities \\
	\hline
	\quickref{Enmitic Annihilator} &  & 18" & Assault 1 & 6 & 4 & Blast, Molecular Dissonance \\
	\quickref{Enmitic Disintegrator Pistol} &  & 18" & Pistol 1 & 6 & 4 & Molecular Deconstruction \\
	\quickref{Enmitic Exterminator} &  & 36" & Heavy 1 & 7 & 4 & Large Blast, Molecular Dissonance \\
	\hline
\end{tabular}

\subsubsection{Gauntlet Weapons}

\label{Gauntlet of Fire} \label{Tachyon Arrow}
\noindent
\begin{tabular}{||m{110pt} m{30pt} m{31pt} m{55pt} m{12pt} m{12pt} m{210pt}||}
	\hline
	Name & & Range & Type & S & AP & Abilities \\
	\hline
	\quickref{Gauntlet of Fire} & x pts& Template & Assault 1 & 4 & 5 & — \\
	\quickref{Tachyon Arrow} & x pts& $\infty$ & Assault 1 & 10 & 1 & Armourbane, Destructor, One use \\
	\hline
\end{tabular}



\subsubsection{Gauss Weapons}

\label{Gauss Cannon} \label{Gauss Blaster} \label{Gauss Flayer} \label{Gauss Reaper} \label{Relic Gauss Blaster}
\noindent
\begin{tabular}{||m{110pt} m{30pt} m{31pt} m{55pt} m{12pt} m{12pt} m{210pt}||}
	\hline
	Name & & Range & Type & S & AP & Abilities \\
	\hline
	\quickref{Gauss Cannon} & & 24" & Heavy 3 & 6 & 3 & \quickref{Gauss} (6+) \\
	\quickref{Gauss Flayer} & & 24" & Rapid Fire 1 & 4 & 5 & \quickref{Gauss} (6+) \\
	\quickref{Gauss Reaper} & & 12" & Assault 2 & 5 & 4 & \quickref{Gauss} (6+) \\
	\quickref{Gauss Blaster} & & 24" & Rapid Fire 1 & 5 & 4 & \quickref{Gauss} (6+) \\
	\quickref{Relic Gauss Blaster} & & 30" & Rapid Fire 2 & 5 & 4 & \quickref{Gauss} (6+), Master-Crafted \\	
	\hline
\end{tabular}

\subsubsection{Particle Weapons}

\label{Particle Caster} \label{Particle Beamer}
\noindent
\begin{tabular}{||m{110pt} m{30pt} m{31pt} m{55pt} m{12pt} m{12pt} m{210pt}||}
	\hline
	Name & & Range & Type & S & AP & Abilities \\
	\hline
	\quickref{Particle Caster} & & 12" & Pistol 1 & 6 & 5 & \\
	\quickref{Particle Beamer} & & 24" & Heavy 1 & 6 & 5 & Blast \\	
	\hline
\end{tabular}

\subsubsection{Synaptic Weapons}
\label{Synaptic Disintegrator}
\noindent
\begin{tabular}{||m{110pt} m{30pt} m{31pt} m{55pt} m{12pt} m{12pt} m{210pt}||}
\hline
Name & & Range & Type & S & AP & Abilities \\
\hline
\quickref{Synaptic Disintegrator} & & 36" & Rapid Fire & 5 & 5 & Rending (5+), Pinning, Sniper \\	
\hline
\end{tabular}

\subsubsection{Tesla Weapons}
\label{Tesla Cannon} \label{Tesla Carbine}
\noindent
\begin{tabular}{||m{110pt} m{30pt} m{31pt} m{55pt} m{12pt} m{12pt} m{210pt}||}
	\hline
	Name & & Range & Type & S & AP & Abilities \\
	\hline
	\quickref{Tesla Cannon} & & 30" & Heavy 3 & 6 & — & \quickref{Tesla} (6+) \\
	\quickref{Tesla Carbine} & & 24" & Assault 2 & 5 & — & \quickref{Tesla} (6+) \\	
	\hline
\end{tabular}


\subsubsection{Transdimensional Weapons}
\label{Transdimensional Beamer}
\noindent
\begin{tabular}{||m{110pt} m{30pt} m{31pt} m{55pt} m{12pt} m{12pt} m{210pt}||}
	\hline
	Name & & Range & Type & S & AP & Abilities \\
	\hline
	\quickref{Transdimensional Beamer} & & 12" & Heavy 1 & 4 & 5 & \quickref{Exile Ray} (6+) \\	
	\hline
\end{tabular}


\subsection{Technoarkana} \label{Technoarcana}


\subsubsection{Bloodswarm Scarabs} \label{Bloodswarm Scarabs}

Friendly units of Flayed Ones and Flayer Kings can re-roll the scatter dice when arriving from Deep Strike Reserve. 

\subsubsection{Dispersion Shield} \label{Dispersion Shield}

Grants a 3+ Invulnerable Save, but the wielder never receives +1 Attack for fighting with two Melee weapons.

\subsubsection{Gloom Prism} \label{Gloom Prism}

This model gains the Anathema sub-type. Additionally, any psychic power targeting a unit within 3" of this model is nullified on a 4+. \\

\subsubsection{Mindshackle Scarabs} \label{Mindshackle Scarabs}

At the start of the Fight sub-phase, select a model in base contact with the bearer. The target must take a Leadership Check on 3D6. If the Check is failed, the victim strikes at his allies instead of attacking normally. During the Fight sub-phase, the target makes his attacks against his own unit (automatically hitting if they are the only model in the unit), resolved as normal with any abilities and penalties from his weapons (the controller of the Mindshackle Scarbs chooses which, if there is a choice). If the target is still alive, the victim returns to normal once all blows in that round of combat have been struck.

\subsubsection{Phylactery} \label{Phylactery}

Increase the models It Will Not Die level to 3+.

\subsubsection{Phase Shifter} \label{Phase Shifter}

Grants a 4+ Invulnerable Save.

\subsubsection{Shadow Ankh} \label{Shadow Ankh}

The bearer gains the Anathema sub-type.

\subsubsection{Resurrection Orb} \label{Resurrection Orb}

Once per battle, on your turn, the bearer can activate their Resurrection Orb. If it does, select one friendly unit with \quickref{Reanimation Protocols} within \quickref{Nodal Range}. The bearer of the Orb and the selected unit immediately \textcolor{violet}{\hyperref[Reanimation Protocols]{reassembles}} a number of wounds equal to the total number of missing wounds and number of wounds from all destroyed models.

\subsubsection{Semipternal Weave} \label{Sempiternal Weave}

Increase the model's save to 2+.

\subsubsection{Sepulchral Scarabs} \label{Sepulchral Scarabs}

Increase this model's It Will Not Die level to 3+.

\subsubsection{Tesseract Labyrinth} \label{Tesseract Labyrinth}

One use only.

The bearer can use the Tesseract Labyrinth in lieu of making a Shooting or Close Combat attack this round. Choose a unit within 6". The unit must immediately roll a Wounds Check for each model based on their current remaining wounds or be trapped within the Labyrinth while the Necron Character remains alive.

This can also be used to carry a unit alongside the bearer. Select a unit to start the game inside the Tesseract Labyrinth. They can be disembarked as if they were in a Transport with no Special Abilities following the relevant rules. 

If paired with the \quickref{Mindshackle Scarabs} wargear, this can also be a unit chosen from an enemy faction. The unit is treated as a Distrusted Ally and still take up a relevent Force Org Slot. In narrative games, units that are captured at the end of the game can be included as well. Leave it your opponents as to whether it includes a Force Org Slot or points.

\subsubsection{Translocation Shroud} \label{Translocation Shroud}

The bearer's unit gains the Fleet (2) special rule. When moving, the bearer's unit can move over all other models and terrain as if they were open ground. However, they cannot end their move on top of other models and can only end their move on top of impassable terrain if it is possible to actually place the models on top of it.

\subsubsection{Quantum Shielding} \label{Quantum Shielding}

A vehicle equipped with active quantum shielding counts all of its Front and Side Armour Values as 13. A vehicle’s quantum shielding is active until it suffers a penetrating hit, at which point it immediately
deactivates. For the remainder of the battle after a vehicle’s quantum shielding deactivates, all subsequent hits against that vehicle (including hits made from subsequent shooting attacks in the same phase – either
from a different weapon or a different unit – or hits made at a lower Initiative step in close combat) are treated as though the vehicle was not equipped with quantum shielding.

\subsection{Artefacts of the Aeons} \label{Artefacts of the Aeons}

TODO: This
