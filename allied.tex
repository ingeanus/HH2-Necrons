\usection{Allied Units}

\begin{tabular}{||c c c c c c c c c c c c c c c c||}
	\hline
	\multicolumn{16}{||c||}{Primary Detachment} \\
	\multirow{15}{*}{\begin{sideways}Allied Detachment\end{sideways}} & & \begin{sideways}Charnovokh \end{sideways} & \begin{sideways}Maynarkh \end{sideways} & \begin{sideways}Mephrit \end{sideways} & \begin{sideways}Mephrit-Ghiar \end{sideways} & \begin{sideways}Nephrekh \end{sideways} & \begin{sideways}Nihilakh \end{sideways} & \begin{sideways}Novokh \end{sideways} & \begin{sideways}Sautekh \end{sideways} & \begin{sideways}Szarekhan \end{sideways} & \begin{sideways}Thokt \end{sideways} & \begin{sideways}Triarch \end{sideways} & \begin{sideways}Destroyer Cult \end{sideways} & \begin{sideways}Flayed Ones \end{sideways} & \begin{sideways}Non-Necrons \end{sideways}\\
	& Charnovokh & & \greyskull & \blackskull & \blackskull & \blackskull & \blackskull & \blackskull & \blackskull & \redskull & \blackskull & \blackskull & \greyskull & \redskull & \redskull \\
	& Maynarhk & \greyskull & & \blackskull & \blackskull & \greyskull & \greyskull & \blackskull & \greyskull & \blackskull & \greyskull & \blackskull & \blackskull & \greyskull & \redskull \\
	& Mephrit & \blackskull & \blackskull & & \redskull & \greyskull & \greyskull  & \blackskull & \blackskull & \yellowskull & \blackskull & \blackskull & \greyskull & \redskull & \redskull \\
	& Mephrit-Ghair& \blackskull  & \blackskull & \redskull & & \blackskull & \greyskull & \blackskull & \blackskull & \blackskull & \blackskull & \greyskull & \greyskull & \redskull & \redskull \\
	& Nephrekh & \blackskull & \greyskull & \greyskull & \blackskull & & \blackskull & \blackskull & \blackskull & \blackskull & \blackskull & \blackskull & \greyskull & \redskull & \redskull \\
	& Nihilakh & \blackskull & \greyskull & \greyskull & \greyskull & \blackskull & & \blackskull & \blackskull & \yellowskull & \blackskull & \blackskull & \greyskull & \redskull & \redskull \\
	& Novokh & \blackskull & \blackskull & \blackskull & \blackskull & \blackskull & \blackskull & & \blackskull & \blackskull & \blackskull & \yellowskull & \yellowskull & \redskull & \redskull \\
	& Sautekh & \blackskull & \greyskull & \blackskull & \blackskull & \blackskull & \blackskull & \blackskull & & \redskull & \greyskull & \greyskull & \greyskull & \redskull & \redskull \\
	& Szarekhan & \redskull & \blackskull & \yellowskull & \blackskull & \blackskull & \yellowskull & \blackskull & \redskull & & \yellowskull & \yellowskull & \greyskull & \redskull & \redskull \\
	& Thokt & \blackskull & \greyskull & \blackskull & \blackskull & \blackskull & \blackskull & \blackskull & \greyskull & \yellowskull & & \yellowskull & \greyskull & \redskull & \redskull \\
	& Triarch & \blackskull & \blackskull & \blackskull & \greyskull & \blackskull & \yellowskull & \blackskull & \greyskull & \yellowskull & \yellowskull & & \greyskull & \redskull & \redskull \\
	& Destroyer Cult & \greyskull & \blackskull & \greyskull & \greyskull & \greyskull & \greyskull & \yellowskull & \greyskull & \greyskull & \greyskull & \greyskull & & \greyskull & \redskull \\
	& Flayed Ones & \redskull & \greyskull & \redskull & \redskull & \redskull & \redskull & \redskull & \redskull & \redskull & \redskull & \redskull & \greyskull & & \redskull \\
	& Non-Necrons & \redskull & \redskull & \redskull & \redskull & \redskull & \redskull & \redskull & \redskull & \redskull & \redskull & \redskull & \redskull & \redskull & \\
	\hline
\end{tabular}

\noindent
\yellowskull Sworn Allies

The closest of allies who have fought beside each other many times. The two forces are considered ‘friendly units’ in all regards. This means, for example, that Sworn Brothers may be joined by allied Independent Characters, are treated as friendly units for the targeting of special abilities, Warlord Traits and so on.

Note: Not even Sworn Brothers can embark in allied Transport Vehicles, and rules that affect a particular force owing to its Legiones Astartes special rule do not carry over to Sworn Brother allied units.

\noindent
\blackskull Fellow Warriors

The two forces are willing to fight together for common cause against their foes. Units in your army treat other units at the Fellow Warriors level of Alliance as not being part of the army with the exception that they may not be deliberately targeted, attacked, targeted with special abilities, etc, (note that Blasts and the like may still scatter over allied forces and adversely affect them).

Fellow Warriors cannot benefit from the effects of allied Warlord Traits or be joined by allied Independent Characters, and are not counted as friendly units for the purposes of special abilities. In essence, the two forces fight alongside each other without any additional positive or negative effect.

\noindent
\greyskull Distrusted Allies

The two forces can make common cause against an enemy, but never fully trust each other due to a long-standing feud or inherent antipathy. Models in the allied detachment are treated exactly like Fellow Warriors except that units in this allied detachment are never counted as Scoring units and may not hold Objectives.

\noindent
\redskull By the Phaeron's

The two forces will only ever fight beside each other in the direst of circumstances or by the direct command of their royal lord. The two forces are dealt with as Distrusted Allies but, in addition, whenever a unit is within 6" of a unit that is part of a Faction that falls under this level of alliance then both units reduce their Leadership by -1 until they are no longer within 6" of any unit from that Faction that is part of the same army.


\newpage
\usubsection{Headquarters}

\newpage
\usubsection{Elites}

\usubsubsection{Flayed Ones}

\noindent
\begin{tabular}{||m{10pt} m{95pt} m{30pt} m{11pt} m{11pt} m{11pt} m{11pt} m{11pt} m{11pt} m{11pt} m{11pt} m{11pt} m{11pt} m{125pt}||}
	\hline
	No & Name & & M & WS & BS & S & T & W & I & A & LD & Sv & Type \\
	\hline
	5 & Flayed Ones & X pts & 6" & 4 & 1 & 4 & 4 & 1 & 2 & 3 & 10 & 4+ & Infantry \\
	\hline
	\hline
	\multicolumn{14}{||Z{532 pt}||}{May include up to 15 additional Flayed Ones for X pts/model.}\\	
	\hline
	\hline
	\multicolumn{14}{||Z{532 pt}||}{Wargear: Two Flayer Claws}\\
	\hline
\end{tabular}

\noindent
\begin{tabular}{||m{110pt} m{30pt} m{31pt} m{55pt} m{12pt} m{12pt} m{210pt}||}
	\hline
	Name & & Range & Type & S & AP & Abilities \\
	\hline
	Flayer Claws & X pt & — & Melee & User & — & — \\
	\hline
\end{tabular}

\noindent
\begin{tabular}{||m{532pt}||}
\hline
Abilities \\
\hline
Deep Strike, Fear(2), Hatred (Non-Necrons, Non-Vehicles), Infiltrate, \quickref{Living Metal}, \quickref{Reanimation Protocols} \\
\textbf{Drawn to Blood:} All Flayed One units must start the game in Reserve or Infiltrate. Each time an enemy unit is completely destroyed the Necron player may move one unit of Flayed Ones from Reserve into Ongoing Reserve, even if this means that a unit will appear turn One. \\
\textbf{Curse of Llandu'gor:} This unit's level of alliance is always By the Phaeron's, however do not reduce Flayed One's Leadership from this effect. \\
\hline
\end{tabular}

\newpage
\usubsection{Fast Attack}

\newpage
\usubsection{Heavy Support}