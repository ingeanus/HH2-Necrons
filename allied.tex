\section{Allied Units}

When selecting your units' Dynasties, Destroyer and Flayed One units count as being both Destroyer Cult and the selected Dynasty. Use the worst Level of Alliance between the two.

\begin{tabular}{||c c c c c c c c c c c c c c c c||}
	\hline
	\multicolumn{16}{||c||}{Primary Detachment} \\
	\multirow{15}{*}{\begin{sideways}Allied Detachment\end{sideways}} & & \begin{sideways}Charnovokh \end{sideways} & \begin{sideways}Maynarkh \end{sideways} & \begin{sideways}Mephrit \end{sideways} & \begin{sideways}Mephrit-Ghiar \end{sideways} & \begin{sideways}Nephrekh \end{sideways} & \begin{sideways}Nihilakh \end{sideways} & \begin{sideways}Novokh \end{sideways} & \begin{sideways}Sautekh \end{sideways} & \begin{sideways}Szarekhan \end{sideways} & \begin{sideways}Thokt \end{sideways} & \begin{sideways}Triarch \end{sideways} & \begin{sideways}Destroyer Cult \end{sideways} & \begin{sideways}Flayed Ones \end{sideways} & \begin{sideways}Non-Necrons \end{sideways}\\
	& Charnovokh & & \greyskull & \blackskull & \blackskull & \blackskull & \blackskull & \blackskull & \blackskull & \redskull & \blackskull & \blackskull & \greyskull & \redskull & \redskull \\
	& Maynarhk & \greyskull & & \blackskull & \blackskull & \greyskull & \greyskull & \blackskull & \greyskull & \blackskull & \greyskull & \blackskull & \blackskull & \greyskull & \redskull \\
	& Mephrit & \blackskull & \blackskull & & \redskull & \greyskull & \greyskull  & \blackskull & \blackskull & \yellowskull & \blackskull & \blackskull & \greyskull & \redskull & \redskull \\
	& Mephrit-Ghair& \blackskull  & \blackskull & \redskull & & \blackskull & \greyskull & \blackskull & \blackskull & \blackskull & \blackskull & \greyskull & \greyskull & \redskull & \redskull \\
	& Nephrekh & \blackskull & \greyskull & \greyskull & \blackskull & & \blackskull & \blackskull & \blackskull & \blackskull & \blackskull & \blackskull & \greyskull & \redskull & \redskull \\
	& Nihilakh & \blackskull & \greyskull & \greyskull & \greyskull & \blackskull & & \blackskull & \blackskull & \yellowskull & \blackskull & \blackskull & \greyskull & \redskull & \redskull \\
	& Novokh & \blackskull & \blackskull & \blackskull & \blackskull & \blackskull & \blackskull & & \blackskull & \blackskull & \blackskull & \yellowskull & \yellowskull & \redskull & \redskull \\
	& Sautekh & \blackskull & \greyskull & \blackskull & \blackskull & \blackskull & \blackskull & \blackskull & & \redskull & \greyskull & \greyskull & \greyskull & \redskull & \redskull \\
	& Szarekhan & \redskull & \blackskull & \yellowskull & \blackskull & \blackskull & \yellowskull & \blackskull & \redskull & & \yellowskull & \yellowskull & \greyskull & \redskull & \redskull \\
	& Thokt & \blackskull & \greyskull & \blackskull & \blackskull & \blackskull & \blackskull & \blackskull & \greyskull & \yellowskull & & \yellowskull & \greyskull & \redskull & \redskull \\
	& Triarch & \blackskull & \blackskull & \blackskull & \greyskull & \blackskull & \yellowskull & \blackskull & \greyskull & \yellowskull & \yellowskull & & \greyskull & \redskull & \redskull \\
	& Destroyer Cult & \greyskull & \blackskull & \greyskull & \greyskull & \greyskull & \greyskull & \yellowskull & \greyskull & \greyskull & \greyskull & \greyskull & & \greyskull & \redskull \\
	& Flayed Ones & \redskull & \greyskull & \redskull & \redskull & \redskull & \redskull & \redskull & \redskull & \redskull & \redskull & \redskull & \greyskull & & \redskull \\
	& Non-Necrons & \redskull & \redskull & \redskull & \redskull & \redskull & \redskull & \redskull & \redskull & \redskull & \redskull & \redskull & \redskull & \redskull & \\
	\hline
\end{tabular}

\noindent
\yellowskull Sworn Allies

The closest of allies who have fought beside each other many times. The two forces are considered ‘friendly units’ in all regards. This means, for example, that Sworn Allies may be joined by allied Independent Characters, are treated as friendly units for the targeting of special abilities, Warlord Traits and so on.

Note: Not even Sworn Allies can embark in allied Transport Vehicles, and rules that affect a particular force owing to its Dynasty special rule do not carry over to Sworn Brother allied units.

\noindent
\blackskull Fellow Warriors

The two forces are willing to fight together for common cause against their foes. Units in your army treat other units at the Fellow Warriors level of Alliance as not being part of the army with the exception that they may not be deliberately targeted, attacked, targeted with special abilities, etc, (note that Blasts and the like may still scatter over allied forces and adversely affect them).

Fellow Warriors cannot benefit from the effects of allied Warlord Traits or be joined by allied Independent Characters, and are not counted as friendly units for the purposes of special abilities. In essence, the two forces fight alongside each other without any additional positive or negative effect.

\noindent
\greyskull Distrusted Allies

The two forces can make common cause against an enemy, but never fully trust each other due to a long-standing feud or inherent antipathy. Models in the allied detachment are treated exactly like Fellow Warriors except that units in this allied detachment are never counted as Scoring units and may not hold Objectives.

\noindent
\redskull By the Phaeron's

The two forces will only ever fight beside each other in the direst of circumstances or by the direct command of their royal lord. The two forces are dealt with as Distrusted Allies but, in addition, whenever a unit is within 6" of a unit that is part of a Faction that falls under this level of alliance then both units reduce their Leadership by -1 until they are no longer within 6" of any unit from that Faction that is part of the same army.


\newpage
\subsection{Headquarters}

\subsubsection{Lokhust Lord}

\noindent
\begin{tabular}{||m{10pt} m{95pt} m{30pt} m{11pt} m{11pt} m{11pt} m{11pt} m{11pt} m{11pt} m{11pt} m{11pt} m{11pt} m{11pt} m{125pt}||}
	\hline
	No & Name & & M & WS & BS & S & T & W & I & A & LD & Sv & Type \\
	\hline
	1 & Lokhust Lord & 65 pts & 9" & 4 & 4 & 5 & 6 & 4 & 2 & 4 & 10 & 3+ & Infantry (Anti-Grav, Character, Monstrous, Noble)\\
	\hline
	\hline
	\multicolumn{14}{||Z{532 pt}||}{Wargear: \quickref{Staff of Light}}\\
	\multicolumn{14}{||Z{532 pt}||}{Wargear Options:} \\
	\multicolumn{14}{||Z{532 pt}||}{Wargear Options:} \\	
	\multicolumn{14}{||Z{532 pt}||}{\begin{itemize}
			\item A Lokhust Lord may exchange their \quickref{Staff of Light} for any of the following:
			\begin{itemize}
				\item \quickref{Hyperphase Sword} \hrulefill X pt
				\item \quickref{Rod of Night} \hrulefill X pt
				\item \quickref{Voidblade} \hrulefill 0 pt
				\item \quickref{Warscythe} \hrulefill X pt
				\item \quickref{Warscythe} wuith built-in \quickref{Relic Gauss Blaster} \hrulefill X pt
			\end{itemize}
			\item A Lokhust Lord can take any of the following:
			\begin{itemize}
				\item A \quickref{Gauntlet of Fire} \hrulefill X pt
				\item A \quickref{Tachyon Arrow} \hrulefill X pt
				\item \quickref{Mindshackle Scarabs} \hrulefill X pt
				\item A \quickref{Phase Shifter} \hrulefill X pt
				\item A \quickref{Phylactery} \hrulefill X pt
				\item A \quickref{Resurrection Orb} \hrulefill X pt
				\item A \quickref{Sempiternal Weave} \hrulefill X pt
				\item A \quickref{Tesseract Labyrinth} \hrulefill X pt
			\end{itemize}
			\item A Lokhust Lord can take equipment from the \quickref{Artefacts of the Aeons} List
	\end{itemize}} \\		
\end{tabular}

\noindent
\begin{tabular}{||m{110pt} m{30pt} m{31pt} m{55pt} m{12pt} m{12pt} m{210pt}||}
	\hline
	Name & & Range & Type & S & AP & Abilities \\
	\hline
	\quickref{Staff of Light} (Shooting) & & 18" & Assault 3 & 5 & 3 & — \\
	\quickref{Staff of Light} (Melee) & & — & Melee & User & 3 & Rending (6+) \\
	\quickref{Hyperphase Sword} &  & — & Melee & User & 3 & Rending (5+) \\
	\quickref{Voidblade} &  & — & Melee & User & 4 & \quickref{Entropic Strike} (4+), Rending(6+) \\
	\quickref{Warscythe} &  & — & Melee & +2 & 2 & Armourbane (Melee), Two-Handed \\
	\quickref{Relic Gauss Blaster} & & 30" & Rapid Fire 2 & 5 & 4 & \quickref{Gauss} (6+), Master-Crafted \\
	\quickref{Rod of Night} (Melee) & & — & Melee & User & — & Energy Siphon, Haywire \\
	\quickref{Rod of Night} (Shooting) & & 24" & Assault 2 & 5 & — & Haywire, \quickref{Tesla} (6+) \\
	\hline
\end{tabular}

\noindent
\begin{tabular}{||m{532pt}||}
	\hline
	Abilities \\
	\hline
	\quickref{Annihilation Protocols}, Bulky (2), \quickref{Command Protocols}, \quickref{Nodal Command}(Silver), \quickref{Living Metal}, Preferred Enemy (Non-Necrons), \quickref{Reanimation Protocols} \\
	A Lokhust Lord can take the \quickref{Decurion Nemesor} ability if the prerequisites are met. \\
	\hline
\end{tabular}

\newpage
\subsubsection{Flayer King}

\noindent
\begin{tabular}{||m{10pt} m{95pt} m{30pt} m{11pt} m{11pt} m{11pt} m{11pt} m{11pt} m{11pt} m{11pt} m{11pt} m{11pt} m{11pt} m{125pt}||}
	\hline
	No & Name & & M & WS & BS & S & T & W & I & A & LD & Sv & Type \\
	\hline
	1 & Flayer King & X pts & 7" & 5 & 4 & 5 & 5 & 4 & 2 & 4 & 10 & 3+ & Infantry (Character, Noble)\\
	\hline
	\hline
	\multicolumn{14}{||Z{532 pt}||}{Wargear: \quickref{Staff of Light}}\\
	\multicolumn{14}{||Z{532 pt}||}{Wargear Options:} \\	
	\multicolumn{14}{||Z{532 pt}||}{\begin{itemize}
			\item A Flayer King may exchange their \quickref{Staff of Light} for any of the following:
			\begin{itemize}
				\item \quickref{Hyperphase Sword} \hrulefill X pt
				\item \quickref{Rod of Night} \hrulefill X pt
				\item \quickref{Voidblade} \hrulefill 0 pt
				\item \quickref{Voidscythe} \hrulefill X pt
				\item \quickref{Warscythe} \hrulefill X pt
				\item \quickref{Warscythe} wuith built-in \quickref{Relic Gauss Blaster} \hrulefill X pt
			\end{itemize}
			\item A Flayer King can take any of the following:
			\begin{itemize}
				\item A \quickref{Gauntlet of Fire} \hrulefill X pt
				\item A \quickref{Tachyon Arrow} \hrulefill X pt
				\item \quickref{Bloodswarm Scarabs} \hrulefill X pt
				\item \quickref{Mindshackle Scarabs} \hrulefill X pt
				\item A \quickref{Phase Shifter} \hrulefill X pt
				\item A \quickref{Phylactery} \hrulefill X pt
				\item A \quickref{Resurrection Orb} \hrulefill X pt
				\item A \quickref{Sempiternal Weave} \hrulefill X pt
				\item A \quickref{Shadow Ankh} \hrulefill X pt
				\item A \quickref{Tesseract Labyrinth} \hrulefill X pt
				\item A \quickref{Translocation Shroud} \hrulefill X pt
			\end{itemize}
			\item A Flayer King can take equipment from the \quickref{Artefacts of the Aeons} List
	\end{itemize}} \\	
	
\end{tabular}

\noindent
\begin{tabular}{||m{110pt} m{30pt} m{31pt} m{55pt} m{12pt} m{12pt} m{210pt}||}
	\hline
	Name & & Range & Type & S & AP & Abilities \\
	\hline
	\quickref{Staff of Light} (Shooting) & & 18" & Assault 3 & 5 & 3 & — \\
	\quickref{Staff of Light} (Melee) & & — & Melee & User & 3 & Rending (6+) \\
	\quickref{Hyperphase Sword} &  & — & Melee & User & 3 & Rending (5+) \\
	\quickref{Voidblade} &  & — & Melee & User & 4 & \quickref{Entropic Strike} (4+), Rending(6+) \\
	\quickref{Voidscythe} &  & — & Melee & x2 & 1 & \quickref{Entropic Strike} (2+), Brutal (2), Unwieldy, Two-Handed \\
	\quickref{Warscythe} &  & — & Melee & +2 & 2 & Armourbane (Melee), Two-Handed \\
	\quickref{Relic Gauss Blaster} & & 30" & Rapid Fire 2 & 5 & 4 & \quickref{Gauss} (6+), Master-Crafted \\
	\quickref{Rod of Night} (Melee) & & — & Melee & User & — & Energy Siphon, Haywire \\
	\quickref{Rod of Night} (Shooting) & & 24" & Assault 2 & 5 & — & Haywire, \quickref{Tesla} (6+) \\
	\hline
\end{tabular}

\noindent
\begin{tabular}{||m{532pt}||}
	\hline
	Abilities \\
	\hline
	\quickref{Command Protocols}, Deep-Strike, Fear (2), \quickref{Hyperspace Hunters}, \quickref{Nodal Command}(Gold), \quickref{Living Metal}, \quickref{Reanimation Protocols} \\
	\textbf{Drawn to Blood:} This model must start the game in Reserve or Infiltrate. Each time an enemy unit is completely destroyed the Necron player may move one unit of Flayed Ones with this unit attached from Reserve into Ongoing Reserve, even if this means that a unit will appear turn One. \\
	\textbf{Curse of Llandu'gor:} This does not suffer the penalties for low Levels of Alliance (e.g. the Leadership penalty for \textit{By the Phaeron's}), although its allied units still do. \\
	A Flayer King can take the \quickref{Tesserarion Nemesor} ability if the prerequisites are met. \\
	\hline
\end{tabular}

\newpage
\subsubsection{Skorpekh Lord}

\noindent
\begin{tabular}{||m{10pt} m{95pt} m{30pt} m{11pt} m{11pt} m{11pt} m{11pt} m{11pt} m{11pt} m{11pt} m{11pt} m{11pt} m{11pt} m{125pt}||}
	\hline
	No & Name & & M & WS & BS & S & T & W & I & A & LD & Sv & Type \\
	\hline
	1 & Skorpekh Lord & X pts & 9" & 5 & 5 & 6 & 6 & 4 & 2 & 4 & 10 & 3+ & Infantry (Character, Destroyer, Monstrous, Noble) \\
	\hline
	\hline
	\multicolumn{14}{||Z{532 pt}||}{May include up to 3 additional Skorpekh Destroyers for X pts/model.}\\	
	\hline
	\hline
	\multicolumn{14}{||Z{532 pt}||}{Wargear: Close Combat Weapon, \quickref{Enmitic Annihilator}, \quickref{Hyperphase Harvester}.} \\
	\multicolumn{14}{||Z{532 pt}||}{Wargear Options:} \\	\multicolumn{14}{||Z{532 pt}||}{\begin{itemize}
			\item A Skorpekh Lord can take any of the following:
			\begin{itemize}
				\item \quickref{Mindshackle Scarabs} \hrulefill X pt
				\item A \quickref{Phase Shifter} \hrulefill X pt
				\item A \quickref{Phylactery} \hrulefill X pt
				\item A \quickref{Sempiternal Weave} \hrulefill X pt
				\item A \quickref{Shadow Ankh} \hrulefill X pt
			\end{itemize}
			\item A Skorpekh Lord can take equipment from the \quickref{Artefacts of the Aeons} List
	\end{itemize}} \\
	\hline
\end{tabular}

\noindent
\begin{tabular}{||m{110pt} m{30pt} m{31pt} m{55pt} m{12pt} m{12pt} m{210pt}||}
	\hline
	Name & & Range & Type & S & AP & Abilities \\
	\hline
	\quickref{Enmitic Annihilator} &  & 18" & Assault 1 & 6 & 4 & Blast, Molecular Dissonance \\
	\quickref{Hyperphase Harvester} &  & — & Melee & +2 & 2 & Murderous Strike (4+), Two-Handed, Unwieldy \\
	\hline
\end{tabular}

\noindent
\begin{tabular}{||m{532pt}||}
	\hline
	Abilities \\
	\hline
	\quickref{Annihilation Protocols}, \quickref{Command Protocols}, Bulky (3), Hammer of Wrath (1), \quickref{Living Metal}, \quickref{Nodal Command} (Silver), Preferred Enemy (Non-Necrons), \quickref{Reanimation Protocols} \\
	\hline
\end{tabular}


\newpage
\subsection{Elites}

\subsubsection{Flayed Ones}

\noindent
\begin{tabular}{||m{10pt} m{95pt} m{30pt} m{11pt} m{11pt} m{11pt} m{11pt} m{11pt} m{11pt} m{11pt} m{11pt} m{11pt} m{11pt} m{125pt}||}
	\hline
	No & Name & & M & WS & BS & S & T & W & I & A & LD & Sv & Type \\
	\hline
	5 & Flayed Ones & X pts & 6" & 4 & 1 & 4 & 4 & 1 & 2 & 3 & 10 & 4+ & Infantry \\
	\hline
	\hline
	\multicolumn{14}{||Z{532 pt}||}{May include up to 15 additional Flayed Ones for X pts/model.}\\	
	\hline
	\hline
	\multicolumn{14}{||Z{532 pt}||}{Wargear: Two Flayer Claws}\\
	\hline
\end{tabular}

\noindent
\begin{tabular}{||m{110pt} m{30pt} m{31pt} m{55pt} m{12pt} m{12pt} m{210pt}||}
	\hline
	Name & & Range & Type & S & AP & Abilities \\
	\hline
	Flayer Claws & X pt & — & Melee & User & — & — \\
	\hline
\end{tabular}

\noindent
\begin{tabular}{||m{532pt}||}
\hline
Abilities \\
\hline
Deep Strike, Fear(2), Hatred (Non-Necrons), \quickref{Hyperspace Hunters}, Infiltrate, \quickref{Living Metal}, \quickref{Reanimation Protocols} \\
\textbf{Drawn to Blood:} All Flayed One units must start the game in Reserve or Infiltrate. Each time an enemy unit is completely destroyed the Necron player may move one unit of Flayed Ones from Reserve into Ongoing Reserve, even if this means that a unit will appear turn One. \\
\textbf{Curse of Llandu'gor:} This does not suffer the penalties for low Levels of Alliance (e.g. the Leadership penalty for \textit{By the Phaeron's}), although its allied units still do. \\
\hline
\end{tabular}

\newpage
\subsubsection{Hexmark Destroyer}

\noindent
\begin{tabular}{||m{10pt} m{95pt} m{30pt} m{11pt} m{11pt} m{11pt} m{11pt} m{11pt} m{11pt} m{11pt} m{11pt} m{11pt} m{11pt} m{125pt}||}
	\hline
	No & Name & & M & WS & BS & S & T & W & I & A & LD & Sv & Type \\
	\hline
	1 & Hexmark Destroyer & X pts & 9" & 4 & 6 & 5 & 5 & 3 & 2 & 4 & 10 & 3+ & Infantry (Character, Destroyer, Monstrous) \\
	\hline
	\multicolumn{14}{||Z{532 pt}||}{Wargear: Six \quickref{Enmitic Disintegrator Pistol}s.} \\
	\hline
\end{tabular}

\noindent
\begin{tabular}{||m{140pt} m{00pt} m{31pt} m{55pt} m{12pt} m{12pt} m{210pt}||}
	\hline
	Name & & Range & Type & S & AP & Abilities \\
	\hline
	\quickref{Enmitic Disintegrator Pistol} &  & 18" & Pistol 1 & 6 & 4 & Molecular Deconstruction \\
	\hline
\end{tabular}

\noindent
\begin{tabular}{||m{532pt}||}
	\hline
	Abilities \\
	\hline
	\quickref{Annihilation Protocols}, \quickref{Awakening Protocols} (Silver), Bulky (3), Deep-Strike, \quickref{Hyperspace Hunters}, Firing Protocols (6), \quickref{Living Metal}, Preferred Enemy (Non-Necrons), \quickref{Reanimation Protocols} \\
	\textbf{Ethereal Interception:} This unit is may perform as many additional Deep Strike Assaults as desired and does not have to take part in the initial assault. If this unit is in Deep Strike Reserve, immediately after an enemy unit arrives from Deep Strike Reserve this unit may choose to immediately arrive using the rules for Deep Strike (if this unit does not enter play in this manner, make Reserve Rolls for it as normal in subsequent turns). At the end of that enemy Movement phase, any friendly Deathmarks unit that arrived on the board in this manner during that turn may fire its weapons at any enemy unit that arrived from Reserves that phase; any Deathmarks unit that does so cannot fire its weapons in its following turn.
	\textbf{Multi-Threat Eliminator:} Each time an enemy model is destroyed by a ranged attack made by this model's enmitic disintegrator pistols, after this model makes the rest of its attacks, it can shoot with one of its enmitic disintegrator pistols one additional time. These attacks cannot generate additional attacks. \\
	\textbf{Inescapable Death:} The Hexmark Destroyer has full BS when firing Snap Shots. In addition all of its Weapons gain the Precision Shot (4+) and Ignores cover special rules and it cannot have its BS reduced. \\
	\hline
\end{tabular}

\newpage
\subsubsection{Skorpekh Destroyers}

\noindent
\begin{tabular}{||m{10pt} m{95pt} m{30pt} m{11pt} m{11pt} m{11pt} m{11pt} m{11pt} m{11pt} m{11pt} m{11pt} m{11pt} m{11pt} m{125pt}||}
	\hline
	No & Name & & M & WS & BS & S & T & W & I & A & LD & Sv & Type \\
	\hline
	3 & Skorpekh Destroyers & X pts & 9" & 4 & 4 & 5 & 5 & 2 & 2 & 3 & 10 & 3+ & Infantry (Destroyer, Monstrous) \\
	\hline
	\hline
	\multicolumn{14}{||Z{532 pt}||}{May include up to 3 additional Skorpekh Destroyers for X pts/model.}\\	
	\hline
	\hline
	\multicolumn{14}{||Z{532 pt}||}{Wargear: Two \quickref{Hyperphase Thresher}.} \\
	\multicolumn{14}{||Z{532 pt}||}{Wargear Options:} \\	\multicolumn{14}{||Z{532 pt}||}{\begin{itemize}
			\item Each model may exchange two \quickref{Hyperphase Thresher} for a \quickref{Hyperphase Reap-Blade} \hrulefill +X pts
	\end{itemize}} \\
	\hline
\end{tabular}

\noindent
\begin{tabular}{||m{110pt} m{30pt} m{31pt} m{55pt} m{12pt} m{12pt} m{210pt}||}
	\hline
	Name & & Range & Type & S & AP & Abilities \\
	\hline
	\quickref{Hyperphase Reap-Blade} &  & — & Melee & +2 & 2 & Murderous Strike (5+), Two-Handed \\
	\quickref{Hyperphase Thresher} &  & — & Melee & User & 3 & Reaping Blow (1), Specialist Weapon \\
	\hline
\end{tabular}

\noindent
\begin{tabular}{||m{532pt}||}
	\hline
	Abilities \\
	\hline
	\quickref{Annihilation Protocols}, \quickref{Awakening Protocols} (Silver), Bulky (3), Hammer of Wrath (1), \quickref{Living Metal}, Preferred Enemy (Non-Necrons), \quickref{Reanimation Protocols} \\
	\hline
\end{tabular}



\newpage
\subsection{Fast Attack}

\subsubsection{Triarch Praetorians}

TODO: This

\newpage
\subsubsection{Ophydian Destroyers}
 
\noindent
\begin{tabular}{||m{10pt} m{95pt} m{30pt} m{11pt} m{11pt} m{11pt} m{11pt} m{11pt} m{11pt} m{11pt} m{11pt} m{11pt} m{11pt} m{125pt}||}
	\hline
	No & Name & & M & WS & BS & S & T & W & I & A & LD & Sv & Type \\
	\hline
	3 & Ophydian Destroyers & X pts & 10" & 4 & 4 & 4 & 4 & 2 & 2 & 3 & 10 & 4+ & Infantry (Destroyer, Monstrous) \\
	\hline
	\hline
	\multicolumn{14}{||Z{532 pt}||}{May include up to 3 additional Ophydian Destroyers for X pts/model.}\\	
	\hline
	\hline
	\multicolumn{14}{||Z{532 pt}||}{Wargear: Two \quickref{Hyperphase Thresher}, \quickref{Whip Coils}.} \\
	\multicolumn{14}{||Z{532 pt}||}{Wargear Options:} \\	\multicolumn{14}{||Z{532 pt}||}{\begin{itemize}
			\item Each model may exchange two \quickref{Hyperphase Thresher} for a \quickref{Hyperphase Reap-Blade} \hrulefill +X pts
	\end{itemize}} \\
	\hline
\end{tabular}

\noindent
\begin{tabular}{||m{110pt} m{30pt} m{31pt} m{55pt} m{12pt} m{12pt} m{210pt}||}
	\hline
	Name & & Range & Type & S & AP & Abilities \\
	\hline
	\quickref{Hyperphase Reap-Blade} &  & — & Melee & +2 & 2 & Murderous Strike (5+), Two-Handed \\
	\quickref{Hyperphase Thresher} &  & — & Melee & User & 3 & Reaping Blow (1), Specialist Weapon \\
	\quickref{Whip Coils} & & — & Melee & User & — & Reach (3) \\
	\hline
\end{tabular}

\noindent
\begin{tabular}{||m{532pt}||}
	\hline
	Abilities \\
	\hline
	\quickref{Annihilation Protocols}, \quickref{Awakening Protocols} (Silver), Bulky (3), Deep-Strike, Hammer of Wrath (2), \quickref{Living Metal}, Preferred Enemy (Non-Necrons), \quickref{Reanimation Protocols} \\
	\hline
\end{tabular}



\newpage
\subsection{Heavy Support}

\subsubsection{Lokhust Destroyers}
 
\noindent
\begin{tabular}{||m{10pt} m{95pt} m{30pt} m{11pt} m{11pt} m{11pt} m{11pt} m{11pt} m{11pt} m{11pt} m{11pt} m{11pt} m{11pt} m{125pt}||}
	\hline
	No & Name & & M & WS & BS & S & T & W & I & A & LD & Sv & Type \\
	\hline
	1 & Lokhust Destroyers & X pts & 9" & 4 & 4 & 4 & 4 & 2 & 2 & 3 & 10 & 4+ & Infantry (Anti-Grav, Destroyer, Monstrous) \\
	\hline
	\hline
	\multicolumn{14}{||Z{532 pt}||}{May include up to 5 additional Lokhust Destroyer for X pts/model.} \\	
	\multicolumn{14}{||Z{532 pt}||}{Up to 1 Lokhust Destroyer may be replaced with a Heavy Destroyer X pts.} \\	
	\hline
	\hline
	\multicolumn{14}{||Z{532 pt}||}{Wargear: Each Lokhust Destroyer is equipped with \quickref{Gauss Cannon}. Each Lokhust Heavy Destroyer is equipped with \quickref{Gauss Destructor}.} \\
	\multicolumn{14}{||Z{532 pt}||}{Wargear Options:} \\	\multicolumn{14}{||Z{532 pt}||}{\begin{itemize}
			\item Each Lokhust Heavy Destroyer may exchange \quickref{Gauss Destructor} for an \quickref{Enmitic Exterminator} \hrulefill +X pts
	\end{itemize}} \\
	\hline
\end{tabular}

\noindent
\begin{tabular}{||m{110pt} m{30pt} m{31pt} m{55pt} m{12pt} m{12pt} m{210pt}||}
	\hline
	Name & & Range & Type & S & AP & Abilities \\
	\hline
	\quickref{Enmitic Exterminator} &  & 36" & Heavy 1 & 7 & 4 & Large Blast, Molecular Dissonance \\
	\quickref{Gauss Cannon} &  & 24" & Heavy 3 & 6 & 2 & Gauss (6+) \\
	\quickref{Gauss Destructor} &  & 36" & Heavy 1 & 10 & 1 &  Gauss (6+) \\
	\hline
\end{tabular}

\noindent
\begin{tabular}{||m{532pt}||}
	\hline
	Abilities \\
	\hline
	\quickref{Annihilation Protocols}, \quickref{Awakening Protocols} (Silver), Bulky (2), \quickref{Living Metal}, Preferred Enemy (Non-Necrons), \quickref{Reanimation Protocols} \\
	\hline
\end{tabular}

\newpage
\subsubsection{Lokhust Heavy Destroyers}

\noindent
\begin{tabular}{||m{10pt} m{95pt} m{30pt} m{11pt} m{11pt} m{11pt} m{11pt} m{11pt} m{11pt} m{11pt} m{11pt} m{11pt} m{11pt} m{125pt}||}
	\hline
	No & Name & & M & WS & BS & S & T & W & I & A & LD & Sv & Type \\
	\hline
	1 & Lokhust Heavy Destroyers & X pts & 9" & 4 & 4 & 4 & 4 & 2 & 2 & 3 & 10 & 4+ & Infantry (Anti-Grav, Destroyer, Monstrous) \\
	\hline
	\hline
	\multicolumn{14}{||Z{532 pt}||}{May include up to 2 additional Lokhust Heavy Destroyers for X pts/model.}\\	
	\hline
	\hline
	\multicolumn{14}{||Z{532 pt}||}{Wargear: Each Lokhust Heavy Destroyer is equipped with \quickref{Gauss Destructor}.} \\
	\multicolumn{14}{||Z{532 pt}||}{Wargear Options:} \\	\multicolumn{14}{||Z{532 pt}||}{\begin{itemize}
			\item Each Lokhust Heavy Destroyer may exchange \quickref{Gauss Destructor} for an \quickref{Enmitic Exterminator} \hrulefill +X pts
	\end{itemize}} \\
	\hline
\end{tabular}

\noindent
\begin{tabular}{||m{110pt} m{30pt} m{31pt} m{55pt} m{12pt} m{12pt} m{210pt}||}
	\hline
	Name & & Range & Type & S & AP & Abilities \\
	\hline
	\quickref{Enmitic Exterminator} &  & 36" & Heavy 1 & 7 & 4 & Large Blast, Molecular Dissonance \\
	\quickref{Gauss Destructor} &  & 36" & Heavy 1 & 10 & 1 &  Gauss (6+) \\
	\hline
\end{tabular}

\noindent
\begin{tabular}{||m{532pt}||}
	\hline
	Abilities \\
	\hline
	\quickref{Annihilation Protocols}, \quickref{Awakening Protocols} (Silver), Bulky (2), \quickref{Living Metal}, Preferred Enemy (Non-Necrons), \quickref{Reanimation Protocols} \\
	\hline
\end{tabular}


\newpage
\subsubsection{Triarch Stalker}
TODO: This
