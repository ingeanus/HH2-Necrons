\section[Necron Rules]{\superhuge{Necron Rules}}

\subsection[Special Rules]{\superlarge{Special Rules}}

\begin{multicols}{2}

\subsubsection{Annihilation Protocols} \label{Annihilation Protocols}

Only a single unit with Annihilation Protocol may be taken in armies using the Nodal Command Force Organisation Chart that contains any Fortifications.

\subsubsection{Awakening Protocols (Tier)} \label{Awakening Protocols}

This rule is accompanied by a tier of Bronze, Silver, Gold, and Platinum. Units with this rule can only be included in lists which contain an HQ model with the corresponding \quickref{Nodal Command} tier or higher. Some units have both of these special rules, in which case they cannot satisfy this rules requirements with their own \quickref{Nodal Command} special rule.

\subsubsection{Command Protocols} \label{Command Protocols}

At the start of the game, after both sides have deployed, each unit with this ability may select a Command Protocol that can be used during the game. Many options allow the player to roll a \textbf{Command Protocol check} for additional benefits. To do so, roll a Leadership Check, with success granting the listed effects. Failure causes the listed negative effect immediately.

\begin{tcolorbox}[breakable,enhanced,watermark graphics=white_marble.jpg,watermark opacity=1,watermark color=blue,watermark stretch=1,sharp corners]
\textbf{Protocol of Eternal Guardian}

This model may give up their Shooting Attacks this turn to use this power during the Movement Phase. Select a single friendly Necron unit within \quickref{Nodal Range} that has not moved yet and apply one of the following affects to the unit:

\begin{itemize}
	\itemsep 0pt
	\item The chosen unit improves its Cover Save by +1, or gains a 6+ Cover Save if it has none, but it may not Run this turn.
	\item The chosen unit can re-roll failed Cover Saves.
	\item When the chosen unit is targeted by a Shooting Attack, treat the distance between the attacking unit and the target as 3" longer than the actual distance for Shooting Attacks as long as it is in cover.
\end{itemize}

You may choose to make a Command Protocol check before using this power, if the check is successful two different options may be applied. If the Check is failed, the target unit suffers -1 to all Armour Saves until your next turn.
\end{tcolorbox}
	
\vfill\null
\columnbreak
\begin{tcolorbox}[breakable,enhanced,watermark graphics=white_marble.jpg,watermark opacity=1,watermark color=blue,watermark stretch=1,sharp corners]
\textbf{Protocol of Sudden Storm}

This model may give up their Shooting Attacks this turn to use this power during the Movement Phase. Select a single friendly Necron unit within \quickref{Nodal Range} that has not moved yet and apply one of the following affects to the unit:

\begin{itemize}
	\itemsep 0pt
	\item The chosen unit may immediately move a number of inches up to its unmodified Initiative Characteristic. If the chosen unit has mixed Initiative Characteristics, use the highest unmodified Characteristics.
	\item The chosen unit gains the Move Through Cover special rule and must re-roll failed Dangerous Terrain Tests until your next turn.
	\item The chosen unit's movement does not trigger enemy reactions until the end of your Movement Phase.
\end{itemize}

You may choose to make a Command Protocol check before using this power, if the check is successful two different options may be applied. If the Check is failed, the target unit may only move half their Movement (rounded down) this turn.
\end{tcolorbox}


\begin{tcolorbox}[breakable,enhanced,watermark graphics=white_marble.jpg,watermark opacity=1,watermark color=blue,watermark stretch=1,sharp corners]
\textbf{Protocol of Vengeful Stars}

This model may give up their Shooting Attacks this turn to use this power during the Shooting Phase. Select a single friendly Necron unit within \quickref{Nodal Range} that has not shot yet and apply one of the following affects to the unit:

\begin{itemize}
	\itemsep 0pt
	\item The chosen unit's Ranged Weapons gain the Ignores Cover rule until your next turn.
	\item The chosen unit's Ranged Weapons gain Breaching (6+) or increase the level of Breaching or Rending by 1 until your next turn.
	\item The chosen unit gains Relentless until your next turn.
\end{itemize}

You may choose to make a Command Protocol check before using this power, if the check is successful two different options may be applied. If the Check is failed, the target unit suffers -1 BS this turn.
\end{tcolorbox}

\vfill\null
\columnbreak
\begin{tcolorbox}[breakable,enhanced,watermark graphics=white_marble.jpg,watermark opacity=1,watermark color=blue,watermark stretch=1,sharp corners]
\textbf{Protocol of the Hungry Void}

This model may give up their Shooting Attacks this turn to use this power during the Shooting Phase. Select a single friendly Necron unit within \quickref{Nodal Range} apply one of the following affects to the unit:

\begin{itemize}
	\itemsep 0pt
	\item The chosen unit gains the Counter Charge (1) special rule until your next turn.
	\item The chosen unit's Melee Weapons gain Breaching (6+) or increase the level of Breaching or Rending by 1 until your next turn.
	\item The chosen unit gains the Furious Charge (1) special rule until your next turn.
\end{itemize}

You may choose to make a Command Protocol check before using this power, if the check is successful two different options may be applied. If the Check is failed, the target unit suffers -1 WS this turn.
\end{tcolorbox}

\begin{tcolorbox}[breakable,enhanced,watermark graphics=white_marble.jpg,watermark opacity=1,watermark color=blue,watermark stretch=1,sharp corners]
\textbf{Protocol of the Undying Legions}

This model may give up their Shooting Attacks this turn to use this power during the Shooting Phase. Select a single friendly Necron unit within \quickref{Nodal Range} and apply one of the following affects to the unit:

\begin{itemize}
	\itemsep 0pt
	\item The chosen unit gains a +1 bonus to \quickref{Reanimation Protocols} rolls until the next turn.
	\item The chosen unit's \quickref{Living Metal} ability has its It Will Not Die level increased by 2 levels (e.g. 5+ -> 3+) until  your next turn.
	\item The chosen unit ignores the first unsaved Wound it would take before your next turn.
\end{itemize}

You may choose to make a Command Protocol check before using this power, if the check is successful two different options may be applied. If the Check is failed, the target unit suffers -1 to \textit{Reanimation Protocol} rolls until next turn.
\end{tcolorbox}

\begin{tcolorbox}[breakable,enhanced,watermark graphics=white_marble.jpg,watermark opacity=1,watermark color=blue,watermark stretch=1,sharp corners]
\textbf{Protocol of the Conquering Tyrant}

This model may give up their Shooting Attacks this turn to use this power during the Movement Phase. Select a single friendly Necron unit within \quickref{Nodal Range} and apply one of the following affects to the unit:

\begin{itemize}
	\itemsep 0pt
	\item The chosen unit counts as being in \quickref{Nodal Range} of all units with the \quickref{Command Protocols} special ability until your next turn.
	\item The chosen unit gains the Hit \& Run ability until your next turn.
	\item The chosen unit gains a +2 bonus to Initiative Tests until your next turn.
\end{itemize}

You may choose to make a Command Protocol check before using this power, if the check is successful two different options may be applied. If the Check is failed, the target unit gains the Soulless Hordes (Bronze) trait this turn, if it did not already have it, and cannot ignore the Engrammatic Attack Patterns provision this turn.
\end{tcolorbox}

\subsubsection{Curse of Llandu'gor} \label{Curse of Llandu'gor}

A model with this special rule does not suffer the penalties for low Levels of Alliance (e.g. the Leadership penalty for \textit{By the Phaeron's}), although its allied units still do.

\subsubsection{Drawn to Blood} \label{Drawn to Blood}

A model with this special rule must start the game in Reserve or Infiltrate. Each time an enemy unit is completely destroyed the Necron player may move one unit with the Flayer Sub-type from Reserve into Ongoing Reserve, even if this means that a unit will appear turn One. Independent Characters may be attached to their units when entering.

\subsubsection{Nodal Command (Tier)} \label{Nodal Command}

This rule is accompanied by a tier of Bronze, Silver, Gold, and Platinum. Each tier provides a specified range known as Nodal Range, which many other abilities reference for their effects. Any unit within Nodal Range of this unit may use this model's Leadership in place of its own and units with the \quickref{Soulless Hordes} special rule may also ignore the Engrammatic Attack Patterns provision. The highest tier unit in your army \textit{must} be your Warlord.

\label{Nodal Range}
\begin{tabular}{|c|c|}
	\hline
	Tier & Nodal Range \\
	\hline
	Bronze & 6" \\
	Silver & 9" \\
	Gold & 12" \\
	Platinum & 12" \\
	\hline
\end{tabular}

\subsubsection{Mark of the Flayer} \label{Mark of the Flayer}

If this model or its attached unit destroys an enemy unit during the Assault Phase or fails a Morale check, immediately roll a D6 and apply the result as determined below: \\
\begin{tabular}{Z{11 pt} Z {0.42\textwidth}}
	D6 & Result \\
	1 & \textbf{Berserk:} The model is seized by murderous fury and unable to tell friend from foe. If part of an infantry unit, resolve D3 automatic hits on that unit using the model's weapons. If alone, the model suffers an immediate Wound, with no save allowed. \\
	2-5 & \textbf{In Control:} The model is able to control their madness by sheer force of will, giving no effect. \\
	6 & \textbf{Transfiguration:} The model is transfigured by madness, their auto-repair system distorting their form to express the malignance that consumes them. They gains the Fearless and Rage (1) trait until the end of combat. \\
\end{tabular}


\subsubsection{Entropic Strike (X)} \label{Entropic Strike}

When allocating wounds from a Wound Pool from a weapon with this special rule, lower the target model's Armour Save by 1 (e.g. 3+ -> 4+) permanently before making its save. Against Vehicles, an Armour Penetration roll equal to or greater than the listed value lowers the AV of all facings by 1 permanently before resolving the effects. Vehicles with lowered to an AV of 0 are destroyed and removed from the battlefield.

\subsubsection{Ethereal Interceptors} \label{Ethereal Interceptors}

This unit is may perform a separate Deep Strike Assault. Additionally, it may make use of the \quickref{Ethereal Interception} Advanced Reaction.

\subsubsection{Exile Ray (X)} \label{Exile Ray}

When firing a weapon with this special rule, a To Wound roll equal to or higher than the value listed wounds automatically, regardless of the target's Toughness, has the Instant Death special rule, and the victim must re-roll successful Invulnerable saves. Against Vehicles and Buildings, an Armour Penetration roll equal to or higher than the listed value inflicts a Penetrating Hit.

\subsubsection{Gauss (X)} \label{Gauss}

When firing a weapon with this special rule, a To Wound roll equal to or higher than the value listed wounds automatically, regardless of the target’s Toughness. Against vehicles and buildings, an Armour Penetration roll equal to or higher than the value listed inflicts a Glancing hit.

\subsubsection{Ground Lash} \label{Ground Lash}

When the weapon with this special rule is used to make a Shooting Attack, draw a line 1" inche wide from the model up to the listed range of the weapon — this is the projectile's path.

\begin{itemize}
	\item Each model (friend and enemy) caught in the path (except the firing model) suffers a hit. Models with the Flyer or Skimmer Sub-type are not affected.
	\item Units suffer a hit equal to the number of models caught under the line.
\end{itemize}

\subsubsection{Hyperspace Hunters} \label{Hyperspace Hunters}

A unit with this rule specializes in combat that makes use of movement through hyperspace pocket dimensions. In your Movement Phase, this unit may slip back into hyperspace instead of moving, as long as it did not arrive from Reserves this turn and is not locked in combat or Falling Back. If it does so, remove it from the board and place it into Ongoing Reserves. It can be assigned to a Deep Strike Assault as normal on your Next Player Turn. Models with this rule still make It Will Not Die rolls while placed back into reserves.

\subsubsection{Decurion/Tesserarion Nemesor} \label{Decurion Nemesor} \label{Tesserarion Nemesor}

The Decurion Nemesor / Tesserarion Nemesor special rule grants the following benefits:

\begin{itemize}
	\itemsep 0pt
	\item Rites of War: If a Detachment with the Necron Faction includes at least one model with the Decurion Nemesor special rule then that Detachment may select a single Rite of War. 
	\item Lord of the Legion: An army may only include a maximum of one model with this special rule per 1,000 points. This counts across all Detachments of an army.
	\item A model with this special rule may also include an Immortals, Pariah Lychguard, or Royal Lychguard unit as part of the same Force Organisation slot as the model with the Decurion Nemesor special rule, but must still meet the required \quickref{Awakening Protocols} tier.
\end{itemize}

If multiple units are eligible to take this ability, it can only be taken by those with the highest Nodal Command tier in the force that has not already taken this ability. 

\subsubsection{Path of Annihilation (X)} \label{Path of Annihilation}

When the weapon with this special rule is used to make a Shooting Attack, draw a line a number of inches wide equal to the Level in the brackets from the model up to the listed range of the weapon — this is the projectile's path.

\begin{itemize}
	\item For each model (friend and enemy) caught in the path (except the firing model), roll to hit as usual for a Shooting Attack, with each model suffering a hit if successful. Models with the Flyer Sub-type are not affected unless the controlling player decides to affect \textit{only} models with the Flyer Sub-type.
	\item If a Terrain piece, Building, or model with the Vehicle Unit Type or any model with 6 or more Wounds is successfully hit and does not suffer a Penetrating Hit or unsaved Wound the attack is blocked and its path will go no further than that model. The blocking model will however, suffer D3 additional hits.
	\item If a model with the Vehicle Unit Type and the Transport Sub-type suffers a Penetrating Hit from a weapon with this special rule, each unit Embarked on it suffers D6 hits from the weapon, in addition to any other effects. Any Wounds caused are allocated by the controller of the target unit.
	\item If a model with the Void Shields special rule is successfully hit by this attack and the Void Shield suffers a Penetrating Hit, immediately resolve another hit against the next Void Shield or the model itself if no Void Shields remain until an Armour Penetration roll is failed. If an Armour Penetration roll is failed against a Void shield the attack is blocked and its path will go no further than that model and it suffers no additional hits.
	\item Successful Invulnerable Saves and Feel No Pain Damage Mitigation rolls must be re-rolled. Successful Shrouded Damage Mitigation rolls are considered to have not hit the model.
\end{itemize}

\subsubsection{Quake} \label{Quake}

All units hit by a weapon with the Quake type treat open ground as difficult terrain until the end of their controlling player's next turn.

\subsubsection{Reanimation Protocols} \label{Reanimation Protocols}

Whenever a friendly unit with Reanimation Protocols suffers unsaved wounds or resolves an effect causing wounds, and casualties have been removed, total the number of wounds that have been lost among models that were destroyed and put them into a \textbf{Reassembling Pool} and a second \textbf{Reassembling Pool} for wounds that have the Instant Death special rule.

For each wound in the \textbf{Reassembling Pool}, roll a D6, subtracting 1 for wounds in the Instant Death \textbf{Reassembling Pool}. This unit is \textbf{Reanimating} a wound for every 5+ roll. Each time such a unit \textbf{Reanimates} a wound, perform the following steps:

\begin{itemize}
	\itemsep 0pt
	\item If that unit is less than its Starting Strength, return one destroyed model to play in cohesion range with one wound remaining.
	\item Then, if that unit is at its Starting Strength and has models that are missing any wounds, select a model with the lowest remaining wounds; it regains one lost wound.
\end{itemize}

If this unit would be destroyed by the attack or effect, Reanimation Protocols still triggers; perform the process as normal, however after models have been returned from successful \textbf{Reanimation} rolls, if there are remaining wounds in the Wound Pool, continue allocating those to the newly \textbf{Reanimated} models until the Wound Pool is empty or all \textbf{Reanimation} rolls have failed; these remaining wounds \textit{can} cause further Reanimation Protocols triggers. Do note that effects that simply destroy the unit (e.g. Sweeping Advance) do not trigger Reanimation Protocols.

When assigning wounds to units that have multiple models missing any wounds, assign them to the models with the least amount of wounds. In case of ties, the attacking player decides which models to apply them to.

Certain effects can cause models to immediately begin \textbf{Reassembling} or \textbf{Reanimating}; \textbf{Reassembling} models create a \textbf{Reassembling Pool} equal to the lost wounds of those model and then roll for them as normal. \textbf{Reanimating} models immediately follow the steps for \textbf{reanimating} a number of times equal to the wounds of the destroyed model.

\subsubsection{Shroud of Despair} \label{Shroud of Despair}

Weapons with this special rule make To Wound rolls against the target's modified Leadership rather than Toughness. When attacking models with the Vehicle or Knight Unit-Type, add any Leadership penalties from special rules like Fear to the weapon's Strength. Successful Glancing Hits or Penetrating hits do not inflict lost Hull Points or roll on the Vehicle Damage Table, with a D6 being rolled instead: The model suffers the Crew Shaken effect on a result of 1-3 or the Crew Stunned effect on a 4-6.

\subsubsection{Soulless Hordes (X)} \label{Soulless Hordes}

Models with the Soulless Hordes special rule are subject to the Engrammatic Attack Patterns provision, which has effects dependent on the tier of the subtype:

\paragraph{Bronze}: During the controlling player's Shooting phase and Charge sub-phase, the a unit with the Soulless Hordes special rule must attempt a Shooting Attack if there is an enemy unit within range (they are not forced to charge), and must target the closest enemy unit possible that is within its line of sight and a valid target for a Shooting Attack or Charge. If two or more targets are equally close then the controlling player chooses which will be the target of a Shooting Attack or Charge. Similar to the Fearless rule, a unit with the Soulless Hordes special rule cannot choose to fail a Morale check due to the Our Weapons Are Useless special rule. In addition, a unit with the Soulless Hordes special rule may not make Reactions.

\paragraph{Silver}: During the controlling player's Shooting phase and Charge sub-phase, \textit{if} a unit with the Soulless Hordes special rule attempts a Shooting Attack and/or Charge they must target the closest enemy unit possible that is within its line of sight and a valid target for a Shooting Attack or Charge. If two or more targets are equally close then the controlling player chooses which will be the target of a Shooting Attack or Charge. Similar to the Fearless rule, a unit with the Soulless Hordes special rule cannot use any Reactions that grant a Cover Save, Armour Save or Invulnerable Save, and cannot choose to fail a Morale check due to the Our Weapons Are Useless special rule.

The \quickref{Nodal Command} trait is able to suppress this special rule's effects while in Nodal Range.

\subsubsection{Supersonic} \label{Supersonic}

A Flyer with this special rule may move up to triple its movement when Zooming.

\subsubsection{Teleportation Reserves} \label{Teleportation Reserves}

If a unit in your army has equipment that allows them to interact with Teleportation Reserves — such as the \quickref{Eternity Gate} — you may have any units from your army start out in Teleportation Reserves. Certain abilities and wargear allow you to bring units from Teleportation Reserves onto the Battlefield, move units into Teleportation Reserves, or otherwise interact with them and will be detailed in their respective sections. While in Teleportation Reserves a unit cannot use any Deep-Strike or other Reserve abilities and may only enter play through another unit or effect that would allow it.

In addition, while in Teleportation Reserves a unit is actively repaired by the Tomb World's vast resources and expertise: At the start of your Player Turn a unit in Teleportation Reserves reassembles a number of wounds equal to the amount of lost wounds from any models plus the amount of wounds among destroyed models in the unit.

\subsubsection{Tesla (X)} \label{Tesla}

When firing a weapon with this special rule, a To Hit roll equal to or higher than the value listed generates an additional 2 hits. These hits may be applie to the target unit, or to any unit within 2" of the target unit.

\subsubsection{Their Number is Legion} \label{Their Number is Legion}

When a model with this special rule rolls for \quickref{Reanimation Protocols}, re-roll unmodified results of 1. If a unit with this special rules has at least half of hits models within 6" of an Objective, they instead re-roll unmodified results of 1 or 2. Dynastic Warrior models increase this range by 1.

\subsubsection{Their Name is Death} \label{Their Name is Death}

If a model with this special rule has not moved or Run during the Movement phase of its controlling player’s turn then that model may re-roll failed To Wound rolls or unmodified Armour Penetration rolls of 1 or 2 with Gauss Weapons and failed To Hit rolls with Tesla Weapons.

\subsubsection{Tomb Guardians} \label{Tomb Guardians}

Only a single unit with the Tomb Guardians rule may be used in any army which is built using the Nodal Command Force Organisation Chart and does not contain any Fortifications, should the army contain Fortifications there is no limit.

\end{multicols}

\subsection[Unit Sub-types]{\superlarge{Unit Sub-types}}

\subsubsection{Living Metal} \label{Living Metal}

The following rules apply to all models with the Living Metal Unit Sub-type:

\begin{itemize}
	\item Models with the Living Metal Sub-type have the It Will Not Die (5+) Special rule.
	\item Successful Wounds inflicted by attacks with the Poisoned or Fleshbane special rules must be re-rolled against models with the Living Metal Sub-type.
	\item The Shock Pulse and Disruption special rules affect models with the Living Metal Sub-type.
	\item Models with the Living Metal Sub-type ignore Leadership penalties caused by the Anathema Sub-type.
	\item Models with the Living Metal Sub-type may not make Sweeping Advances, unless a rule specifies otherwise.
	\item Models with the Vehicle Unit-Type and the Living Metal Sub-type ignore the effects of Crew Shaken (but still lose a Hull Point).
	\item Models with Super-Heavy Vehicle Unit-Type and the Living metal Sub-type ignore the effects of any variant of the Lance and Armourbane special rules when resolving attacks made against it and reduce the results of all rolls on the Vehicle Damage Chart caused by weapon without the Destroyer type by 1.
\end{itemize} 

\subsubsection{Canoptek} \label{Canoptek}

Models with the Canoptek subtype gain the Fearless special rule. \\


\subsubsection{C'Tan} \label{C'Tan}

Models with the C'Tan Unit Sub-type gain the following effects:

\begin{itemize}
	\item Models with the C'Tan Unit Sub-type cannot be Pinned.
	\item Models with the C'Tan Unit Sub-type may fire Heavy and Ordnance weapons and counts as Stationary even if it moved in the preceding Movement phase, and may declare Charges as normal regardless of any Shooting Attacks made in the same turn.
	\item Models with the C'Tan Unit Sub-type are not affected by special rules that negatively modify their Characteristics (other than Wounds). Note that this does not affect modifiers to Strength and Toughness conferred by the Daemon Unit Type.
	\item Models with the C'Tan Unit Sub-type have the It Will Not Die (5+) Special rule.
	\item Models with the C'Tan Unit Sub-type may only make Reactions triggered by models with the Armiger, Dreadnought, Primarch, Mechanised or Vehicle Unit Type, or	any model with a Wounds Characteristic of 8 or more and additionally do trigger reactions from models with the Gargantuan Unit Sub-type.
	\item Successful Wounds scored by attacks with the Poisoned (X) or Fleshbane special rules must be re-rolled against models with the C'Tan Unit Sub-type.
	\item Models with the C'Tan Sub-type ignore Leadership penalties caused by the Anathema Sub-type.
	\item A model with the C'Tan Unit Sub-type may attack with all weapons it has in each Shooting Attack it makes including as part of a Reaction.
	\item No model that does not have the C'Tan Unit Sub-type may join a unit that includes a model with the C'Tan Unit Sub-type, and no model with the C'Tan Unit Sub-type may join a unit that does not have the C'Tan Unit Sub-type.
	\item Models with the C'Tan Unit Sub-type have the Eternal Warrior, Fearless, and Night Vision Special Rules.
\end{itemize}

\subsubsection{Destroyer} \label{Destroyer}

The following rules apply to all models with the Destroyer Unit Sub-type:

\begin{itemize}
	\item All models gain the Preferred Enemy (Non-Necrons) special rule.
	\item All models with the Destroyer Sub-type ignore the Living Metal Sub-type's restriction on performing Sweeping Advances. 
\end{itemize}

\subsubsection{Flayer} \label{Flayer}

The following rules apply to all models with the Flayer Unit Sub-type:

\begin{itemize}
	\item All models gain the Hatred (Non-Necrons) special rule.
	\item All models gain the Fear (2) special rule.
	\item All models gain the Deep-Strike and Infiltrate special rule.
	\item All models with the Flayer Sub-type ignore the Living Metal Sub-type's restriction on performing Sweeping Advances.
\end{itemize}

\subsubsection{Floating} \label{Floating}

A unit that includes only models with the Floating Sub-type may ignore the effects of any and all terrain it passes over during movement, including passing over vertical terrain and Impassable Terrain without penalty or restriction. However, such units may not begin or end their movement in Impassable Terrain, and if beginning or ending their movement in Dangerous Terrain must take Dangerous Terrain tests as normal.

\subsubsection{Noble} \label{Noble}

A model with the Noble Sub-type gains the Independent Character special rule. \\

\subsubsection{Phaeron} \label{Phaeron}

The following rules apply to all models with the Phaeron Unit Sub-type:

\begin{itemize}
	\item All Phaerons have the following Special Rules: Eternal Warrior, Fearless, It Will Not Die (3+), Relentless
	\item Phaerons are not affected by special rules that negatively modify their Characteristics (other than Wounds) and, in addition, Phaerons always resolves Snap Shots at their normal BS.
	\item Any Hits inflicted by a Phaeron, as part of either Shooting Attacks or in close combat, are allocated by the Phaeron’s controlling player and not the controlling player of the target unit. These Hits should form a separate Wound Pool.
\end{itemize}
