\usection{Necron Rules}


\usubsubsection{Awakening Protocols (Tier)} \label{Awakening Protocols}

This rule is accompanied by a tier of Bronze, Silver, Gold, and Platinum. Units with this rule can only be included in lists which contain an HQ model with the corresponding \quickref{Command Protocols} tier or lower.

\usubsubsection{Command Protocols} \label{Command Protocols}

At the start of the game, after both sides have deployed, each unit with this ability may select a Command Protocol that can be used during the game. Many options allow the player to roll a \textbf{Command Protocol check} for additional benefits. To do so, roll a Leadership Check, with success granting the listed effects. Failure causes the listed negative effect immediately.

\textbf{Protocol of Eternal Guardian}

This unit may give up their Shooting Attacks this turn to use this power during the Movement Phase. Select a single friendly Necron unit within \quickref{Nodal Range} that has not moved yet and apply one of the following affects to the unit:

\begin{itemize}
	\itemsep 0pt
	\item The chosen unit gains a 6+ Cover Save, but may not Run, until your next turn
	\item 
\end{itemize}

You may choose to make a 

\textbf{Protocol of Sudden Storm}

This unit may give up their Shooting Attacks this turn to use this power during the Movement Phase. Select a single friendly Necron unit within \quickref{Nodal Range} that has not moved yet and apply one of the following affects to the unit:

\begin{itemize}
	\itemsep 0pt
	\item The chosen unit may immediately move a number of inches up to twice its unmodified Initiative Characteristic. If the chosen unit has mixed Initiative Characteristics, use the highest unmodified Characteristics.
	\item The chosen unit ignores Difficult and Dangerous Terrain alongside negative modifiers to their movement until your next turn.
\end{itemize}

You may choose to make a Command Protocol check before using this power, if the check is successful two different options may be applied. If the Check is failed, the target unit may only move half their Movement (rounded down) this turn.

\textbf{Protocol of Vengeful Stars}

This unit may give up their Shooting Attacks this turn to use this power during the Shooting Phase. Select a single friendly Necron unit within \quickref{Nodal Range} that has not shot yet and apply one of the following affects to the unit:

\begin{itemize}
	\itemsep 0pt
	\item The chosen unit's Ranged Weapons gain the Ignores Cover rule until your next turn.
	\item The chosen unit's Ranged Weapons gain Breaching (6+) or increase the level of Breaching or Rending by 1 until your next turn.
	\item The chosen unit gains Relentless until your next turn.
\end{itemize}

You may choose to make a Command Protocol check before using this power, if the check is successful two different options may be applied. If the Check is failed, the target unit suffers -1 BS this turn.

\textbf{Protocol of the Hungry Void}

This unit may give up their Shooting Attacks this turn to use this power during the Shooting Phase. Select a single friendly Necron unit within \quickref{Nodal Range} apply one of the following affects to the unit:

\begin{itemize}
	\itemsep 0pt
	\item The chosen unit gains the Hatred trait until your next turn. 
	\item The chosen unit's Melee Weapons gain Breaching (6+) or increase the level of Breaching or Rending by 1 until your next turn.
	\item 
\end{itemize}

You may choose to make a Command Protocol check before using this power, if the check is successful two different options may be applied. If the Check is failed, the target unit suffers -1 WS this turn.

\textbf{Protocol of the Undying Legions}

This unit may give up their Shooting Attacks this turn to use this power during the Shooting Phase. Select a single friendly Necron unit within \quickref{Nodal Range} and apply one of the following affects to the unit:

\begin{itemize}
	\itemsep 0pt
	\item The chosen unit can re-roll results of 1 once for \quickref{Reanimation Protocols} until next turn. Dynastic Warriors can re-rolls results of 1-2.
	\item The chosen unit's \quickref{Living Metal} ability has its It Will Not Die level increased to 3+ until next turn.
	\item 
\end{itemize}

You may choose to make a Command Protocol check before using this power, if the check is successful two different options may be applied. If the Check is failed, the target unit suffers -1 to \textit{Reanimation Protocol} rolls until next turn.

\textbf{Protocol of the Conquering Tyrant}

This unit may give up their Shooting Attacks this turn to use this power during the Movement Phase. Select a single friendly Necron unit within \quickref{Nodal Range} and apply one of the following affects to the unit:

\begin{itemize}
	\itemsep 0pt
	\item 
	\item The chosen unit gains the Hit \& Run ability until your next turn.
	\item 
\end{itemize}

You may choose to make a Command Protocol check before using this power, if the check is successful two different options may be applied. If the Check is failed, the target unit gains the Soulless Hordes (Bronze) trait this turn, if it did not already have it, and cannot ignore the Engrammatic Attack Patterns provision this turn.

\usubsubsection{Nodal Command (Tier)} \label{Nodal Command}

This rule is accompanied by a tier of Bronze, Silver, Gold, and Platinum. Each tier provides a specified range known as Nodal Range, which many other abilities reference for their effects. Additionally, any unit with the \quickref{Soulless Hordes} sub-type may ignore the Engrammatic Attack Patterns provision while within Nodal Range of a unit with this rule. The highest tier unit in your army \textit{must} be your Warlord.

\label{Nodal Range}
\begin{tabular}{|c|c|}
	\hline
	Tier & Nodal Range \\
	\hline
	Bronze & 6" \\
	Silver & 9" \\
	Gold & 12" \\
	Platinum & 12" \\
	\hline
\end{tabular}


\usubsubsection{Entropic Strike (X)} \label{Entropic Strike}

When allocating wounds from a Wound Pool from a weapon with this special rule, lower the target model's Armour Save by 1 (e.g. 3+ -> 4+) permanently before making its save. Against Vehicles, an Armour Penetration roll equal to or greater than the listed value lowers the AV of all facings by 1 permanently before resolving the effects. Vehicles with lowered to an AV of 0 are destroyed and removed from the battlefield.

\usubsubsection{Gauss (X)} \label{Gauss}

When firing a weapon with this special rule, a To Wound roll equal to or higher than the value listed wounds automatically, regardless of the target’s Toughness. Against vehicles and buildings, an Armour Penetration roll equal to or higher than the value listed inflicts a Glancing hit. Super Heavy and Gargantuan Creatures are immune to this effect.

\usubsubsection{Living Metal} \label{Living Metal}

Infantry and Vehicles with this rule have \textbf{It Will Not Die (5+)}. Successful Wounds inflicted by attacks with the Poisoned or Fleshbane special rules must be re-rolled. The Shock Pulse affects models with this rule as well. Leadership penalties caused by the Anethema sub-type are ignored.


Vehicles ignore the effects of Crew Shaken (but still lose a Hull Point). If the vehicle is Heavy or Super-Heavy, they are not subject to the particular effects of the Lance and Melta special rules by attacks made against it.
In addition, it reduces the effects of all rolls on the Vehicle Damage chart caused by Penetrating hits (other than by Destroyer type weaponry) by -1.

\usubsubsection{Decurion/Tesserarion Nemesor} \label{Decurion Nemesor} \label{Tesserarion Nemesor}

The Decurion Nemesor / Tesserarion Nemesor special rule grants the following benefits:

\begin{itemize}
	\itemsep 0pt
	\item Rites of War: If a Detachment with the Necron Faction includes at least one model with the Decurion Nemesor special rule then that Detachment may select a single Rite of War. 
	\item Lord of the Legion: An army may only include a maximum of one model with this special rule per 1,000 points. This counts across all Detachments of an army.
	\item A model with this special rule may also include an Immortals, Pariah Lychguard, or Royal Lychguard unit as part of the same Force Organisation slot as the model with the Decurion Nemesor special rule, but must still meet the required \quickref{Awakening Protocols} tier.
\end{itemize}

If multiple units are eligible to take this ability, it can only be taken by those with the highest Nodal Command tier in the force that has not already taken this ability. 

\usubsubsection{Reanimation Protocols} \label{Reanimation Protocols}

Whenever a unit with Reanimation Protocols would suffer wounds, after the opponent's unit's attacks or the effects have been resolved and models have been removed total the number of wounds that have been lost among models that were destroyed, the unit begins \textbf{reassembling} a number of wounds equal to this amount. For each wound that is being reassembled, roll a d6, subtracting one for wounds with Instant Death. This unit \textbf{reanimates} a wound for every 5+ roll. Each time such a unit reanimates a wound:

\begin{itemize}
	\itemsep 0pt
	\item If that unit is less than its Starting Strength, return one destroyed model to play in cohesion range with one wound remaining.
	\item Then, if that unit is at its Starting Strength and has models that are missing any wounds, one of the models with the least wounds regains a wound. Controlling player decides in case of ties.
\end{itemize}

When assigning wounds to units that have multiple models missing any wounds, assign them to the models with the lowest amount of wounds. In case of ties, the attacking player decides which models to apply them to.


\usubsubsection{Soulless Hordes (X)} \label{Soulless Hordes}

Models with the Soulless Hordes subtype are subject to the Engrammatic Attack Patterns provision, which has effects dependent on the tier of the subtype:

\paragraph{Bronze:} During the controlling player's Shooting phase and Charge sub-phase, the Soulless Hordes unit must attempt a Shooting Attack if there is an enemy unit within range (they are not forced to charge), and must target the closest enemy unit possible that is within its line of sight and a valid target for a Shooting Attack or Charge. If two or more targets are equally close then the controlling player chooses which will be the target of a Shooting Attack or Charge. Similar to the Fearless rule, units with this subtype cannot use any Reactions that grant a Cover Save, Armour Save or Invulnerable Save, and cannot choose to fail a Morale check due to the Our Weapons Are Useless special rule. They additionally may not make Reactions.

\paragraph{Silver:} During the controlling player's Shooting phase and Charge sub-phase, \textit{if} the unit attempts a Shooting Attack and/or Charge they must target the closest enemy unit possible that is within its line of sight and a valid target for a Shooting Attack or Charge. If two or more targets are equally close then the controlling player chooses which will be the target of a Shooting Attack or Charge. Similar to the Fearless rule, units with this subtype cannot use any Reactions that grant a Cover Save, Armour Save or Invulnerable Save, and cannot choose to fail a Morale check due to the Our Weapons Are Useless special rule.

The \quickref{Command Protocols} trait is able to suppress this sub-type's effects while in Nodal Range.

\usubsubsection{Tesla (X)} \label{Tesla}

When firing a weapon with this special rule, a To Hit roll equal to or higher than the value listed generates an additional 2 hits. These hits may be applie to the target unit, or to any unit within 2" of the target unit.